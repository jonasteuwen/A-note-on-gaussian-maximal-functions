\RequirePackage[l2tabu, orthodox]{nag}
\documentclass[a4paper,oneside,10pt]{amsproc}

\usepackage{fixltx2e}
\usepackage[all, error]{onlyamsmath}
\usepackage{fixmath} % http://ctan.org/pkg/fixmath
\usepackage{refcheck}
% \usepackage{gitinfo} % Information about the version
\norefnames
% \showrefnames
\usepackage{microtype}
\usepackage{amsmath}
\usepackage{amsthm}
\usepackage[x11names]{xcolor}%
\usepackage{textcomp}
\usepackage[english]{babel}
\usepackage{xfrac} % Nice / fractions
\usepackage[utf8]{inputenc}
\usepackage[T1]{fontenc}
\usepackage[strict=true]{csquotes} % Needs to be loaded *after* inputenc

%% Tekstiin ja taulukoihin liittyvi� lis�paketteja
\usepackage{enumerate} 
\usepackage{booktabs}% Better tables

%% Matematiikkaan liittyvi� lis�paketteja

\usepackage{latexsym}
\usepackage{bbm}
\usepackage{enumitem}
\usepackage[bitstream-charter]{mathdesign}%

\usepackage[bookmarks,colorlinks,breaklinks]{hyperref} % Add hyperref type links in the document, colors
\definecolor{dullmagenta}{rgb}{0.4,0,0.4} % #660066
\definecolor{darkblue}{rgb}{0,0,0.4}%
\hypersetup{linkcolor=red,citecolor=blue,filecolor=dullmagenta,urlcolor=darkblue} % coloured links


\usepackage{etoolbox}
\newcounter{hints}
\renewcommand{\thehints}{\roman{hints}}
\newcommand{\hintedrel}[2][]{%
  \stepcounter{hints}%
  \if\relax\detokenize{#1}\relax\else\csxdef{hint@#1}{\thehints}\fi
  \mathrel{\overset{\textrm{(\thehints)}}{\vphantom{\le}{#2}}}%
}
\newcommand{\restarthintedrel}{\setcounter{hints}{0}}
\newcommand{\hintref}[1]{\csuse{hint@#1}}
\AfterEndEnvironment{align*}{\setcounter{hints}{0}}% Resets numrel at the end of align*
\AfterEndEnvironment{align}{\setcounter{hints}{0}}% Resets numrel at the end of align
\AfterEndEnvironment{equation*}{\setcounter{hints}{0}}% Resets numrel at the end of equation*
\AfterEndEnvironment{equation}{\setcounter{hints}{0}}% Resets numrel at the end of equation

\makeatletter

\@namedef{subjclassname@2010}{%

  \textup{2010} Mathematics Subject Classification}

\makeatother

% \linespread{1.3}


%%%%%%%% Teoreemaymp�rist�t
% HUOM! K�ytet��n amsthm-pakettia, eik� theorem-pakettia, jos
% theorem-pakettia k�ytet��n t�ytyy my�s n�it� muuttaa

% Perusteoreematyyli, lause ja lemma numeroidaan sectionin
% mukaan. Propositio numeroidaan lauseen numeroinnilla

\theoremstyle{plain}
% \newtheorem*{thmA6.6}{[A, Theorem 6.6]}
\newtheorem*{problem}{Problem}


\swapnumbers
\newtheorem{theorem}{Theorem}
\newtheorem{definition}{Definition}
\newtheorem{lemma}{Lemma}
\newtheorem{corollary}{Corollary}
\newtheorem{proposition}{Proposition}
\theoremstyle{remark}
\renewcommand{\qedsymbol}{\ensuremath{\blacksquare}}
\newtheorem*{remark}{Remark}
\newtheorem*{examples}{Examples}


% M��ritelm�tyyli

\theoremstyle{definition} 
% M��ritelm�, jolla on oma juokseva numerointi 
% \newtheorem{definition}[theorem]{Definition}
\newtheorem*{def*}{Definition}

% Huomautustyyli


% Omat k�skyt (esimerkkin� joukkosymbolit)

\newcommand{\E}{\mathbb{E}}
\newcommand{\Sp}{\mathbb{S}}
\newcommand{\prob}{\mathbb{P}}
\newcommand{\D}{\,\textup{d}}
\newcommand{\Dn}{\textup{d}} % One without space.
\newcommand{\Dt}{\,\frac{\textup{d} t}{t}}
\newcommand{\Ds}{\,\frac{\textup{d} s}{s}}
\newcommand{\DyDt}{\frac{\textup{d} y \, \textup{d} t}{t^{n+1}}}
\newcommand{\Dd}{\mathscr{D}}
\newcommand{\Tt}{\mathscr{T}}
\newcommand{\Cc}{\mathscr{C}}
\newcommand{\Bb}{\mathscr{B}}
\newcommand{\Rr}{\mathscr{R}}
\newcommand{\Ff}{\mathscr{F}}
\newcommand{\Ll}{\mathscr{L}}
\newcommand{\Mm}{\mathscr{M}}
\newcommand{\hh}{\mathfrak{h}}
\newcommand{\ttt}{\mathfrak{t}}
\newcommand{\la}{\langle}
\newcommand{\ra}{\rangle}
\newcommand{\Rad}{\textup{Rad}}
\newcommand{\Car}{\textup{Car}}
\newcommand{\BMO}{\textup{BMO}}
\newcommand{\loc}{\textup{loc}}
\newcommand{\LH}{{L^2_\mu}}
\newcommand{\LHG}{{L^2_\gamma}}

%% Bar int
\def\Xint#1{\mathchoice
  {\XXint\displaystyle\textstyle{#1}}%
  {\XXint\textstyle\scriptstyle{#1}}%
  {\XXint\scriptstyle\scriptscriptstyle{#1}}%
  {\XXint\scriptscriptstyle\scriptscriptstyle{#1}}%
  \!\int}
\def\XXint#1#2#3{{\setbox0=\hbox{$#1{#2#3}{\int}$ }
    \vcenter{\hbox{$#2#3$ }}\kern-.535\wd0}}
\def\ddashint{\Xint=}
\def\dashint{\Xint-}

\def\LI{{L^1_\gamma}}


%% Symbols
\renewcommand{\bar}[1]{\overline#1}
\renewcommand{\vec}[1]{\boldsymbol{\mathbf{#1}}}
\renewcommand{\leq}{\leqslant}
\renewcommand{\Im}{\operatorname{Im}}
\renewcommand{\Re}{\operatorname{Re}}
\renewcommand{\leq}{\leqslant}
\renewcommand{\geq}{\geqslant}
\renewcommand{\epsilon}{\varepsilon}
\renewcommand{\emptyset}{\varnothing}
\newcommand{\Fo}{\mathcal{F}}
\newcommand{\R}{\mathbf R}
\newcommand{\C}{\mathbf C}
\newcommand{\N}{\mathbf N}
\newcommand{\T}{\mathbb T}
\newcommand{\Z}{\mathbf Z}
\newcommand{\B}{\mathcal B}
\newcommand{\e}{\mathrm{e}} %Roman e for exponentials

\renewcommand{\leq}{\leqslant}%
\renewcommand{\geq}{\geqslant}%
\DeclareMathOperator{\supp}{supp}
\newcommand{\Dg}{\frac{\textup{d}\gamma (y)}{\gamma (B(y,t))}}
\newcommand{\Dmu}{\frac{\textup{d}\mu (y)}{\mu (B(y,t))}}


\renewcommand{\Re}{\operatorname{Re}}
\renewcommand{\Im}{\operatorname{Im}}
\renewcommand{\bar}{\overline}

\usepackage{tikz}

\def\lemmaautorefname{Lemma}
\def\definitionautorefname{Definition}
\def\theoremautorefname{Theorem}
\def\corollaryautorefname{Corollary}


%% Bibliography
\usepackage[backend=biber,doi=false,url=false,isbn=false]{biblatex}
\bibliography{~/Documents/BibTeX/library.bib}

\newbibmacro{string+doiurlisbn}[1]{%
  \iffieldundef{doi}{%
    \iffieldundef{url}{%
      \iffieldundef{isbn}{%
        \iffieldundef{issn}{%
          #1%
        }{%
          \href{http://books.google.com/books?vid=ISSN\thefield{issn}}{#1}%
        }%
      }{%
        \href{http://books.google.com/books?vid=ISBN\thefield{isbn}}{#1}%
      }%
    }{%
      \href{\thefield{url}}{#1}%
    }%
  }{%
    \href{http://dx.doi.org/\thefield{doi}}{#1}%
  }%
}

\DeclareFieldFormat{title}{\usebibmacro{string+doiurlisbn}{\mkbibemph{#1}}}
\DeclareFieldFormat[article,incollection]{title}%
{\usebibmacro{string+doiurlisbn}{\mkbibquote{#1}}}


\title[Gaussian maximal functions]{A note on the Gaussian maximal function}
%\author{Mikko Kemppainen}
%\address{Department of Mathematics and Statistics, University of Helsinki,
%  Gustaf H�llstr�min katu 2b, FI-00014 Helsinki, Finland}
%\email{mikko.k.kemppainen@helsinki.fi}

\author{Jonas Teuwen}%
\address{Delft Institute of Applied Mathematics,
  Delft University of Technology, P.O. Box 5031, 2600 GA Delft, The
  Netherlands} \email{j.j.b.teuwen@tudelft.nl}%
\urladdr{http://fa.its.tudelft.nl/~teuwen/}%
\thanks{}%
\date{\today}


\begin{document}
\begin{abstract}
  In this note we give an improvement on a result first demonstrated
  by \textcite{Pineda2008}. In particular we present an improvement to
  their Lemma 1.1 which gives the boundedness of the
  Gaussian maximal function associated to the Ornstein-Uhlenbeck
  operator.

  We present a proof which is at least to the author more transparant.
  Our main finding in this note is that our proof allows to use a
  larger cone and actually obtain the maximal function boundedness for
  a whole class of cones $\Gamma^{(A, a)}_x(\gamma)$.
\end{abstract}

% \subjclass[2010]{42B25 (Primary); 46E40 (Secondary)}
% \keywords{R-bounds, dyadic cubes}

\maketitle
\section{Introduction}
Maximal functions are one of the most studied objects in harmonic
analysis. 
For instance, the classical real-valued harmonic analytic maximal
function can be seen to be of the form
\begin{equation*}
  \sup_{(y, t) \in \Gamma_x} |\e^{-t^2 \Delta} u(y)| \lesssim \sup_{r
    > 0}  \dashint_{B_r(x)} |u| \D\lambda.
\end{equation*}
\textbf{(verify)}
In this note we are interested in gaussian harmonic analysis
which is harmonic analysis with respect to the gaussian measure as
opposed to the Lebesgue measure in classical real-valued harmonic
analysis. As the gaussian maximal function will be of our main
interest we 

The gaussian maximal function will be our main interest.
Formally we are looking for 

We define the \emph{Gaussian measure} as:
\begin{equation}
  \label{eq:Gaussian-measure}
  \mathrm{d}\gamma(x) := \frac{\e^{-|x|^2}}{\pi^{\frac{d}2}} \D{x}.
\end{equation}



\textcite{Pineda2008}

\subsection{Notation}
To begin, let us fix some notation. As is common, we use $N$ to
represent a positive integer. That is, $N \in \Z_+ = \{1, 2, 3,
\dots\}$. In the same way we cast letters that denote the number of
dimensions, e.g.\ $d \in \Z_+$ as positive integers.

We use the capital letter $T$ to denote a ``time'' endpoint, for
instance, when writing $t \in (0, T]$.

% How do cone scalings behave wrt several maximal functions?
% Especially for the latter.
%
% Need to add domain and measure in first section, but that can be in
% the introduction later on.
%
%

\section{The Mehler kernel}
\subsection{Setting}
We are concerned with the \emph{Ornstein-Uhlenbeck} operator $L$ which
is defined as:
\begin{equation}
  \label{eq:Ornstein-Uhlenbeck-operator}
  L := -\frac12 \Delta + x \cdot \nabla,
\end{equation}
We define the Mehler kernel (see e.g., \textcite{Sjogren1997}) as the
Schwartz kernel associated to the Ornstein-Uhlenbeck semigroup
$(\e^{-tL})_{t \geq 0}$. More precisely, this means:
\begin{equation}
  \label{eq:Ornstein-Uhlenbeck-semigroup-integral}
  \e^{-tL} u(x) = \int_{\R^d} M_t(x, \cdot) u \, \D\gamma.
\end{equation}
It is often more convenient to use $\e^{-t^2 L}$ instead of $\e^{-tL}$
as is done in e.g., \textcite{Portal2012}.

\subsection{The Mehler kernel}
For the calculation of the Mehler kernel $M_t$ in
\eqref{eq:Ornstein-Uhlenbeck-semigroup-integral} we refer to e.g.,
\textcite{Sjogren1997} which additionally offers related results such
as those related to Hermite polynomials.

The kernel $M_t$ is invariant under the permutation $x
\leftrightarrow y$. A formula for $M_{t^2}$ which honors this
observation is:
\begin{equation}
  \label{eq:Mehler-kernel}
  M_t(x, y) = \frac{\exp\biggl(-\e^{-2t} \dfrac{|x - y|^2}{1
      - \e^{-2 t}}  \biggr)}{(1 - \e^{-t})^{\frac{d}2}}
  \frac{\exp\biggl(2\e^{-t} \dfrac{\la x, y \ra}{1 + \e^{-t}}
    \biggr)}{(1 + \e^{-t})^{\frac{d}2}}.
\end{equation}

\section{Some lemmata and definitions}
\subsection{$m$inimal function}
We recall the lemma from \cite[lemma 2.3]{Maas2011b} which first
--although implicitly-- appeared in \cite{Mauceri2007}. For what
follows it will be convenient to define a function $m$ as:
\begin{equation*}
  m(x) := \min\biggl\{1, \frac1{|x|} \biggr\} = 1 \wedge \frac1{|x|}.
\end{equation*}
We use $m$ in our next lemma.
\begin{lemma}\label{lem:m-xy-equivalence}%\label{lem:m-xy-equivalence-no-cone}
  Let $a, A$ be strictly positive real numbers and $t > 0$. We have
  for $x, y \in \R^d$ that:
  \begin{enumerate}
  \item If $|x - y| < A t$ and $t \leq a m(x)$, then $t
    \leq (1 + aA) m(y)$;
  \item If $|x - y| < A m(x)$, then $m(x) \leq (1 +
    A) m(y)$ and $m(y) \leq 2 (1 + A) m(x)$. 
  \end{enumerate}
\end{lemma}
We define:
\begin{equation}
  \label{eq:Gaussian-cone}
  \Gamma_x^{(A, a)} := \Gamma_x^{(A, a)}(\gamma) := \{(y, t) \in
  \R^d_+ : |x - y| < At \:\text{and}\: t \leq a m(x)\}.
\end{equation}
To ease the notational burden a bit, we will write $\Gamma_x^a$ and
mean $\Gamma_x^{(1, a)}$. 
The next lemma will come useful when we want to cancel exponential
growth in one variable with exponential decay in the other as long
both variables are in a Gaussian cone.
\begin{lemma}\label{lem:Cone-Gaussians-comparable}
  Let $\alpha > 0$ and $|x - y| \leq \alpha m(x)$. We get the
  equivalence:
  \begin{equation*}
    \e^{-\alpha^2(1 + \alpha)^2} \e^{-2\alpha(1 + \alpha)} \e^{-|y|^2}
    \leq \e^{-|x|^2} \leq \e^{\alpha^2} \e^{2\alpha} \e^{-|y|^2}.
  \end{equation*}
\end{lemma}
\begin{proof}
  By the inverse triangle inequality we get, 
  \begin{align*}
    |y|^2 &\leq (\alpha m(x) + |x|)^2\\
    &= \alpha^2 + 2 \alpha m(x) |x| + |x|^2\\
    &\leq \alpha^2 + 2 \alpha + |x|^2.
  \end{align*}
  Therefore,
  \begin{equation}
    \label{eq:Cone-Gaussians-comparable-1}
    \e^{-|x|^2} \leq \e^{-|y|^2} \e^{\alpha^2} \e^{2\alpha}.
  \end{equation}
  For the reverse direction we use
  \autoref{lem:m-xy-equivalence} to infer $m(x) \leq (1 + \alpha)
  m(y)$. Proceeding as before we obtain: 
  \begin{equation*}
    |x|^2 \leq \alpha^2 (1 + \alpha)^2 + 2 \alpha (1 + \alpha) + |y|^2.
  \end{equation*}
  Therefore,
  \begin{equation}
    \label{eq:Cone-Gaussians-comparable-2}
    \e^{-|y|^2} \leq \e^{-|x|^2} \e^{\alpha^2 (1 + \alpha)^2} \e^{2
      \alpha (1 + \alpha)}.
  \end{equation}
  Combining we get:
  \begin{equation*}
    \e^{-\alpha^2(1 + \alpha)^2} \e^{-2\alpha(1 + \alpha)} \e^{-|y|^2}
    \overset{\eqref{eq:Cone-Gaussians-comparable-2}}{\leq} \e^{-|x|^2}
    \overset{\eqref{eq:Cone-Gaussians-comparable-1}}{\leq}
    \e^{\alpha^2} \e^{2\alpha} \e^{-|y|^2}.
  \end{equation*}
  As required.
\end{proof}

\section{On-diagonal estimates}
\subsection{Kernel estimates}
We begin with a technical lemma which will be useful on several
occasions.
\begin{lemma}\label{lem:Time-part-Mehler-time-transform}
  Let $t > 0$ and let $\alpha > 2$. Then,
  \begin{equation}
    \label{eq:Time-part-Mehler-time-transform-1}
    \alpha \e^{-\frac{t^2}{\alpha}} \leq \frac{1 - \e^{-t^2}}{1 -
      \e^{-\frac{t^2}{\alpha}}} \leq \alpha,
  \end{equation}
  and
  \begin{equation}
    \label{eq:Time-part-Mehler-time-transform-2}
       0 \leq \frac1{t^2} \biggl[\frac{1}{1 + \e^{-t^2}} - \frac{1}{1 +
         \e^{-\frac{t^2}{\alpha}}} \biggr] \leq \frac{1}{4} \biggl(1 -
       \frac1\alpha \biggr) \leq \frac18.
  \end{equation}
\end{lemma}
\begin{proof}
  We start with \eqref{eq:Time-part-Mehler-time-transform-1} and apply
  the mean value theorem to the function $f(\xi) = \xi^\alpha$. For $0
  < \xi < \xi'$ this gives that:
  \begin{equation*}
    f(\xi) - f(\xi') = \alpha \hat{\xi}^{\alpha - 1} (\xi - \xi')
    \text{ for some $\hat \xi$ in $[\xi, \xi']$}.
  \end{equation*}
  Picking $\xi = 1$ and $\xi' = \e^{-\frac{t^2}{\alpha}}$ yields:
  \begin{equation}
    \label{eq:Time-part-Mehler-time-transform-proof-1}
    \frac{1 - \e^{-t^2}}{1 - \e^{-\frac{t^2}{\alpha}}} = \alpha
    \hat{\xi}^{\alpha - 1} \:\text{for some}\: \hat{\xi} \:\text{in}\:
    \biggl[\exp \Bigl(-\frac{t^2}{\alpha} \Bigr), 1 \biggr].
  \end{equation}
  Combining this result with the monotonicity of $\xi \mapsto
  \alpha \xi^{\alpha - 1}$ we obtain:
  \begin{equation*}
    \alpha \e^{-\frac{t^2}\alpha} \leq \alpha \exp\biggl(-t^2 \frac{\alpha -
      1}{\alpha} \biggr) \leq \frac{1 - \e^{-t^2}}{1 -
      \e^{-\frac{t^2}{\alpha}}} \leq \alpha,
  \end{equation*}
  where the last bound follows from the monotonicity together with the
  limit as $t \downarrow 0$.

  We proceed with \eqref{eq:Time-part-Mehler-time-transform-2}.
  Recalling that $\alpha > 1$ one can directly verify that the
  function
  \begin{equation*}
    \frac1{t^2} \biggl[\frac{1}{1 + \e^{-t^2}} - \frac{1}{1 +
      \e^{-\frac{t^2}{\alpha}}} \biggr]
  \end{equation*}
  is non-negative and decreasing in $t$. To find
  an upper bound we compute the limit as $t$ goes to $0$. That is:
  \begin{align*}
    \lim_{t \to 0} \frac1{t^2}\biggl[\frac{1}{1 +
      \e^{-t^2}} - \frac{1}{1 + \e^{-\frac{t^2}{\alpha}}} \biggr] 
    &= \lim_{t \to 0} \frac1{2t} \biggl[\frac{2t \e^{-t^2}}{(1 +
      \e^{-t^2})^2} - \frac1\alpha \frac{2t
      \e^{-\frac{t^2}{\alpha}}}{(1 + \e^{-\frac{t^2}{\alpha}})^2}
    \biggr]\\ 
    &= \lim_{t \to 0} \biggl[\frac{\e^{-t^2}}{(1 + \e^{-t^2})^2} -
    \frac1\alpha \frac{\e^{-\frac{t^2}{\alpha}}}{(1 +
      \e^{-\frac{t^2}{\alpha}})^2} \biggr]\\
    &\uparrow \frac{1}{4} \biggl(1 - \frac1\alpha \biggr).
  \end{align*}
  Which is as asserted and completes the proof.
\end{proof}

The following lemma will be useful when transfering estimates from
$M_{\frac{t^2}{\alpha}}$ to $M_{t^2}$. It follows from the mean value
theorem applied to $\xi \mapsto \xi^\alpha$.
\begin{lemma}\label{lem:Exponential-estimates}
  For $\alpha > 1$ and $t$ in $(0, T]$ and all let $x, y$ in $\R^d$
  we have that:
  \begin{equation}
    \label{eq:Exponential-estimates-1}
    \exp \biggl (-\frac12\frac{|x - y|^2}{1 - \e^{-\frac{t^2}\alpha}}
    \biggr ) \leq  \exp \biggl (-\frac12\frac{\alpha}{\e^{T^2}} \frac{|x -
      y|^2}{1 - \e^{-t^2}} \biggr ).
  \end{equation}
\end{lemma}
\begin{proof}
First note that
\begin{equation*}
  \frac1{2 \e^{2T^2}} \leq \frac{\e^{-2t^2}}{1 + \e^{-t^2}} \leq \frac12.
\end{equation*}
Therefore,
\begin{align*}
 \exp\biggl(-\e^{-2t^2} \dfrac{|x - y|^2}{1 - \e^{-2 t^2}} \biggr)
 &\leq \exp\biggl(-\dfrac{\e^{-2t^2}}{1 + \e^{-t^2}} \dfrac{|x -
   y|^2}{1 - \e^{-t^2}} \biggr)\\
 &\leq \exp\biggl(-\frac1{2\e^{2T^2}} \dfrac{|x - y|^2}{1 - \e^{-t^2}} \biggr)
\end{align*}




  Let $t > 0$. Applying
  \eqref{eq:Time-part-Mehler-time-transform-1} we get:
  \begin{align*}
    \exp \biggl (-\e^{-\frac{t^2}\alpha} \frac{|x - y|^2}{1 - \e^{-2\frac{t^2}\alpha}}
    \biggr ) &= \exp \biggl (-\frac1{2\e^{2\frac{T^2}{\alpha}}} \frac{1
      - \e^{-t^2}}{1 - \e^{-\frac{t^2}\alpha}}  \frac{|x - y|^2}{1 - \e^{-t^2}} \biggr )\\
    &\leq \exp \biggl(-\frac{\alpha}{2\e^{2\frac{T^2}{\alpha}}}
    \frac{\alpha}{\e^{T^2}} \frac{|x - y|^2}{1 - \e^{-t^2}} \biggr).
  \end{align*}
  Which is as asserted.
\end{proof}
Our first lemma is about estimating $M_{\frac{t^2}\alpha}$ in terms of
$M_{t^2}$.
\subsubsection{Time-scaling of the Mehler kernel}
\begin{lemma}\label{lem:Kernel-estimates-1}
 Let $T > 0$, $\alpha \geq 2 \e^{T^2}$, $t$ in $(0, T]$ and $x, y \in
 \R^d$. Then:
  \begin{equation}
    \label{eq:Kernel-lemma-1-estimate} 
    M_{\frac{t^2}{\alpha}}(x, y) \leq \alpha^{\frac{d}2}
    \e^{\frac{t^2}4 |\la x, y \ra|} \exp\biggl (-\frac{\alpha}{2
      \e^{T^2}} \frac{|x - y|^2}{1 - \e^{-t^2}} \biggr ) M_{t^2}(x,
    y).
  \end{equation}
\end{lemma}
\begin{proof}
  To prove the lemma we compute $M_{\frac{t^2}{\alpha}} M_{t^2}^{-1}$.
  First note that \eqref{eq:Time-part-Mehler-time-transform-1} gives
  \begin{equation*}
        \alpha^{\frac{d}2} \e^{-\frac{d}2 T^2} \leq \frac{(1 - \e^{-t^2})^{\frac{d}2}}{(1 -
          \e^{-\frac{t^2}{\alpha}})^{\frac{d}2}} \leq \alpha^{\frac{d}2}.
  \end{equation*}
  Combining the exponentials also gives,
  \begin{align*}
    \exp \biggl (-2\e^{-\frac{t^2}\alpha} \frac{\la x, y \ra}{1 + \e^{-\frac{t^2}{\alpha}}}
    \biggr ) &\exp \biggl (2 \e^{-t^2} \frac{\la x, y \ra}{1 + \e^{-t^2}}
    \biggr )\\
    &\stackrel{\phantom{\eqref{eq:Time-part-Mehler-time-transform-2}}}{=}
    \biggl (\frac2{t^2}\biggl[\frac{1}{1 + \e^{-t^2}} - \frac{1}{1 +
      \e^{-\frac{t^2}{\alpha}}} \biggr] t^2 \la x, y \ra \biggr)\\
    &\stackrel{\eqref{eq:Time-part-Mehler-time-transform-2}}{\leq}
    \exp\biggl (\frac{t^2}4 |\la x, y \ra| \biggr).
  \end{align*}

  Combining \autoref{lem:Exponential-estimates} and equation
  \eqref{eq:Exponential-estimates-1} gives is almost the final estimate.
  \begin{align*}
    \frac{M_{\frac{t^2}{\alpha}}(x, y)}{M_{t^2}(x, y)} &\leq
    \alpha^{\frac{d}2} \e^{\frac{t^2}4 |\la x, y \ra|} \exp\biggl(\dfrac12 \dfrac{|x - y|^2}{1 -
      \e^{-t^2}} \biggr) \exp\biggl(-\dfrac12 \dfrac{|x - y|^2}{1
      - \e^{-\frac{t^2}{\alpha}}}  \biggr)\\
    &\leq \alpha^{\frac{d}2} \e^{\frac{t^2}4 |\la x, y \ra|} \exp \biggl (\frac12 \biggl[1
    -\frac{\alpha}{2\e^{T^2}} \biggr] \frac{|x - y|^2}{1 - \e^{-t^2}}
    \biggr ) \exp \biggl (-\frac{\alpha}{2\e^{T^2}} \frac{|x - y|^2}{1
      - \e^{-t^2}} \biggr ).
  \end{align*}
  Finally, we apply the assumption $\alpha \geq 2 \e^{T^2}$ to obtain:
  \begin{equation*}
    \frac{M_{\frac{t^2}{\alpha}}(x, y)}{M_{t^2}(x, y)} \leq \alpha^{\frac{d}2}
    \e^{\frac{t^2}4 |\la x, y \ra|} \exp\biggl(-\frac{\alpha}{2\e^{T^2}} \frac{|x -
      y|^2}{1 - \e^{-t^2}} \biggr).
  \end{equation*}
  Which is as asserted.
\end{proof}

\subsection{An estimate on Gaussian balls}
\begin{lemma}\label{lem:Gaussian-ball-shift-lemma}
  Let $B_t(x)$ be the Euclidean ball with radius $t$ and center $x$
  and let $\gamma$ be the Gaussian measure
  \eqref{eq:Gaussian-measure}. We have the inequality:
  \begin{equation}\label{eq:Gaussian-ball-shift-lemma}
    \frac{\gamma(B_t(x))}{V_d(t)} \leq d \e^{-t^2} \e^{2 t |x|} \gamma(x).
  \end{equation}
\end{lemma}
\begin{proof}
    Next, remark that for a ball $B:= B_t(x)$ there holds that
  \begin{align*}
    \int_B \e^{-|\xi|^2} \D\xi &= \e^{-|x|^2} \int_{B} \e^{-|\xi -
      x|^2} \e^{-2 \la x, \xi - x \ra} \D\xi\\
    &\leq \e^{-|x|^2} \int_{B} \e^{-|\xi - x|^2} \e^{2 |x| |\xi - x|}
    \D\xi\\
    &\leq \pi^{\frac{d}2} \e^{-|x|^2} \e^{2 t |x|} \gamma(B_t(0)).
  \end{align*}
  That is:
  \begin{equation}\label{eq:Gaussian-ball-shift-lemma-proof-1}
    \gamma(B_t(x)) \leq \e^{-|x|^2} \e^{2 t |x|} \gamma(B_t(0)).
  \end{equation}
  We will estimate the Gaussian volume of the ball $B_t(0)$. To ease
  the notation, let $S_d$ and $V_d$ be the surface area and volume
  respectively of the $d$-dimensional unit sphere. Using polar coordinates
  we then obtain: 
  \begin{align*}
    \gamma(B_t(0)) &= \pi^{-\frac{d}2} \int_{B_t(0)} \e^{-|\xi|^2} \D\xi\\
    &= S_d \pi^{-\frac{d}2} \int_0^t \e^{-r^2} r^{d - 1} \D{r}\\
    &\leq S_d t^d \pi^{-\frac{d}2}\e^{-t^2}\\
    &= d V_d(t) \pi^{-\frac{d}2}\e^{-t^2}.
  \end{align*}
  Upon combining this result with
  \eqref{eq:Gaussian-ball-shift-lemma-proof-1} we obtain
  \eqref{eq:Gaussian-ball-shift-lemma}, which is as promised.
\end{proof}

\subsection{On-diagonal kernel estimates on annuli}
As is common in harmonic analysis, we often wish to decompose
$\R^d$ into sets on which certain phenomena are easier to handle. Here
we will decompose the space into disjoint annuli $C_k$. For the sake
of simplicity we will write $B := B_t(x)$ and mean that $B$ is the
closed ball with center $x$ and radius $t$. Furthermore, we use
notations such as $2B$ to mean the ball obtained from $B$ by
multiplying its radius by $2$.

The $C_k$ are given by,
\begin{equation}
  \label{eq:C_k-annulus-decomposition}
  C_k(B) := C_k = (2^{k + 1} - 1)B \setminus (2^k - 1)B.
\end{equation}
So, whenever $\xi$ is in $C_k(B_t(x))$, we get for $k \geq 0$:
\begin{equation}
  \label{eq:C_k-annulus-decomposition-expand}
  (2^k - 1) t < |y - \xi| \leq (2^{k + 1} - 1) t.
\end{equation}

\begin{lemma}\label{lem:On-diagonal-kernel-estimates-on-Ck}
  Given $A > 0$, let $B = B_{At}(y)$, $t \in (0, T]$ and $\xi \in C_k$. Then we have:
  \begin{equation*}
    M_{t^2}(y, \xi) \leq \frac{\e^{-\beta} \e^{|y|^2}}{(1 - \e^{-t^2})^{\frac{d}2}}
    \exp\bigl((2^{k + 1} - 1) A t |y| \bigr) \e^{A^2 2^k} \e^{-\beta 4^k},
  \end{equation*}
  where $\beta = \frac{A^2}{2 \e^{2T^2}}$.
\end{lemma}
\begin{proof}
  Let $B = B_{At}(y)$ and let $C_k$ be as in
  \eqref{eq:C_k-annulus-decomposition}. Considering the first
  exponential which occurs in the Mehler kernel
  \eqref{eq:Mehler-kernel} together with
  \eqref{eq:C_k-annulus-decomposition-expand} gives for $k \geq 0$:
  \begin{align*}
    \exp\biggl(-\e^{-2t^2} \dfrac{|y - \xi|^2}{1 - \e^{-2t^2}} \biggr)
    &\leq \exp\biggl(-\e^{-2t^2} \dfrac{(2^k - 1)^2 A^2 t^2}{1 -
      \e^{-2t^2}} \biggr)\\
   &\hintedrel[rel1]{\leq} \exp\biggl(-\frac{A^2}{2 \e^{2t^2}} (2^k - 1)^2 \biggr).
  \end{align*}
  Where (\hintref{rel1}) follows from
  \begin{equation*}
    \frac{t^2}{1 - \e^{-2t^2}} \geq \frac12.
  \end{equation*}
  Before we consider the last exponential in the Mehler kernel we note
  that by Cauchy-Schwarz:
  \begin{equation}
    \label{lem:On-diagonal-kernel-estimates-on-Ck-proof-1}
    |\langle y, \xi \rangle| \leq |\la y - \xi, y \ra| + |\la y, y \ra|
    \leq |y - \xi||y| + |y|^2.
  \end{equation}
  Furthermore we have the estimate:
  \begin{equation*}
    \frac{\e^{-t^2}}{1 + \e^{-t^2}} \leq \frac12, 
  \end{equation*}
  Using these we get for the last exponential in the Mehler kernel
  \eqref{eq:Mehler-kernel}:
  \begin{align*}
    \exp\biggl(2\e^{-t^2} \dfrac{\la y, \xi \ra}{1 + \e^{-t^2}}
    \biggr) &\leq \exp(|\la y, \xi \ra|)\\
    &\stackrel{\eqref{lem:On-diagonal-kernel-estimates-on-Ck-proof-1}}{\leq}
    \exp(|y - \xi||y|) \e^{|y|^2}.
  \end{align*}
  Wrapping it up, we can estimate the Mehler kernel
  \eqref{eq:Mehler-kernel} $M_{t^2}$ on $C_k$ from above by:
  \begin{equation*}
    M_{t^2}(y, \xi) \leq \frac{\e^{|y|^2}}{(1 - \e^{-t^2})^{\frac{d}2}}
    \exp\bigl((2^{k + 1} - 1) A t |y| \bigr) \exp\biggl(-\frac{A^2}{2 \e^{2t^2}} (2^k - 1)^2 \biggr).
  \end{equation*}
  Setting $\beta = \frac{A^2}{2 \e^{2T^2}}$ and expanding the last
  exponential we get:
  \begin{equation*}
    M_{t^2}(y, \xi) \leq \frac{\e^{-\beta} \e^{|y|^2}}{(1 - \e^{-t^2})^{\frac{d}2}}
    \exp\bigl((2^{k + 1} - 1) A t |y| \bigr) \e^{A^2 2^k} \e^{-\beta 4^k}.
  \end{equation*}
  Which is as claimed.
\end{proof}

\section{The boundedness of some non-tangential maximal operators}
Our theorem is a small modification of \cite[lemma 1.1]{Pineda2008} with a new proof.
\begin{theorem}\label{lem:Maximal-function-cone}
  Let $A, a > 0$. For all $x$ in $\R^d$ and all $u$ in $\LHG$ we have
  \begin{equation}
    \label{eq:Maximal-function-cone}
    \sup_{(y, t) \in \Gamma_x^{(A, a)}} |\e^{-t^2 L} u(y)| \lesssim
    \sup_{r > 0} \dashint_{B_r(x)} |u| \, \D\gamma.
  \end{equation}
\end{theorem}
\begin{proof}
  We will prove \eqref{eq:Maximal-function-cone} by splitting up the
  integration domain in annuli.
  \begin{equation*}
    \e^{-t^2 L} |u(y)| \leq \sum_{k = 0}^\infty I_k(y),
    \:\text{where}\: I_k(y) := \int_{C_k(B)} M_{t^2}(y,
    \cdot) |u| \,\D\gamma.
  \end{equation*}
  More precisely, we will set $B = B(y, At)$ in the above and find a
  suitable upper bound for each integral on the right-hand side which
  we will denote by $I_k$ for the sake of simplicity.

  Since we have $|x - y| < At$ and $t \leq a m(x)$ we infer that $t
  |x| \leq a$. By \autoref{lem:m-xy-equivalence} we also that $t
  |y| \leq 1 + aA$. From this and
  \autoref{lem:On-diagonal-kernel-estimates-on-Ck} we infer that:
  \begin{equation}
    \label{eq:Mehler-kernel-estimate-one-sided-bound-1}
    M_{t^2}(y, \xi) \leq \frac{\e^{-\beta} \e^{|y|^2}}{(1 - \e^{-t^2})^{\frac{d}2}}
    \exp\bigl((2^{k + 1} - 1) A (1 + aA) \bigr) \e^{A^2 2^k} \e^{-\beta 4^k}.
  \end{equation}
  Setting $\alpha := A(1 + aA)$ we get:
  \begin{equation}
    \label{eq:Mehler-kernel-estimate-one-sided-bound}
    M_{t^2}(y, \xi) \leq \e^{-(\alpha + \beta)} \frac{\e^{|y|^2}}{(1 -
      \e^{-t^2})^{\frac{d}2}} \e^{(2\alpha + A^2) 2^k} \e^{-\beta 4^k}.
  \end{equation}
  Recalling \autoref{lem:Gaussian-ball-shift-lemma} we get using $t|x|
  \leq a$ that
  \begin{equation}
    \label{eq:Gaussian-ball-Maximal-function-cone-proof-1}
    \frac{\gamma(B_t(x))}{V_d(t)} \leq d \e^{-t^2} \e^{2 t |x|} \gamma(x).
  \end{equation}
  This allows us to estimate the remaining unbounded exponential in the
  Mehler kernel and allow a penalty up to $\e^{-|x|^2}$. Furthermore,
  we have the following estimate which will make clear how to handle the
  time part in the Mehler kernel:
  \begin{equation*}
    \frac{t^d}{(1 - \e^{-t^2})^{\frac{d}2}} \leq \biggl(\frac{t^2}{1 -
      \e^{-t^2}} \biggr)^{\frac{d}2} \leq \biggl(\frac{a^{2 d^{-1}}}{1 -
      \e^{-a^2}} \biggr)^{\frac{d}2} = C_{a,d}.
  \end{equation*}
  Let $B' := B(x, 2^{k + 1}At)$ and $B$ is as before the ball $B(y,
  At)$. In the next step we will bound ....
  by the maximal function centered at $x$. For this we need to scale
  up the $C_k$. So,
  \begin{equation*}
    |x - \xi| \leq |x - y| + |\xi - y| \leq At + (2^{k + 1} - 1) At =
    2^{k + 1} A t.
  \end{equation*}
  So, we can bound the integral on the right-hand side of
  \eqref{eq:Maximal-function-cone-intermediate-step-1} by
  \begin{align*}
    \int_{C_k(B)}  |u| \,\D\gamma &\leq \int_{B'}  |u| \,\D\gamma\\
    &\leq \gamma(B(x, 2^{k + 1} A t))
    \sup_{r > 0} (M_\gamma u)(x)\\
    &\hintedrel[rel1]{\leq}  d \e^{2^{k + 1} 2aA} \gamma(x)
    V_d(2^{k + 1} A t) (M_\gamma u)(x).
  \end{align*}
  Where (\hintref{rel1}) uses \autoref{lem:Gaussian-ball-shift-lemma}
  and $t |x| \leq a$. \autoref{lem:Cone-Gaussians-comparable} gives us
  by using $|x - y| \leq aA m(x)$ the following estimate:
  \begin{equation*}
    \gamma(x) \leq \e^{(aA)^2} \e^{2aA} \gamma(y).
  \end{equation*}
  
  \begin{equation}
    \label{eq:Mehler-kernel-estimate-one-sided-bound}
    M_{t^2}(y, \xi) \leq \e^{-(\alpha + \beta)} \frac{\e^{|y|^2}}{(1 -
      \e^{-t^2})^{\frac{d}2}} \e^{(2\alpha + A^2) 2^k} \e^{-\beta 4^k}
    =: U_k(y)
  \end{equation}

  \begin{align*}
 \int_{C_k(B)} |u| \,\D\gamma &\leq \int_{B'} |u| \,\D\gamma\\
    &\leq \gamma(B(x, 2^{k + 1} A t)) (M_\gamma u)(x)\\
    &\hintedrel[rel1]{\leq} d \e^{2^{k + 1} 2aA} 2^{d(k + 1)} V_d(A t)
    \gamma(x) (M_\gamma u)(x).
  \end{align*}



  \begin{align*}
    I_k := \int_{C_k(B)} M_{t^2}(y, \cdot) |u| \,\D\gamma &\leq t^d d V_d(A)
    \e^{2^{k + 1} 2aA}  2^{d \cdot 2^{k + 1} a A} \gamma(x) U_k(y) (M_\gamma u)(x).
  \end{align*}

  \begin{align*}
    U_k(y) t^d \gamma(x) &= \e^{-(\alpha + \beta)} \frac{\e^{|y|^2} t^d \gamma(x)}{(1 -
      \e^{-t^2})^{\frac{d}2}} \e^{(2\alpha + A^2) 2^k} \e^{-\beta
      4^k}\\
&\leq \e^{-(\alpha + \beta)} \frac{\e^{|y|^2} T^d \gamma(y)}{(1 -
      \e^{-T^2})^{\frac{d}2}} \e^{(2\alpha + A^2) 2^k} \e^{-\beta 4^k}
    \e^{(aA)^2} \e^{2aA}\\
&\leq \frac{T^d}{(1 - \e^{-T^2})^{\frac{d}2}}\pi^{-\frac{d}2}
\e^{-(\alpha + \beta)} \e^{(aA)^2} \e^{2aA} \e^{(2\alpha + A^2) 2^k} \e^{-\beta 4^k} \\
  \end{align*}


  \begin{align*}
    I_k &\leq t^d d V_d(A) \e^{2^{k + 1} 2aA}  2^{d \cdot 2^{k + 1} a
      A} \gamma(x) U_k(y) (M_\gamma u)(x)\\
    &\leq d V_d(A) \frac{T^d}{(1 - \e^{-T^2})^{\frac{d}2}}\pi^{-\frac{d}2}
\e^{-(\alpha + \beta)} \e^{(aA)^2} \e^{2aA} \e^{(2\alpha + A^2) 2^k}
\e^{-\beta 4^k} \e^{2^{k + 1} 2aA}  2^{d \cdot 2^{k + 1} a A}
(M_\gamma u)(x)\\
    &\leq \frac{d A^d}{\Gamma\Bigl(1 + \frac12 d\Bigr)} \frac{T^d \e^{-(\alpha + \beta)} \e^{(aA)^2} \e^{2aA}}{(1 - \e^{-T^2})^{\frac{d}2}}
\e^{(2\alpha + A^2 + 4aA) 2^k}  2^{d \cdot 2^{k + 1} a A} \e^{-\beta
  4^k} (M_\gamma u)(x)\\
  \end{align*}





  We can then bound the maximal function:
  \begin{equation}
    \label{eq:Maximal-function-cone-intermediate-step-1}
    \begin{split}
      \e^{-t^2 L} |u(y)| &= \sum_{k = 0}^\infty I_k\\
      &\leq (M_\gamma u)(x) \frac{d A^d}{\Gamma\Bigl(1 + \frac12
        d\Bigr)}  \frac{T^d \e^{-(\alpha + \beta)} \e^{(aA)^2} \e^{2aA}}{(1 - \e^{-T^2})^{\frac{d}2}}
\sum_{k = 0}^\infty \e^{(2\alpha + A^2 + 4aA) 2^k}  2^{d \cdot 2^{k + 1} a A} \e^{-\beta
  4^k} \\
    \end{split}
  \end{equation}
  Wrapping it up, we have that:
  \begin{equation*}
    \e^{-t^2 L} |u(y)| \lesssim \dashint_{B_r(x)} |u| \, \D\gamma.
  \end{equation*}
  With implied constant

  Which is what we wanted to prove.
\end{proof}


\printbibliography

\end{document}

\RequirePackage[l2tabu, orthodox]{nag}
\documentclass[a4paper,oneside,10pt]{amsproc}

\usepackage{fixltx2e}
\usepackage[all, error]{onlyamsmath}
\usepackage{fixmath} % http://ctan.org/pkg/fixmath
\usepackage{refcheck}
% \usepackage{gitinfo} % Information about the version
\norefnames
% \showrefnames
\usepackage{microtype}
\usepackage{amsmath}
\usepackage{amsthm}
\usepackage[x11names]{xcolor}%
\usepackage{textcomp}
\usepackage[english]{babel}
\usepackage{xfrac} % Nice / fractions
\usepackage[utf8]{inputenc}
\usepackage[T1]{fontenc}
\usepackage[strict=true]{csquotes} % Needs to be loaded *after* inputenc

%% Tekstiin ja taulukoihin liittyvi� lis�paketteja
\usepackage{enumerate} 
\usepackage{booktabs}% Better tables

%% Matematiikkaan liittyvi� lis�paketteja

\usepackage{latexsym}
\usepackage{bbm}
\usepackage{enumitem}
\usepackage[bitstream-charter]{mathdesign}%

\usepackage[bookmarks,colorlinks,breaklinks]{hyperref} % Add hyperref type links in the document, colors
\definecolor{dullmagenta}{rgb}{0.4,0,0.4} % #660066
\definecolor{darkblue}{rgb}{0,0,0.4}%
\hypersetup{linkcolor=red,citecolor=blue,filecolor=dullmagenta,urlcolor=darkblue} % coloured links


\usepackage{etoolbox}
\newcounter{hints}
\renewcommand{\thehints}{\roman{hints}}
\newcommand{\hintedrel}[2][]{%
  \stepcounter{hints}%
  \if\relax\detokenize{#1}\relax\else\csxdef{hint@#1}{\thehints}\fi
  \mathrel{\overset{\textrm{(\thehints)}}{\vphantom{\le}{#2}}}%
}
\newcommand{\restarthintedrel}{\setcounter{hints}{0}}
\newcommand{\hintref}[1]{\csuse{hint@#1}}
\AfterEndEnvironment{align*}{\setcounter{hints}{0}}% Resets numrel at the end of align*
\AfterEndEnvironment{align}{\setcounter{hints}{0}}% Resets numrel at the end of align
\AfterEndEnvironment{equation*}{\setcounter{hints}{0}}% Resets numrel at the end of equation*
\AfterEndEnvironment{equation}{\setcounter{hints}{0}}% Resets numrel at the end of equation



\makeatletter

\@namedef{subjclassname@2010}{%

  \textup{2010} Mathematics Subject Classification}

\makeatother

% \linespread{1.3}


%%%%%%%% Teoreemaymp�rist�t
% HUOM! K�ytet��n amsthm-pakettia, eik� theorem-pakettia, jos
% theorem-pakettia k�ytet��n t�ytyy my�s n�it� muuttaa

% Perusteoreematyyli, lause ja lemma numeroidaan sectionin
% mukaan. Propositio numeroidaan lauseen numeroinnilla

\theoremstyle{plain}
% \newtheorem*{thmA6.6}{[A, Theorem 6.6]}
\newtheorem*{problem}{Problem}


\swapnumbers
\newtheorem{theorem}{Theorem}
\newtheorem{definition}{Definition}
\newtheorem{lemma}{Lemma}
\newtheorem{corollary}{Corollary}
\newtheorem{proposition}{Proposition}
\theoremstyle{remark}
\renewcommand{\qedsymbol}{\ensuremath{\blacksquare}}
\newtheorem*{remark}{Remark}
\newtheorem*{examples}{Examples}


% M��ritelm�tyyli

\theoremstyle{definition} 
% M��ritelm�, jolla on oma juokseva numerointi 
% \newtheorem{definition}[theorem]{Definition}
\newtheorem*{def*}{Definition}

% Huomautustyyli


% Omat k�skyt (esimerkkin� joukkosymbolit)

\newcommand{\E}{\mathbb{E}}
\newcommand{\Sp}{\mathbb{S}}
\newcommand{\prob}{\mathbb{P}}
\newcommand{\D}{\,\textup{d}}
\newcommand{\Dn}{\textup{d}} % One without space.
\newcommand{\Dt}{\,\frac{\textup{d} t}{t}}
\newcommand{\Ds}{\,\frac{\textup{d} s}{s}}
\newcommand{\DyDt}{\frac{\textup{d} y \, \textup{d} t}{t^{n+1}}}
\newcommand{\Dd}{\mathscr{D}}
\newcommand{\Tt}{\mathscr{T}}
\newcommand{\Cc}{\mathscr{C}}
\newcommand{\Bb}{\mathscr{B}}
\newcommand{\Rr}{\mathscr{R}}
\newcommand{\Ff}{\mathscr{F}}
\newcommand{\Ll}{\mathscr{L}}
\newcommand{\Mm}{\mathscr{M}}
\newcommand{\hh}{\mathfrak{h}}
\newcommand{\ttt}{\mathfrak{t}}
\newcommand{\la}{\langle}
\newcommand{\ra}{\rangle}
\newcommand{\Rad}{\textup{Rad}}
\newcommand{\Car}{\textup{Car}}
\newcommand{\BMO}{\textup{BMO}}
\newcommand{\loc}{\textup{loc}}
\newcommand{\LH}{{L^2_\mu}}
\newcommand{\LHG}{{L^2_\gamma}}

%% Bar int
\def\Xint#1{\mathchoice
  {\XXint\displaystyle\textstyle{#1}}%
  {\XXint\textstyle\scriptstyle{#1}}%
  {\XXint\scriptstyle\scriptscriptstyle{#1}}%
  {\XXint\scriptscriptstyle\scriptscriptstyle{#1}}%
  \!\int}
\def\XXint#1#2#3{{\setbox0=\hbox{$#1{#2#3}{\int}$ }
    \vcenter{\hbox{$#2#3$ }}\kern-.535\wd0}}
\def\ddashint{\Xint=}
\def\dashint{\Xint-}

\def\LI{{L^1_\gamma}}


%% Symbols
\renewcommand{\bar}[1]{\overline#1}
\renewcommand{\vec}[1]{\boldsymbol{\mathbf{#1}}}
\renewcommand{\leq}{\leqslant}
\renewcommand{\Im}{\operatorname{Im}}
\renewcommand{\Re}{\operatorname{Re}}
\renewcommand{\leq}{\leqslant}
\renewcommand{\geq}{\geqslant}
\renewcommand{\epsilon}{\varepsilon}
\renewcommand{\emptyset}{\varnothing}
\newcommand{\Fo}{\mathcal{F}}
\newcommand{\R}{\mathbf R}
\newcommand{\C}{\mathbf C}
\newcommand{\N}{\mathbf N}
\newcommand{\T}{\mathbb T}
\newcommand{\Z}{\mathbf Z}
\newcommand{\B}{\mathcal B}
\newcommand{\e}{\mathrm{e}} %Roman e for exponentials

\renewcommand{\leq}{\leqslant}%
\renewcommand{\geq}{\geqslant}%
\DeclareMathOperator{\supp}{supp}
\newcommand{\Dg}{\frac{\textup{d}\gamma (y)}{\gamma (B(y,t))}}
\newcommand{\Dmu}{\frac{\textup{d}\mu (y)}{\mu (B(y,t))}}


\renewcommand{\Re}{\operatorname{Re}}
\renewcommand{\Im}{\operatorname{Im}}
\renewcommand{\bar}{\overline}

\usepackage{tikz}

\def\lemmaautorefname{Lemma}
\def\definitionautorefname{Definition}
\def\theoremautorefname{Theorem}
\def\corollaryautorefname{Corollary}


%% Bibliography
\usepackage[backend=biber,doi=false,url=false,isbn=false]{biblatex}
\bibliography{~/Documents/BibTeX/library.bib}

\newbibmacro{string+doiurlisbn}[1]{%
  \iffieldundef{doi}{%
    \iffieldundef{url}{%
      \iffieldundef{isbn}{%
        \iffieldundef{issn}{%
          #1%
        }{%
          \href{http://books.google.com/books?vid=ISSN\thefield{issn}}{#1}%
        }%
      }{%
        \href{http://books.google.com/books?vid=ISBN\thefield{isbn}}{#1}%
      }%
    }{%
      \href{\thefield{url}}{#1}%
    }%
  }{%
    \href{http://dx.doi.org/\thefield{doi}}{#1}%
  }%
}

\DeclareFieldFormat{title}{\usebibmacro{string+doiurlisbn}{\mkbibemph{#1}}}
\DeclareFieldFormat[article,incollection]{title}%
{\usebibmacro{string+doiurlisbn}{\mkbibquote{#1}}}


\title[Gaussian estimates]{Gaussian estimates}
%\author{Mikko Kemppainen}
%\address{Department of Mathematics and Statistics, University of Helsinki,
%  Gustaf H�llstr�min katu 2b, FI-00014 Helsinki, Finland}
%\email{mikko.k.kemppainen@helsinki.fi}

\author{Jonas Teuwen}%
\address{Delft Institute of Applied Mathematics,
  Delft University of Technology, P.O. Box 5031, 2600 GA Delft, The
  Netherlands} \email{j.j.b.teuwen@tudelft.nl}%
\urladdr{http://fa.its.tudelft.nl/~teuwen/}%
\thanks{}%
\date{\today}



\begin{document}

\begin{abstract}
  Maximal function! An attempt! A good attempt! I hope!
\end{abstract}

% \subjclass[2010]{42B25 (Primary); 46E40 (Secondary)}
% \keywords{R-bounds, dyadic cubes}

\maketitle
\section{The Mehler kernel and friends}
\subsection{Notation}
To begin, let us fix some notation. As is common, we use $N$ to
represent a positive integer. That is, $N \in \Z_+ = \{1, 2, 3,
\dots\}$. In the same way we cast letters that denote the number of
dimensions, e.g.\ $d$ in $\R^d$ as positive integers.

We use the capital letter $T$ to denote a ``time'' endpoint, for
instance, when writing $t$ in $(0, T]$.

\subsection{Setting}
Given the Ornstein-Uhlenbeck operator $L$ defined as:
\begin{equation}
  \label{eq:Ornstein-Uhlenbeck-operator}
  L = -\frac12 \Delta + x \cdot \nabla,
\end{equation}
We define the Mehler kernel (see e.g., \textcite{Sjogren1997}) as the
Schwartz kernel associated to the Ornstein-Uhlenbeck semigroup
$(\e^{-tL})_t$. More precisely, this means:
\begin{equation}
  \label{eq:Ornstein-Uhlenbeck-semigroup-integral}
  \e^{-tL} u(x) = \int_{\R^d} M_t(x, \cdot) u \, \D\gamma.
\end{equation}
It is often more convenient to use $\e^{-t^2 L}$ instead of $\e^{-tL}$
as is done in e.g., \textcite{Portal2012} and we will also do so.

\subsection{The Mehler kernel}
For the computation of the Mehler kernel in
\eqref{eq:Ornstein-Uhlenbeck-semigroup-integral} we refer to e.g.,
\textcite{Sjogren1997} which additionally offers related results such
as those concerning Hermite polynomials.

If one observes that the kernel $M_{t^2}$ is symmetric in its
arguments, a useful expression is:
\begin{equation}
  \label{eq:Mehler-kernel}
  M_{t^2}(x, y) = \frac{\exp\biggl(-\e^{-t^2} \dfrac{|x - y|^2}{1
      - \e^{-2 t^2}}  \biggr)}{(1 - \e^{-t^2})^{\frac{d}2}}
  \frac{\exp\biggl(\e^{-t^2} \dfrac{|x|^2 + |y|^2}{1 + \e^{-t^2}}
    \biggr)}{(1 + \e^{-t^2})^{\frac{d}2}}.
\end{equation}


\section{Some fine lemmata and definitions}
\subsection{$m$inimal function}
We recall the lemma from \cite[lemma 2.3]{Maas2011b} which first,
--although implicitly-- appeared in \cite{Mauceri2007}. It will be
convenient to define a function $m$ as:
\begin{equation*}
  m(x) := \min\biggl\{1, \frac1{|x|} \biggr\} = 1 \vee \frac1{|x|}.
\end{equation*}
\begin{lemma}\label{lem:m-xy-equivalence-no-cone}
  Let $a, A$ be strictly positive numbers. We have for $x, y$
  in $\R^d$ that:
  \begin{enumerate}
  \item If $|x - y| < A t$ and $t \leq a m(x)$, then $t
    \leq (1 + aA) m(y)$;
  \item Likewise, if $|x - y| < A m(x)$, then $m(x) \leq (1 +
    A) m(y)$ and $m(y) \leq 2 (1 + A) m(x)$. 
  \end{enumerate}
\end{lemma}
We rewrite this lemma using the Gaussian cone $\Gamma_x^{(A, a)}$.
Recall that:
\begin{equation}
  \label{eq:Gaussian-cone}
  \Gamma_x^{(A, a)} := \Gamma_x^{(A, a)}(\gamma) := \{(y, t) \in
  \R^d_+ : |x - y| < At \:\text{and}\: t \leq a m(x)\}.
\end{equation}
We will also write $\Gamma_x^a$ to mean $\Gamma_x^{(1, a)}$. So we can
infer from \autoref{lem:m-xy-equivalence-no-cone} that:
\begin{lemma}\label{lem:m-xy-equivalence}
  Let $a, A$ be strictly positive numbers. Then:
  \begin{enumerate}
  \item If $(y, t) \in \Gamma_x^{(A, a)}$ then $t \leq (1 + aA) m(y)$;
  \item If $(y, t) \in \Gamma_x^{(A, a)}$ then  $(x, t) \in
    \Gamma_y^{(1 + aA, a)}$.
  \end{enumerate}
\end{lemma}
We will use a global/local region dichotomy which we define as
follows.
\begin{definition}
  Given $\tau > 0$, the set $N_\tau$ is given as:
  \begin{equation}
    \label{eq:Definition-local-region}
    N_\tau(x) := N_\tau := \{(x, y) \in \R^{2d} : |x - y| \leq \tau
    m(x) \}.
  \end{equation}
  Sometimes it is easier to work with the set $N_\tau(B)$, which is
  given for $B := B_r(y)$ as:
  \begin{equation}
    \label{eq:Definition-local-region-ball}
    N_\tau(B) := \{y \in \R^d : |x - y| \leq \tau m(x) \}.
  \end{equation}
  When we partition the space into $N_\tau$ and its complement, we
  call the part belonging to $N_\tau$ the \emph{local region} and the
  part belonging to $\complement N_\tau$ the \emph{global region}.
\end{definition}
The set $t \leq a m(x)$ is used in the definition of the cones
$\Gamma_x^{(A, a)}$ and we will name it $D^a$, that is:
\begin{equation}
  \label{eq:Definition-cut-off-set-D}
  D^a := \{(x, t) \in \R^d_+ : t \leq a m(x)\}.
\end{equation}
We will write $D := D^1$ for simplicity.

The next lemma will come useful when we want to cancel exponential
growth in one variable with exponential decay in the other as long
both variables are in a Gaussian cone.
\begin{lemma}\label{lem:Cone-Gaussians-comparable}
  Let $(y, t) \in \Gamma_x^{(A, a)}$. Then the Gaussians in $x$ and $y$
  respectively are comparable. In particular this means that,
  \begin{equation*}
    \e^{-|x|^2} \simeq \e^{-|y|^2}.
  \end{equation*}
\end{lemma}
\begin{remark}
  More precisely, from the proof we get the estimates \eqref{eq:Cone-Gaussians-comparable-1}
  and \eqref{eq:Cone-Gaussians-comparable-2}. That is:
  \begin{equation*}
    \e^{-|x|^2} \leq \e^{(1 + aA)^2 - 1} \e^{-|y|^2},
  \end{equation*}
  and,
  \begin{equation*}
    \e^{-|y|^2} \leq \e^{(1 + aA)^2} \e^{2(1 + aA)} \e^{-|x|^2}.
  \end{equation*}
\end{remark}

\begin{proof}
  Let $(y, t) \in \Gamma_x^{(A, a)}$. Unwrapping the definition we
  have
  \begin{equation*}
    |x - y| < A t \:\text{and}\: t \leq a m(x).
  \end{equation*}
  Hence, by the inverse triangle inequality we get,
  \begin{align*}
    |y|^2 &\leq (aA m(x) + |x|)^2\\
    &= (aA)^2 + 2 a A m(x) |x| + |x|^2\\
    &\leq (aA)^2 + 2 a A + |x|^2.
  \end{align*}
  Therefore,
  \begin{equation}
    \label{eq:Cone-Gaussians-comparable-1}
    \e^{-|y|^2} \geq \e^{-(aA)^2} \e^{-2 aA} \e^{-|x|^2}.
  \end{equation}
  By \autoref{lem:m-xy-equivalence-no-cone} we have $t \leq (1 + aA)
  m(y)$
  \begin{align*}
    |x|^2 &\leq ((1 + aA) m(y) + |y|)^2\\
    &= ((1 + aA) m(y))^2 + 2(1 + aA)m(y)|y| + |y|^2\\
    &\leq (1 + aA)^2 + 2(1 + aA) + |y|^2.
  \end{align*}
  Therefore,
  \begin{equation}
    \label{eq:Cone-Gaussians-comparable-2}
    \e^{-|x|^2} \geq \e^{-(1 + aA)^2} \e^{-2(1 + aA)} \e^{-|y|^2}.
  \end{equation}
  Summarizing we thus have that,
  \begin{equation*}
    \e^{-|x|^2} \simeq \e^{-|y|^2},
  \end{equation*}
  as required.
\end{proof}
\begin{lemma}
  \label{def:Global-region-cone-lemma}
  Let $x, y$ and $z$ in $\R^d$. Set
  \begin{equation*}
    \tau = \frac12 (1 + 2aA)(1 + aA).
  \end{equation*}
  If $|y - z| > \tau m(y)$ (i.e., $(y, z) \notin N_\tau$) and $(y, t)
  \in \Gamma_x^{(A, a)}$ then $|x - z| > \frac12 m(x)$ (i.e., $(x, z)
  \notin N_{\frac12}$).
\end{lemma}
\begin{proof}
  We assume that $(y, z) \notin N_\tau$ and $(y, t) \in \Gamma_x^{(A,
    a)}$. Written out this gives by \eqref{eq:Definition-local-region}
  the inequality $|y - z| > \tau m(y)$, and by
  \eqref{eq:Gaussian-cone} the inequality $|x - y| < aA m(x)$. Note
  that the latter inequality together with
  \autoref{lem:m-xy-equivalence-no-cone} yields,
  \begin{equation}
    \label{eq:Global-region-cone-lemma-proof-1}
    \frac12 \frac1{1 + aA} m(y) \leq m(x) \leq (1 + aA) m(y).
  \end{equation}
  Combining we get $|x - y| < aA (1 + aA) m(y)$. Now we are in
  position to apply the triangle inequality:
  \begin{equation*}
    |x - z| \geq |y - z| - |x - y| > \tau m(y) - aA (1 + aA) m(y).
  \end{equation*}
  As we require an lower bound in terms of $m(x)$ and not $m(y)$, we
  again apply \eqref{eq:Global-region-cone-lemma-proof-1} to obtain:
  \begin{align*}
    |x - z| \geq |y - z| - |x - y| &> \tau m(y) - aA m(y)\\
    &\geq \tau \frac1{1 + aA} m(x) - aA m(x)\\
    &\geq \frac12 m(x).
  \end{align*}
  So we are done.
\end{proof}


\section{On-diagonal estimates}
\subsection{Kernel estimates}
We begin with a technical lemma which will be useful on several
occasions.
\begin{lemma}\label{lem:Time-part-Mehler-time-transform}
  Let $t$ in $(0, T]$ and $\alpha > 1$. Then,
  \begin{equation}\label{eq:Time-part-Mehler-time-transform}
    \alpha \e^{-T^2} \leq \frac{1 - \e^{-t^2}}{1 -
      \e^{-\frac{t^2}{\alpha}}} \leq \alpha.
  \end{equation}
\end{lemma}
\begin{proof}
  Let $t$ in $(0, T]$ and $\alpha > 1$. Applying the mean value theorem to the function $f(\xi) = \xi^\alpha$ gives, for $0 < \xi < \xi'$:
  \begin{equation*}
    f(\xi) - f(\xi') = \alpha \hat{\xi}^{\alpha - 1} (\xi - \xi') \text{ for some $\hat \xi$ in $[\xi, \xi']$}.
  \end{equation*}
  Picking $\xi = 1$ and $\xi' = \e^{-\frac{t^2}{\alpha}}$ gives:
  \begin{equation}
    \label{eq:Time-part-Mehler-time-transform-proof-1}
    \frac{1 - \e^{-t^2}}{1 - \e^{-\frac{t^2}{\alpha}}} = \alpha
    \hat{\xi}^{\alpha - 1} \:\text{for some}\: \hat{\xi} \:\text{in}\:
    [\e^{-\frac{t^2}{\alpha}}, 1].
  \end{equation}
  Applying this result together with the monotonicity of $\xi \mapsto
  \alpha \xi^{\alpha - 1}$ we get:
  \begin{equation*}
    \alpha \e^{-T^2} \leq \alpha \e^{-t^2} \leq \alpha \exp\biggl(-t^2 \frac{\alpha -
      1}{\alpha} \biggr) \leq \frac{1 - \e^{-t^2}}{1 -
      \e^{-\frac{t^2}{\alpha}}}.
  \end{equation*}
  Hence,
  \begin{equation*}
   \alpha \e^{-T^2} \leq \frac{1 - \e^{-t^2}}{1 - \e^{-\frac{t^2}{\alpha}}} \downarrow \alpha.
  \end{equation*}
  Which completes the proof.
\end{proof}

The following lemma will be useful when transfering estimates from
$M_{\frac{t^2}{\alpha}}$ to $M_{t^2}$. It follows from the mean value
theorem applied to $\xi \mapsto \xi^\alpha$.
\begin{lemma}\label{lem:Exponential-estimates}
  For $C, T > 0, \alpha > 1, t$ in $(0, T]$ and all $x, y$ in $\R^d$ we have that
  \begin{equation}
    \label{eq:Exponential-estimates-1}
    \exp \biggl (-C \frac{|x - y|^2}{1 - \e^{-\frac{t^2}\alpha}}
    \biggr ) \leq  \exp \biggl (-C \frac{\alpha}{\e^{T^2}} \frac{|x - y|^2}{1 -
      \e^{-t^2}} \biggr ).
  \end{equation}
\end{lemma}
\begin{proof}
  Let $t$ in $(0, T]$. Applying
  \autoref{lem:Time-part-Mehler-time-transform} we get:
  \begin{align*}
    \exp \biggl (-C \frac{|x - y|^2}{1 - \e^{-\frac{t^2}\alpha}} \biggr )
    \leq \exp \biggl (-C \frac{|x - y|^2}{1 - \e^{-t^2}} \frac{1 -
      \e^{-t^2}}{1 - \e^{-\frac{t^2}\alpha}} \biggr ) \leq \exp \biggl
    (-C \frac{\alpha}{\e^{T^2}} \frac{|x - y|^2}{1 - \e^{-t^2}} \biggr ).
  \end{align*}
  Which is as asserted.
\end{proof}
Later on we will study kernel estimates of kernels related to the
Mehler kernel, but our first lemma is about estimating
$M_{\frac{t}\alpha}$ in terms of $M_t$.
\begin{lemma}\label{lem:Kernel-estimates-1}
 Let $\alpha \geq 2 \e^{T^2}$, $t$ in $(0, T]$ and $x, y$ in $\R^d$.
 If $t |x| \lesssim 1$ and $t |y| \lesssim 1$ then:
  \begin{equation}
    \label{eq:Kernel-lemma-1-estimate} 
    M_{\frac{t^2}{\alpha}}(x, y) \lesssim \exp\biggl (-\frac{\alpha}{2
      \e^{T^2}} \frac{|x - y|^2}{1 - \e^{-t^2}} \biggr ) M_{t^2}(x,
    y),
  \end{equation}
  where the implied constant does not depend on $x, y$ and $t$.
\end{lemma}
\begin{remark}
  If $C$ is the positive constant such that $t |x| \leq C$ and $t |y|
  \leq C$ then the proof gives that the implied constant is bounded
  from above by $\alpha^{\frac{d}2} \e^{\frac{\alpha}2 C^2}$.
\end{remark}
\begin{proof}
  To prove the lemma we compute $M_{\frac{t^2}{\alpha}} M_{t^2}^{-1}$.
  First note that dividing the time-parts by
  \eqref{eq:Time-part-Mehler-time-transform} gives the upper-bound
  $\alpha^{\frac{d}2}$. Furthermore, we can verify that  
  \begin{equation*}
    \frac{1}{1 + \e^{-t^2}} - \frac{1}{1 + \e^{-\frac{t^2}{\alpha}}}
    \geq 0.
  \end{equation*}
  Also,
  \begin{align*}
    \exp \biggl (-\frac12 &\frac{|x + y|^2}{1 + \e^{-\frac{t^2}{\alpha}}}
    \biggr ) \exp \biggl (\frac12 \frac{|x + y|^2}{1 + \e^{-t^2}}
    \biggr )\\
    &\leq \exp \biggl (\frac12 \biggl[\frac{1}{1 +
      \e^{-t^2}} - \frac{1}{1 + \e^{-\frac{t^2}{\alpha}}}
    \biggr] |x + y|^2 \biggr)\\
    &= \exp \biggl (\frac12 \frac1{t^2}\biggl[\frac{1}{1 +
      \e^{-t^2}} - \frac{1}{1 + \e^{-\frac{t^2}{\alpha}}}
    \biggr] t^2 |x + y|^2 \biggr).
  \end{align*}
  Next, as the inner most function is decreasing,
  \begin{align*}
    \lim_{t \to 0} \frac1{t^2}\biggl[\frac{1}{1 +
      \e^{-t^2}} - \frac{1}{1 + \e^{-\frac{t^2}{\alpha}}} \biggr] 
    &= \lim_{t \to 0} \frac1{2t} \biggl[\frac{2t \e^{-t^2}}{(1 +
      \e^{-t^2})^2} - \frac1\alpha \frac{2t
      \e^{-\frac{t^2}{\alpha}}}{(1 + \e^{-\frac{t^2}{\alpha}})^2}
    \biggr]\\ 
    &= \lim_{t \to 0} \biggl[\frac{\e^{-t^2}}{(1 + \e^{-t^2})^2} -
    \frac1\alpha \frac{\e^{-\frac{t^2}{\alpha}}}{(1 +
      \e^{-\frac{t^2}{\alpha}})^2} \biggr]\\
    &\uparrow \frac{1}{4} \biggl(1 - \frac1\alpha \biggr).
  \end{align*}
  So that
  \begin{align*}
    \exp \biggl (-\frac12 &\frac{|x + y|^2}{1 + \e^{-\frac{t^2}{\alpha}}}
    \biggr ) \exp \biggl (\frac12 \frac{|x + y|^2}{1 + \e^{-t^2}}
    \biggr )\\
    &\leq \exp \biggl (\frac18 \biggl(1 - \frac1\alpha \biggr) t^2 |x +
    y|^2 \biggr)\\
    &\leq \exp \biggl (\frac14 t^2 |x|^2 \biggr) \exp \biggl (\frac14
    t^2 |y|^2 \biggr).
  \end{align*}
  So using \autoref{lem:Exponential-estimates} and equation
  \eqref{eq:Exponential-estimates-1} we get
  \begin{align*}
    \frac{M_{\frac{t^2}{\alpha}}(x, y)}{M_{t^2}(x, y)} &\leq
    \alpha^{\frac{d}2}  \exp\biggl(\dfrac12 \dfrac{|x - y|^2}{1 -
      \e^{-t^2}}  \biggr) \exp\biggl(\dfrac12 \dfrac{|x + y|^2}{1 +
      \e^{t^2}} \biggr)\\ 
    &\quad \times \exp\biggl(-\dfrac12 \dfrac{|x - y|^2}{1
      - \e^{-\frac{t^2}{\alpha}}}  \biggr) \exp\biggl(-\dfrac12
    \dfrac{|x + y|^2}{1 + \e^{-\frac{t^2}{\alpha}}} \biggr)\\
    &\leq \alpha^{\frac{d}2} \exp \biggl (\frac12 \biggl[1
    -\frac{\alpha}{2\e^{T^2}} \biggr] \frac{|x - y|^2}{1 - \e^{-t^2}}
    \biggr ) \exp \biggl (-\frac{\alpha}{2\e^{T^2}} \frac{|x - y|^2}{1
      - \e^{-t^2}} \biggr )\\
    &\quad \times  \exp\biggl(\dfrac12 \dfrac{|x + y|^2}{1 + \e^{t^2}}
    \biggr) \exp\biggl(-\dfrac12 \dfrac{|x + y|^2}{1 +
      \e^{-\frac{t^2}{\alpha}}} \biggr).
  \end{align*}
  Thus that,
  \begin{align*}
    \frac{M_{\frac{t^2}{\alpha}}(x, y)}{M_{t^2}(x, y)} &\leq
    \alpha^{\frac{d}2} \exp \biggl (\frac12 \biggl[1 -
    \frac{\alpha}{2\e^{T^2}} \biggr] \frac{|x - y|^2}{1 - \e^{-t^2}}
    \biggr ) \exp \biggl (-\frac{\alpha}{2\e^{T^2}} \frac{|x - y|^2}{1
      - \e^{-t^2}}  \biggr )\\ 
    &\quad \times \exp \biggl (t^2 \frac{|x|^2 + |y|^2}4 \biggr).
  \end{align*}
  For $\alpha \geq 2 \e^{T^2}$ we then obtain:
  \begin{equation*}
    \frac{M_{\frac{t^2}{\alpha}}(x, y)}{M_{t^2}(x, y)} \leq
    \alpha^{\frac{d}2} \exp\biggl(-\frac{\alpha}{2\e^{T^2}} \frac{|x -
      y|^2}{1 - \e^{-t^2}} \biggr) \exp \biggl (\frac14 t^2 |x|^2 \biggr)
    \exp \biggl (\frac14 t^2 |y|^2 \biggr).
  \end{equation*}
  From $t |x| \lesssim 1$ and $t |y| \lesssim 1$ we infer that there
  exists a positive constant $C$ such that $t |x| \leq C$ and $t |y|
  \leq C$.
  \begin{equation*}
    \frac{M_{\frac{t^2}{\alpha}}(x, y)}{M_{t^2}(x, y)} \leq \alpha^{\frac{d}2}
    \e^{\frac{C^2}2} \exp\biggl(-\frac{\alpha}{2\e^{T^2}} \frac{|x -
      y|^2}{1 - \e^{-t^2}} \biggr).
  \end{equation*}
  Which is as asserted.
\end{proof}
\begin{remark}
  More precisely we have the estimate:
  \begin{align*}
    \exp \biggl (-\frac12 &\frac{|x + y|^2}{1 + \e^{-\frac{t^2}{\alpha}}}
    \biggr ) \exp \biggl (\frac12 \frac{|x + y|^2}{1 + \e^{-t^2}}
    \biggr )\\
    &\leq \exp \biggl (\frac14 t^2 |x|^2 \biggr) \exp \biggl (\frac14
    t^2 |y|^2 \biggr) \exp \biggl (-\frac{t^2}\alpha \frac18 |x + y|^2
    \biggr).
  \end{align*}
  Which then produces:
  \begin{equation*}
    \frac{M_{\frac{t^2}{\alpha}}(x, y)}{M_{t^2}(x, y)} \leq \alpha^{\frac{d}2}
    \e^{\frac{C^2}2} \exp\biggl(-\frac{\alpha}{2\e^{T^2}} \frac{|x - y|^2}{1 - \e^{-t^2}} \biggr)
    \exp\biggl(-\frac{t^2}\alpha \frac{|x + y|^2}8 \biggr).
  \end{equation*}
\end{remark}

\subsection{On-diagonal kernel estimates on annuli}
As is common in harmonic analysis, we often wish to decompose
$\R^d$ into sets on which certain phenomena are easier to handle. Thus
we will decompose the space into annuli $C_k$. We will write $B :=
B_t(x)$ and assume that $B$ is the closed ball with center $x$ and
radius $t$. Recall that $2B$ is the ball obtain from $B$ by
multiplying its radius by $2$.

The $C_k$ are given by,
\begin{equation}
  \label{eq:C_k-annulus-decomposition}
  C_k(B) := C_k = (2^{k + 1} - 1)B \setminus (2^k - 1)B.
  \begin{cases}
    2B &\text{if $k = 0$,}\\
    2^{k + 1}B \setminus 2^k B &\text{for $k \geq 1$.}
  \end{cases}
\end{equation}
So, whenever $\xi$ is in $C_k$, we get for $k \geq 1$:
\begin{equation}
  \label{eq:C_k-annulus-decomposition-expand-nonzero}
  2^k a t < |y - \xi| \leq 2^{k + 1} a t.
\end{equation}
While we get for $k = 0$:
\begin{equation}
  \label{eq:C_k-annulus-decomposition-expand-zero}
  |y - \xi| \leq 2 a t.
\end{equation}

\begin{lemma}\label{lem:On-diagonal-kernel-estimates-on-Ck}
  Given $a > 0$, let $B = B_{at}(y)$ and $\xi$ in $C_k$. Furthermore,
  assume that $t \leq aA m(y)$ for some $A > 0$. Then we have
  for $k \geq 1$:
  \begin{equation*}
    M_{t^2}(y, \xi) \leq \frac{\e^{|y|^2}}{(1 - \e^{-t^2})^{\frac{d}2}}
    \exp\biggl(-\frac{a^2}{2} 4^{k + 1} \biggr) \exp\bigl(2^{k + 1} a t |y|
    \bigr).
  \end{equation*}
  and for $k = 0$:
  \begin{equation*}
    M_{t^2}(y, \xi) \leq \frac{\e^{|y|^2}}{(1 -
      \e^{-t^2})^{\frac{d}2}} \exp\bigl(2^{k + 1} a t |y| \bigr).
  \end{equation*}
\end{lemma}
\begin{proof}
  Let $B = B_{at}(y)$ and let $C_k$ be as in
  \eqref{eq:C_k-annulus-decomposition}. We consider the two
  exponentials in the Mehler kernel \eqref{eq:Mehler-kernel}
  separately. First we consider  
  \begin{equation*}
    \exp\biggl(\e^{-t^2} \dfrac{|y|^2 + |\xi|^2}{1 + \e^{-t^2}}
    \biggr).
  \end{equation*}
  Using the triangle inequality we note that:
  \begin{equation}
    \label{lem:On-diagonal-kernel-estimates-on-Ck-proof-1}
    |\xi|^2 \leq |y - \xi|^2 + |y|^2 + 2 |y - \xi||y|.
  \end{equation}
  Next, note that
  \begin{equation*}
    \frac{\e^{-t^2}}{1 + \e^{-t^2}} \leq \frac12.
  \end{equation*}
  Together with \eqref{lem:On-diagonal-kernel-estimates-on-Ck-proof-1}
  this gives for $k \geq 1$:
  \begin{align*}
     \exp\biggl(\e^{-t^2} \dfrac{|y|^2 + |\xi|^2}{1 + \e^{-t^2}}
     \biggr) &\leq \exp\biggl(\e^{-t^2} \dfrac{|y - \xi|^2}{1 +
       \e^{-t^2}} \biggr) \exp(|y - \xi||y|) \exp(|y|^2)\\
         &\hintedrel[rel1]{\leq} \exp\biggl(\e^{-t^2} \dfrac{|y - \xi|^2}{1 +
       \e^{-t^2}} \biggr) \exp( 2^{k + 1} a t |y|) \exp(|y|^2)
  \end{align*}
  Where (\hintref{rel1})
  uses \eqref{eq:C_k-annulus-decomposition-expand-nonzero}
  or \eqref{eq:C_k-annulus-decomposition-expand-zero}. Next we
  consider the exponential:
  \begin{equation*}
   \exp\biggl(\e^{-t^2} \dfrac{|y - \xi|^2}{1 + \e^{-t^2}} \biggr).
  \end{equation*}
  Combining this with the first exponential in the Mehler kernel
  \eqref{eq:Mehler-kernel} we get:
  \begin{align*}
    \exp\biggl(-\e^{-t^2} \dfrac{|y - \xi|^2}{1 - \e^{-2t^2}} & \biggr)
    \exp\biggl(\e^{-t^2} \dfrac{|y - \xi|^2}{1 + \e^{-t^2}}  \biggr)\\
    &\leq \exp\biggl(-\e^{-t^2}\frac{|y - \xi|^2}{1 + \e^{-t^2}}
    \biggl[\dfrac{1}{1 - \e^{-t^2}} - 1 \biggr] \biggr)\\
   &\leq \exp\biggl(-\e^{-t^2}\frac{|y - \xi|^2}{1 + \e^{-t^2}}
    \biggl[\frac{1}{1 - \e^{-t^2}} - \dfrac{1 - \e^{-t^2}}{1 -
      \e^{-t^2}} \biggr] \biggr)\\
   &\leq \exp\biggl(-\e^{-2t^2}\frac{|y - \xi|^2}{1 - \e^{-2t^2}}
   \biggr).
  \end{align*}
  Using \eqref{eq:C_k-annulus-decomposition-expand-nonzero}
  or \eqref{eq:C_k-annulus-decomposition-expand-zero} we get:
  \begin{equation*}
 \exp\biggl(-\e^{-2t^2}\frac{|y - \xi|^2}{1 - \e^{-2t^2}} \biggr) \leq
 \begin{cases}
   1 &\text{if $k = 0$,}\\
   \exp\biggl(-\dfrac{a^2}{2 \e^{2t^2}} 4^{k + 1} \biggr) &\text{if $k
     \geq 1$.}
 \end{cases}
 \end{equation*}
 Using the assumption that $t \leq aA m(y)$ gives that
 \begin{equation*}
   \exp\biggl(-\dfrac{a^2}{2 \e^{2t^2}} 4^{k + 1} \biggr) \leq
   \exp\biggl(-\dfrac{a^2}{2 \e^{2a^2 A^2}} 4^{k + 1} \biggr).
 \end{equation*}
 Combining we get for the Mehler kernel \eqref{eq:Mehler-kernel}:
 \begin{align*}
     M_{t^2}(y, \xi) &\leq \frac{\e^{|y|^2}}{(1 -
       \e^{-t^2})^{\frac{d}2}} \exp(2^{k + 1} a t |y|)\\
     &\leq \frac{\e^{|y|^2}}{(1 -
       \e^{-t^2})^{\frac{d}2}} \exp(2^{k + 1} a^2 A).
 \end{align*}

  This inequality together with
  \begin{equation*}
    \frac{t^2}{1 - \e^{-2t^2}} \geq \frac12,
  \end{equation*}
  yields,
    \begin{equation*}
    \exp\biggl(-\e^{-t^2} \dfrac{|y - \xi|^2}{1 - \e^{-2t^2}}  \biggr)
    \exp\biggl(-\e^{-t^2} \dfrac{|y - \xi|^2}{1 + \e^{-t^2}}  \biggr)
    \leq \exp\biggl(-\dfrac{a^2}{2 \e^{2 t^2}} 4^{k + 1} \biggr).
  \end{equation*}
  Thus, we can estimate the Mehler kernel $M_{t^2}$ on $C_k$ for $k
  \geq 1$ from above by:
  \begin{equation*}
    M_{t^2}(y, \xi) \leq \frac{\e^{|y|^2}}{(1 - \e^{-t^2})^{\frac{d}2}}
    \exp\biggl(-\frac{a^2}{2} 4^{k + 1} \biggr) \exp\bigl(2^{k + 1} a t |y|
    \bigr).
  \end{equation*}
  We are left with the case $k = 0$, which can be done similarly and
  yields:
  \begin{equation*}
    M_{t^2}(y, \xi) \leq \frac{\e^{|y|^2}}{(1 - \e^{-t^2})^{\frac{d}2}}
    \exp\bigl(2^{k + 1} a t |y| \bigr).
  \end{equation*}
  Done.
\end{proof}

\subsection{The Ornstein-Uhlenbeck maximal function}


\printbibliography

\end{document}

