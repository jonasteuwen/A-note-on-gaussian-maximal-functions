%\RequirePackage[l2tabu, orthodox]{nag}
\documentclass{amsart}

%\usepackage{fixltx2e}
%\usepackage[all, error]{onlyamsmath}
%\usepackage{fixmath} % http://ctan.org/pkg/fixmath
%\usepackage{refcheck}
%\norefnames
% \showrefnames
%\usepackage{microtype}
\usepackage{amsmath, amssymb, color, verbatim}
\usepackage[active]{srcltx}
\usepackage{amsthm}
%\usepackage[x11names]{xcolor}%
%\usepackage{textcomp}
%\usepackage[english]{babel}
%\usepackage{xfrac} % Nice / fractions
%\usepackage[utf8]{inputenc}
%\usepackage[T1]{fontenc}
%\usepackage[strict=true]{csquotes} % Needs to be loaded *after* inputenc

\usepackage{enumerate} 
%\usepackage{booktabs}% Better tables

\usepackage{latexsym}
\usepackage{bbm}
\usepackage{enumitem}
%\usepackage[bitstream-charter]{mathdesign}%

%\usepackage[bookmarks,colorlinks,breaklinks]{hyperref} % Add hyperref type links in the document, colors
%\definecolor{dullmagenta}{rgb}{0.4,0,0.4} % #660066
%\definecolor{darkblue}{rgb}{0,0,0.4}%
%\hypersetup{linkcolor=red,citecolor=blue,filecolor=dullmagenta,urlcolor=darkblue} % coloured links


%\makeatletter

%\@namedef{subjclassname@2010}{%

%  \textup{2010} Mathematics Subject Classification}

%\makeatother

\swapnumbers
\newtheorem{theorem}{Theorem}
\newtheorem{definition}{Definition}
\newtheorem{lemma}{Lemma}
\newtheorem{corollary}{Corollary}
\newtheorem{proposition}{Proposition}
\theoremstyle{remark}
\renewcommand{\qedsymbol}{\ensuremath{\blacksquare}}
\newtheorem*{remark}{Remark}
\newtheorem*{examples}{Examples}


\newcommand{\E}{\mathbb{E}}
\newcommand{\Sp}{\mathbb{S}}
\newcommand{\prob}{\mathbb{P}}
\newcommand{\D}{\,\textup{d}}
\newcommand{\Dn}{\textup{d}} % One without space.
\newcommand{\Dt}{\,\frac{\textup{d} t}{t}}
\newcommand{\Ds}{\,\frac{\textup{d} s}{s}}
\newcommand{\DyDt}{\frac{\textup{d} y \, \textup{d} t}{t^{n+1}}}
\newcommand{\Dd}{\mathscr{D}}
\newcommand{\Tt}{\mathscr{T}}
\newcommand{\Cc}{\mathscr{C}}
\newcommand{\Bb}{\mathscr{B}}
\newcommand{\Rr}{\mathscr{R}}
\newcommand{\Ff}{\mathscr{F}}
\newcommand{\Ll}{\mathscr{L}}
\newcommand{\Mm}{\mathscr{M}}
\newcommand{\hh}{\mathfrak{h}}
\newcommand{\ttt}{\mathfrak{t}}
\newcommand{\la}{\langle}
\newcommand{\ra}{\rangle}
\newcommand{\Rad}{\textup{Rad}}
\newcommand{\Car}{\textup{Car}}
\newcommand{\BMO}{\textup{BMO}}
\newcommand{\loc}{\textup{loc}}
\newcommand{\LH}{{L^2_\mu}}
\newcommand{\LHG}{{L^2(\R^d,\gamma)}}

%% Bar int
% \def\Xint#1{\mathchoice
%   {\XXint\displaystyle\textstyle{#1}}%
%   {\XXint\textstyle\scriptstyle{#1}}%
%   {\XXint\scriptstyle\scriptscriptstyle{#1}}%
%   {\XXint\scriptscriptstyle\scriptscriptstyle{#1}}%
%   \!\int}
% \def\XXint#1#2#3{{\setbox0=\hbox{$#1{#2#3}{\int}$ }
%     \vcenter{\hbox{$#2#3$ }}\kern-.535\wd0}}
% \def\ddashint{\Xint=}
% \def\dashint{\Xint-}

\def\LI{{L^1_\gamma}}


%% Symbols
\renewcommand{\bar}[1]{\overline#1}
\renewcommand{\vec}[1]{\boldsymbol{\mathbf{#1}}}
\renewcommand{\leq}{\leqslant}
\renewcommand{\Im}{\operatorname{Im}}
\renewcommand{\Re}{\operatorname{Re}}
\renewcommand{\leq}{\leqslant}
\renewcommand{\geq}{\geqslant}
\renewcommand{\epsilon}{\varepsilon}
\renewcommand{\emptyset}{\varnothing}
\newcommand{\Fo}{\mathcal{F}}
\newcommand{\R}{\mathbf R}
\newcommand{\C}{\mathbf C}
\newcommand{\N}{\mathbf N}
\newcommand{\T}{\mathbb T}
\newcommand{\Z}{\mathbf Z}
\newcommand{\B}{\mathcal B}
\newcommand{\e}{\mathrm{e}} %Roman e for exponentials

\renewcommand{\leq}{\leqslant}%
\renewcommand{\geq}{\geqslant}%
\DeclareMathOperator{\supp}{supp}
\newcommand{\Dg}{\frac{\textup{d}\gamma (y)}{\gamma (B(y,t))}}
\newcommand{\Dmu}{\frac{\textup{d}\mu (y)}{\mu (B(y,t))}}


\renewcommand{\Re}{\operatorname{Re}}
\renewcommand{\Im}{\operatorname{Im}}
\renewcommand{\bar}{\overline}

%\usepackage{tikz}

\def\lemmaeqrefname{Lemma}
\def\definitioneqrefname{Definition}
\def\theoremeqrefname{Theorem}
\def\corollaryeqrefname{Corollary}

\newcommand{\red}{\color{red}}
%% Bibliography
% \usepackage[backend=biber,doi=false,url=false,isbn=false]{biblatex}
% \bibliography{~/Documents/BibTeX/library.bib}
% 
% \newbibmacro{string+doiurlisbn}[1]{%
%   \iffieldundef{doi}{%
%     \iffieldundef{url}{%
%       \iffieldundef{isbn}{%
%         \iffieldundef{issn}{%
%           #1%
%         }{%
%           \href{http://books.google.com/books?vid=ISSN\thefield{issn}}{#1}%
%         }%
%       }{%
%         \href{http://books.google.com/books?vid=ISBN\thefield{isbn}}{#1}%
%       }%
%     }{%
%       \href{\thefield{url}}{#1}%
%     }%
%   }{%
%     \href{http://dx.doi.org/\thefield{doi}}{#1}%
%   }%
% }
% 
% \DeclareFieldFormat{title}{\usebibmacro{string+doiurlisbn}{\mkbibemph{#1}}}
% \DeclareFieldFormat[article,incollection]{title}%
% {\usebibmacro{string+doiurlisbn}{\mkbibquote{#1}}}


\begin{document}
\title[Gaussian maximal functions]{A note on the Gaussian maximal
  function - Version 20 October 2013 + JvN additions}



\author{Jonas Teuwen}%
\address{Delft Institute of Applied Mathematics,
  Delft University of Technology, P.O. Box 5031, 2600 GA Delft, The
  Netherlands}%
\email{j.j.b.teuwen@tudelft.nl}%
\urladdr{http://fa.its.tudelft.nl/~teuwen/}%
\thanks{}%
\date{\today}

\maketitle

\begin{abstract}
  This note presents a proof that 
  the non-tangential maximal function of the Ornstein-Uhlenbeck semigroup
  is bounded almost surely by the Gaussian Hardy-Littlewood maximal
  function.  In particular this entails improvement on a result by
  Pineda and Urbina \cite{Pineda2008} who proved a similar result for 
  a `trunctated' version of the non-tangential maximal function. 
  We actually obtain boundedness of the maximal function on non-tangential
  cones of arbitrary aperture.
  % 
  % 
  % This note presents a proof that the Gaussian Hardy-Littlewood maximal
  % function bounds the general non-tangential Gaussian heat semigroup -the
  % so-called Ornstein-Uhlenbeck semigroup.
  % 
  % In particular we give an improvement on a result by
  % \cite{Pineda2008} which gives the boundedness of the Gaussian
  % maximal function associated to the Ornstein-Uhlenbeck operator.
  % 
  % We present a proof which is at least to the author more transparant.
  % Our main finding in this note is that our proof allows to use a
  % larger cone and actually obtain the maximal function boundedness for
  % a whole class of cones $\Gamma^{(A, a)}_x(\gamma)$.
\end{abstract}

% \subjclass[2010]{42B25 (Primary); 46E40 (Secondary)}
% \keywords{R-bounds, dyadic cubes}

\maketitle
\section{Introduction}
Maximal functions are among the most studied objects in harmonic
analysis. 
It is well known that the classical {\red non-tangential} maximal
function associated with the heat semigroup is bounded almost everywhere
by the Hardy-Littlewood maximal function, 
\begin{equation}\label{eq:classical}
  \sup_{{\red \substack{(y, t) \in   \R^d \times \R_+\\ |x - y| < t}}} |\e^{-t \Delta} u(y)| \lesssim \sup_{r
    > 0}  \frac1{|B_r(x)|}\int_{B_r(x)} |u| \D\lambda.
\end{equation}
Here
the action of \emph{heat semigroup} $\e^{-{\red t} \Delta} u = \rho_t \ast u$ is
given by a convolution of $u$ with the \emph{heat kernel}
\begin{equation*}
  \rho_t(s) := \frac{\e^{-|s|^2/4t}}{(4\pi t)^{\frac{d}2}}.
\end{equation*}
% so that,
% \begin{equation*}
%   \e^{-t \Delta} u(x) = (\rho_t \ast u)(x).
% \end{equation*}
In this note we are interested in its Gaussian counterpart. 
% Gaussian harmonic analysis seems to be conceptually
% nothing more than harmonic analysis with the Gaussian measure, but
% this is far off from reality. 
The change from Lebesgue measure to the Gaussian measure
\begin{equation}
  \label{eq:Gaussian-measure}
  \mathrm{d}\gamma(x) := \pi^{-\frac{d}2} \e^{-|x|^2} \D{x}
\end{equation}
introduces quite
some intricate technical and conceptual difficulties which are caused 
by the fact that the Gaussian measure is non-doubling.
The Gaussian analogue to the Laplacian is the
\emph{Ornstein-Uhlenbeck operator} $L$,
\begin{equation}
  \label{eq:Ornstein-Uhlenbeck-operator}
  L := -\frac12 \Delta + \la x, \nabla \ra = \frac12 \nabla_\gamma^* \nabla_\gamma,
\end{equation}
where $\nabla_\gamma$ denotes the realisation of the gradient in $L^2(\R^d,\gamma)$.
Our main result, to be proved in Theorem \ref{thm:Gaussian-maximal-function}, is the 
following Gaussian analogue of \eqref{eq:classical}:
\begin{equation}\label{eq:main}
  \sup_{(y, t) \in \Gamma_x^{(A, a)}} |\e^{-{\red t^2} L} u(y)| \lesssim
  \sup_{r > 0} \frac1{\gamma(B_r(x))}\int_{B_r(x)} |u| \, \D\gamma.
\end{equation}
Here, 
\begin{equation}
  \label{eq:Gaussian-cone}
  \Gamma_x^{(A, a)} := \Gamma_x^{(A, a)}(\gamma) := \{(y, t) \in
  {\red \R^d\times \R_+}  \,: \, |x - y| < At \:\text{and}\: t \leq a m(x)\}
\end{equation}
 is the Gaussian cone with aperture $A$ and cut-off parameter $a$, and 
\begin{equation}\label{eq:m-function}
  m(x) := \min\biggl\{1, \frac1{|x|} \biggr\}. % = 1 \wedge \frac1{|x|}.
\end{equation}

A slighly weaker version of the inequality \eqref{eq:main} has been proved by 
Pineda and Urbina \cite{Pineda2008} who showed that 
\begin{equation*}
  \sup_{(y, t) \in \widetilde{\Gamma}_x} |\e^{-{\red t^2 L}} u(y)|
  \lesssim \sup_{r > 0}  \frac1{\gamma(B_r(x))}\int_{B_r(x)} |u| \D\gamma,
\end{equation*}
where
\begin{equation*}
  \widetilde{\Gamma}_x(x) = \{(y, t) \in \R^d_+ : |x - y| < t \leq
  \widetilde{m}(x)\}
\end{equation*}
is the `reduced' Gaussian cone corresponding to the function
\begin{equation*}
  \widetilde{m}(x) = \min\biggl\{\frac12, \frac1{|x|}\biggr\}.
\end{equation*}
Their proof does not seem to easily generalize the range of $t$ from $\frac12$ up
to $1$. \footnote{Pierre beweert van wel}
Our proof of \eqref{eq:main} is different and, we believe, more transparent 
than the one presented in \cite{Pineda2008}. It has the further advantage of allowing 
the extension to cones
with arbitrary aperture $A>0$ and cut-off parameter $a>0$. This additional generality
is very important and has already been used by Portal (cf. 
the claim made in \cite[discussion preceding Lemma 2.3]{Portal2012}) to prove the 
$H^1$-boundedness of the Riesz transform associated with $L$.

Before we continue, let us fix some notation.  
%We will use without
%further reference notation such as $\Z^d$ while we implicitly imply
%that 
The number $d$ is a positive integer. To avoid possible confusion, we define
the \emph{positive integers} as the set $\Z_+ = \{1, 2, 3, \dots\}$. 


% How do cone scalings behave wrt several maximal functions?
% Especially for the latter.
% 
% Need to add domain and measure in first section, but that can be in
% the introduction later on.
% 
% 

%\subsubsection{$m$inimal function}
%We recall the lemma from \cite[lemma 2.3]{Maas2011} which first
%--although implicitly-- appeared in \cite{Mauceri2007}.
%\fbox{Nog materiaal toevoegen?}

\section{The Mehler kernel}
%\subsection{Setting}
%Recall that we work with the \emph{Ornstein-Uhlenbeck} operator $L$ as
%given by \eqref{eq:Ornstein-Uhlenbeck-operator}.

The Mehler kernel (see e.g., \cite{Sjogren1997}) is the
Schwartz kernel associated to the Ornstein-Uhlenbeck semigroup
$(\e^{-tL})_{t \geq 0}$, that is,
\begin{equation}
  \label{eq:Ornstein-Uhlenbeck-semigroup-integral}
  \e^{-tL} u(x) = \int_{\R^d} M_t(x, \cdot) u \, \D\gamma.
\end{equation}
%It is often more convenient to use $\e^{-t^2 L}$ instead of $\e^{-tL}$
%as is done in e.g., \cite{Portal2012}.
%\subsection{The Mehler kernel}
There is an abundance of literature on the Mehler kernel and its
properties. We shall only use the fact, proved e.g. in the review paper
\cite{Sjogren1997}, that the Mehler kernel is given explicitly
by
\begin{equation*}
 M_t(x,y) = {\red\hbox{(formule geven)}}
\end{equation*}
%$M_t$ of \eqref{eq:Ornstein-Uhlenbeck-semigroup-integral} is computed
%there. In addition it offers related results with to the Hermite
%polynomials.
Note that $M_t(x,y)$ is {\red symmetric in $x$ and $y$}. 
%invariant under the permutation $x \leftrightarrow y$. 
A formula for $M_t$ which honors this observation is:
\begin{equation}
  \label{eq:Mehler-kernel}
  M_t(x, y) = \frac{\exp\biggl(-\e^{-2t} \dfrac{|x - y|^2}{1
      - \e^{-2 t}}  \biggr)}{(1 - \e^{-t})^{\frac{d}2}}
  \frac{\exp\biggl(2\e^{-t} \dfrac{\la x, y \ra}{1 + \e^{-t}}
    \biggr)}{(1 + \e^{-t})^{\frac{d}2}}.
\end{equation}

\section{Some lemmata}
We use $m$ as defined in \eqref{eq:m-function} in our next lemma,
which is taken from \cite{MaasNeervenPortal2011}.
\begin{lemma}\label{lem:m-xy-equivalence}
  Let $a, A$ be strictly positive real numbers and $t > 0$. We have
  for $x, y \in \R^d$ that:
  \begin{enumerate}
  \item If $|x - y| < A t$ and $t \leq a m(x)$, then $t
    \leq (1 + aA) m(y)$;
  \item If $|x - y| < A m(x)$, then $m(x) \leq (1 +
    A) m(y)$ and $m(y) \leq 2 (1 + A) m(x)$. 
  \end{enumerate}
\end{lemma}

The next lemma will come useful when we want to cancel exponential
growth in one variable with exponential decay in the other as long
both variables are in a Gaussian cone.
\begin{lemma}\label{lem:Cone-Gaussians-comparable}
  Let $\alpha > 0$ and $|x - y| \leq \alpha m(x)$. Then:
  \begin{equation*}
   \e^{-\alpha^2-2\alpha} \e^{|y|^2}
  \leq \e^{|x|^2} \leq
    \e^{\alpha^2(1 + \alpha)^2+2\alpha(1 + \alpha)} \e^{|y|^2} .
  \end{equation*}
\end{lemma}
\begin{proof}
  By the inverse triangle inequality and $m(x)|x| \leq 1$ we get, 
  \begin{equation*}
    |y|^2 \leq (\alpha m(x) + |x|)^2 \leq \alpha^2 + 2 \alpha + |x|^2.
  \end{equation*}
This gives the first inequality.  For the second we use
  Lemma \ref{lem:m-xy-equivalence} to infer $m(x) \leq (1 + \alpha)
  m(y)$. Proceeding as before we obtain: 
  \begin{equation*}
    |x|^2 \leq \alpha^2 (1 + \alpha)^2 + 2 \alpha (1 + \alpha) + |y|^2.
  \end{equation*}
As required.
%  Combining we get:
%  \begin{equation}
%    \label{eq:Cone-Gaussians-comparable}
%    \e^{-\alpha^2(1 + \alpha)^2} \e^{-2\alpha(1 + \alpha)} \e^{-|y|^2}
%    \leq \e^{-|x|^2} \leq \e^{\alpha^2} \e^{2\alpha} \e^{-|y|^2}.
%  \end{equation}
%  
\end{proof}


\begin{comment}
 %%%% The results of this section are not used in the proof of the maximal estimate!

\section{On-diagonal estimates}
\subsection{Kernel estimates}
Given a fixed number $\alpha>1$ we define $\kappa(\alpha)$ and
$\mu(\alpha)$ as:
\begin{equation*}
  \kappa(\alpha) = 2\Bigl(1 + \frac1\alpha \Bigr)^{-1}, \quad \mu(\alpha)
  = 2\Bigl(1 - \frac1\alpha \Bigr)^{-1}.
\end{equation*}
Note that $\kappa(\alpha)$ and $\mu(\alpha)$ are conjugate exponents:
$  \frac1{\kappa(\alpha)} + \frac1{\mu(\alpha)} = 1$.

We proceed with a simple technical lemma which is given here as it
will be used on several occasions.
\begin{lemma}\label{lem:Time-part-Mehler-time-transform}
  Let $t > 0$ and $\alpha {\red >} 1$. Then,
  \begin{equation}
    \label{eq:Time-part-Mehler-time-transform-1}
    \alpha \e^{-\frac{2t}{\mu(\alpha)}} \leq \frac{1 -
      \e^{-t}}{1 - \e^{-\frac{t}{\alpha}}} \leq \alpha,
  \end{equation}
  \begin{equation}
    \label{eq:Time-part-Mehler-time-transform-2}
    0 \leq \frac1t \biggl[\frac{\e^{-\frac{t}\alpha}}{1 + \e^{-\frac{t}{\alpha}}}
    - \frac{\e^{-t}}{1 + \e^{-t}} \biggr] \leq \frac{1}{2 \mu(\alpha)}.
  \end{equation}
\end{lemma}
\begin{proof}
  We start with \eqref{eq:Time-part-Mehler-time-transform-1} and apply
  the mean value theorem to the function $f(\xi) = \xi^\alpha$. For $0
  < \xi < \xi'$ this gives that:
  \begin{equation*}
    f(\xi) - f(\xi') = \alpha \hat{\xi}^{\alpha - 1} (\xi - \xi')
    \text{ for some $\hat \xi$ in $[\xi, \xi']$}.
  \end{equation*}
  Picking $\xi = 1$ and $\xi' = \e^{-\frac{t}{\alpha}}$ yields:
  \begin{equation}
    \label{eq:Time-part-Mehler-time-transform-proof-1}
    \frac{1 - \e^{-t}}{1 - \e^{-\frac{t}{\alpha}}} = \alpha
    \hat{\xi}^{\alpha - 1} \:\text{for some}\: \hat{\xi} \:\text{in}\:
    \Bigl[\e^{-\frac{t}{\alpha}}, 1 \Bigr].
  \end{equation}
  Combining this result with the monotonicity of $\xi \mapsto
  \alpha \xi^{\alpha - 1}$ we obtain:
  \begin{equation*}
    \alpha \e^{-\frac{2t}{\mu(\alpha)}} \leq \frac{1 - \e^{-t}}{1 -
      \e^{-\frac{t}{\alpha}}} \leq \alpha,
  \end{equation*}
  where the last bound follows from the monotonicity together with the
  limit as $t \downarrow 0$.
  We proceed with \eqref{eq:Time-part-Mehler-time-transform-2}.
  Recalling that $\alpha > 1$ one can directly verify that the
  function
  \begin{equation*}
    t \mapsto \frac1t \biggl[\frac{\e^{-\frac{t}{\alpha}}}{1 +
      \e^{-\frac{t}{\alpha}}} - \frac{\e^{-t}}{1 + \e^{-t}} \biggr]
  \end{equation*}
  is non-negative and decreasing. To find an upper bound we compute the limit
  as $t$ goes to $0$. That is: 
   \begin{equation*}
    \lim_{t \to 0} \frac1t \biggl[\frac{\e^{-\frac{t}\alpha}}{1 +
      \e^{-\frac{t}{\alpha}}} - \frac{\e^{-t}}{1 + \e^{-t}} \biggr] 
    = \lim_{t \to 0} \biggl[\frac1\alpha \frac{\e^{-\frac{2t}{\alpha}}}{(1 +
      \e^{-\frac{t}{\alpha}})^2} - \frac{\e^{-2t}}{(1 + \e^{-t})^2} \biggr] {\red = } \frac{1}{2\mu(\alpha)}.
  \end{equation*}
  Which is as asserted and completes the proof.
\end{proof}

The following lemma will be useful when transfering estimates from
$M_{\frac{t}{\alpha}}$ to $M_t$. 
%It follows from the mean value
%theorem applied to $\xi \mapsto \xi^\alpha$.
\begin{lemma}\label{lem:Exponential-estimates}
  For $\alpha {\red >} 1$ and $0 < t \leq T < \infty$ and all let $x, y \in \R^d$
  we have that:
  \begin{equation}
    \label{eq:Exponential-estimates-1}
    \exp \biggl (-\frac1{\e^{\frac{2t}\alpha}} \frac{|x - y|^2}{1 - \e^{-\frac{2t}\alpha}}
    \biggr ) \leq \exp \biggl(-\frac{\alpha}{2\e^{\frac{2t}\kappa(\alpha)}} \frac{|x -
      y|^2}{1 - \e^{-t}} \biggr).
  \end{equation}
\end{lemma}
\begin{proof}
%  Noting that
%  \begin{equation*}
%$    1 - \e^{-2t} = (1 - \e^{-t})(1 + \e^{-t})$
%  \end{equation*}
%  we can quickly deduce t
The left-hand inequality in:
  \begin{equation}\label{eq:extra1}
     {\red \frac1{2 \e^{2t}} \leq\frac{\e^{-2t}}{1 - \e^{-t}} \leq\frac12}
  \end{equation}
 leads to:
  \begin{equation}\label{eq:extra2}
    \exp\biggl(-\e^{-2t} \dfrac{|x - y|^2}{1 - \e^{-2 t}} \biggr)
    \leq \exp\biggl(-\frac1{2\e^{2t}} \dfrac{|x - y|^2}{1 - \e^{-t}} \biggr).
  \end{equation}
  Substituting $t \to t/\alpha$ into \eqref{eq:extra2}, we obtain
  \begin{align*}
    \exp\biggl(-\e^{-\frac{2t}{\alpha}} \frac{|x - y|^2}{1 - \e^{-\frac{2t}{\alpha}}} \biggr)
    &\overset{\phantom{\eqref{eq:Time-part-Mehler-time-transform-1}}}{\leq} \exp\biggl(-\frac1{2\e^{\frac{2t}{\alpha}}} \frac{|x - y|^2}{1 -
      \e^{-\frac{t}{\alpha}}} \biggr)\\
    &\overset{\phantom{\eqref{eq:Time-part-Mehler-time-transform-1}}}{{\red =}}
    \exp\biggl(-\frac1{2\e^{\frac{2t}{\alpha}}} \frac{1 - \e^{-t}}{1 - \e^{-\frac{t}{\alpha}}}
    \frac{|x - y|^2}{1 - \e^{-t}} \biggr)\\
    &\overset{\eqref{eq:Time-part-Mehler-time-transform-1}}{\leq}
    \exp\biggl(-\frac\alpha{2\e^{\frac{2t}{\alpha}}} \frac1{\e^{2
        \frac{t}{\mu(\alpha)}}} \frac{|x - y|^2}{1 - \e^{-t}} \biggr)\\
    &\overset{\vphantom{\eqref{eq:Time-part-Mehler-time-transform-1}}}{\leq}
    \exp\biggl(-\frac\alpha{2\e^{\frac{2t}{\kappa(\alpha)}}} \frac{|x - y|^2}{1 -
      \e^{-t}} \biggr).
  \end{align*}
  This completes the proof.
\end{proof}
We now come to our estimate of
$M_{\frac{t}{\alpha}}$ in terms of $M_t$: {\red (Is this result needed later on?)}
%Our first proposition will be useful when we want transfer bounds on $M_t$
%for large values of $t$ where the potential plays a smaller role compared to
%the diffusion to a bound on $M_{t'}$ for $t'$ close to $0$ where diffusion
%plays a much smaller role.
%\subsubsection{Time-scaling of the Mehler kernel}
\begin{proposition}\label{lem:Kernel-estimates-1}
  Let ${\red t>0}$ and $x, y \in \R^d$. If $\alpha \geq {\red 2} \e^{2
    \frac{t}{\kappa(\alpha)}}$, then:
  \begin{equation}
    \label{eq:Kernel-lemma-1-estimate} 
    M_{\frac{t}{\alpha}}(x, y) \leq \alpha^{\frac{d}2}
    \exp\biggl (\frac{t}{\mu(\alpha)} |\la x, y \ra| \biggr)
   \exp\biggl(-\frac{\alpha}{4\e^{\frac{2t}{\kappa(\alpha)}}} \frac{|x - y|^2}{1 - \e^{-t}}
   \biggr) M_{t}(x, y).
  \end{equation}
\end{proposition}
\begin{proof}
  To prove the lemma we compute $M_{\frac{t}{\alpha}} M_t^{-1}$.
  First note that \eqref{eq:Time-part-Mehler-time-transform-1} gives
%   \begin{equation*}
%     \alpha \e^{-\frac{2t}\mu(\alpha)} \leq \frac{1 - \e^{-t}}{1 -
%       \e^{-\frac{t}{\alpha}}} \leq \alpha.
%   \end{equation*}
%   This implies that
  \begin{equation*}
    \frac{(1 - \e^{-t})^{\frac{d}2}}{(1 -
      \e^{-\frac{t}{\alpha}})^{\frac{d}{2}}} \leq \alpha^{\frac{d}2}.
  \end{equation*}
  We shall give bounds for the exponentials in the product $M_{\frac{t}{\alpha}}
  M_t^{-1}$ separately. To begin with,
  \begin{align*}
    \exp \biggl (2\e^{-\frac{t}\alpha} \frac{\la x, y \ra}{1 + \e^{-\frac{t}{\alpha}}}
    \biggr ) &\exp \biggl (-2 \e^{-t} \frac{\la x, y \ra}{1 + \e^{-t}}
    \biggr )\\
    &\stackrel{\phantom{\eqref{eq:Time-part-Mehler-time-transform-2}}}{=}
    \exp\biggl (\frac2{t}\biggl[\frac{\e^{-\frac{t}\alpha}}{1 +
      \e^{-\frac{t}{\alpha}}} - \frac{\e^{-t}}{1 + \e^{-t}} \biggr] t \la x, y \ra \biggr)\\
    &\stackrel{\eqref{eq:Time-part-Mehler-time-transform-2}}{\leq}
    \exp\biggl (\frac{t}{\mu(\alpha)} |\la x, y \ra| \biggr).
  \end{align*}
  Hence, 
  \begin{align*}
    \frac{M_{\frac{t}{\alpha}}(x, y)}{M_{t}(x, y)} &\stackrel{\phantom{\eqref{eq:Exponential-estimates-1}}}{\leq}
    \alpha^{\frac{d}2} \exp\biggl (\frac{t}{\mu(\alpha)} |\la x, y \ra| \biggr)
    \exp\biggl(\e^{-2t} \frac{|x - y|^2}{1 - \e^{-2t}} \biggr)
    \exp\biggl(-\e^{-\frac{2t}{\alpha}} \frac{|x - y|^2}{1 -
      \e^{-\frac{2t}{\alpha}}}  \biggr)\\ 
 % extra line inserted
  &\stackrel{\eqref{eq:Exponential-estimates-1}}{\leq}  {\red
 \alpha^{\frac{d}2} \exp\biggl (\frac{t}{\mu(\alpha)} |\la x, y \ra| \biggr)
    \exp\biggl(\frac{\e^{-2t}}{1 + \e^{-2t}} \frac{|x - y|^2}{1 - \e^{-t}} \biggr)
\exp \biggl(-\frac{\alpha}{2\e^{\frac{2t}\kappa(\alpha)}} \frac{|x -
      y|^2}{1 - \e^{-t}} \biggr)}
\\ 
  &\stackrel{\eqref{eq:extra1}}{\leq} \alpha^{\frac{d}2} \exp\biggl (\frac{t}{\mu(\alpha)} |\la x, y \ra|
    \biggr) \exp \biggl (\biggl[{\red{\frac12}}-\frac{\alpha}{4\e^{2
        \frac{t}{\kappa(\alpha)}}} \biggr]  \frac{|x - y|^2}{1 - \e^{-t}}
    \biggr ) \exp \biggl (-\frac{\alpha}{4\e^{\frac{2t}{\kappa(\alpha)}}} \frac{|x - y|^2}{1
      - \e^{-t}} \biggr). 
  \end{align*}
  Finally, we apply the assumption $\alpha \geq {\red 2} \e^{2
    \frac{t}{\kappa(\alpha)}}$ to obtain: 
  \begin{equation*}
    \frac{M_{\frac{t}{\alpha}}(x, y)}{M_t(x, y)} \leq \alpha^{\frac{d}2}
   \exp\biggl (\frac{t}{\mu(\alpha)} |\la x, y \ra| \biggr)
   \exp\biggl(-\frac{\alpha}{4\e^{\frac{2t}{\kappa(\alpha)}}} \frac{|x - y|^2}{1 - \e^{-t}}
   \biggr). 
 \end{equation*}
  Which is as asserted. 
% The assumption $\alpha \geq 4 \e^{2
%     \frac{t}{\kappa(\alpha)}}$ can be rephrased as the requirement that
%   $\frac{\kappa(\alpha)}2 \log\Bigl( \frac\alpha4 \Bigr) \geq t$.
\end{proof}
\end{comment}

\subsection{An estimate on Gaussian balls}
 Let $B = B_t(x)$ be the open Euclidean ball with radius $t$ and center $x$
  and let $\gamma$ be the Gaussian measure as defined by
  \eqref{eq:Gaussian-measure}. We shall denote by $S_d$ the surface area 
  of the unit sphere in $\R^d$.

\begin{lemma}\label{lem:Gaussian-ball-shift-lemma}
  For all $x\in\R^d$ and $t>0$ we have the inequality:
  \begin{equation}\label{eq:Gaussian-ball-shift-lemma}
    \gamma(B_t(x))\leq {\red  \frac12 S_d t^{d} \e^{2t|x|} \e^{-|x|^2}}. 
  \end{equation}
\end{lemma}
\begin{proof}
  Remark that
  \begin{align*}
    \int_B \e^{-|\xi|^2} \D\xi &= \e^{-|x|^2} \int_{B} \e^{-|\xi -
      x|^2} \e^{-2 \la x, \xi - x \ra} \D\xi\\
    &\leq \e^{-|x|^2} \int_{B} \e^{-|\xi - x|^2} \e^{2 |x| |\xi - x|}
    \D\xi\\
    &\leq \e^{-|x|^2} \e^{2 |x| t} \int_{B} \e^{-|\xi - x|^2} \D\xi\\
    &= \pi^{\frac{d}2} \e^{-|x|^2} \e^{2 t |x|} \gamma(B_t(0)).
  \end{align*}
  So, there holds that
  \begin{equation}\label{eq:Gaussian-ball-shift-lemma-proof-1}
    \gamma(B_t(x)) \leq \e^{-|x|^2} {\red \e^{2a}} \gamma(B_t(0)).
  \end{equation}
  We will estimate the Gaussian volume of the ball $B_t(0)$. Using polar coordinates
  we proceed by: 
  \begin{align*}
    \gamma(B_t(0)) &= \pi^{-\frac{d}2} \int_{B_t(0)} \e^{-|\xi|^2} \D\xi\\
    &=\pi^{-\frac{d}2}S_d \int_0^t \e^{-r^2} r^{d - 1} \D{r}\\
    & {\red \leq\frac12 \pi^{-\frac{d}2}S_d t^{d-2} \int_0^t 2r \e^{-r^2} \D{r}}\\
    & {= \frac12 \pi^{-\frac{d}2}S_d t^{d-2} (1-\e^{-t^2})} \\
    & {\red \leq \frac12 \pi^{-\frac{d}2}S_d t^{d},}
  \end{align*}
  where the last step uses $1-\e^{-x} \leq x$ for $x\ge 0$.
  Upon combining this result with
  \eqref{eq:Gaussian-ball-shift-lemma-proof-1} we obtain
  \eqref{eq:Gaussian-ball-shift-lemma}.
\end{proof}


\subsection{On-diagonal kernel estimates on annuli}
As is common in harmonic analysis, we often wish to decompose
$\R^d$ into sets on which certain phenomena are easier to handle. Here
we will decompose the space into disjoint annuli. 

Throughout this subsection we fix $x\in \R^d$, constants $A,a>0$, a pair $(y,t)\in \Gamma_x^{(A,a)}$,
and we put $\alpha:=Aa$. 
%{\red \fbox{Later need $\alpha > 2\e^{\frac{2t}{\kappa(\alpha)}}$?}}
%We write {\red $B^\alpha := B_{\alpha t}(x)$} and mean that $B^\alpha$ is the
%{\red open} ball with center $x$ and radius $\alpha t$. Furthermore, w
We use
the notation $mB$ to mean the ball obtained from the ball $B$ by
multiplying its radius by $m$.

The annuli $C_k$ are given by:
\begin{equation}
  \label{eq:C_k-annulus-decomposition}
  C_k : = (2^{k + 1} - 1)B_{\alpha t}(x) \setminus (2^k - 1)B_{\alpha t}(x), \quad {\red k\ge 0.}
\end{equation}
{\red Note that $C_0 = B_{\alpha t}(x)$.} Whenever $\xi$ is in $C_k$, we get for $k \geq 0$:
\begin{equation}
  \label{eq:C_k-annulus-decomposition-expand}
  (2^k - 1)\alpha t < |y - \xi| \leq (2^{k + 1} - 1) \alpha t.
\end{equation}

On $C_k$ we have the following bound for $M_{t^2}(y,\cdot)$:
\begin{lemma}\label{lem:On-diagonal-kernel-estimates-on-Ck}
%Suppose that $\alpha { \red \, >0} $, and let 
  %$B = B_{\alpha t}(y)$, 
%  ${\red t>0}$.
  %< t \leq T < \infty$
  For all $\xi \in C_k$ we have:
  \begin{equation}
    M_{t^2}(y, \xi) \leq \frac{\e^{|y|^2}}{(1 - \e^{-{\red 2}{t^2}})^{\frac{d}2}}
    \exp\bigl((2^{k + 1} - 1) \alpha t |y| \bigr) \exp\Big(\!-\!\frac{\alpha^2}{2 \e^{2{\red a}^2}} (2^k - 1)^2 \Big),
  \end{equation}
\end{lemma}
\begin{proof}
 Considering the first
  exponential which occurs in the Mehler kernel
  \eqref{eq:Mehler-kernel} together with
  \eqref{eq:C_k-annulus-decomposition-expand} gives for $k \geq 0$:
  \begin{align*}
    \exp\biggl(-\e^{-2t^2} \frac{|y - \xi|^2}{1 - \e^{-2t^2}} \biggr)
    &\leq \exp\biggl(-\e^{-2t^2} \frac{(2^k - 1)^2 \alpha^2 t^2}{1 - \e^{-2t^2}} \biggr)\\
    &\overset{(\dagger)}{\leq} \exp\biggl(-\frac{\alpha^2}{2 \e^{2t^2}} (2^k - 1)^2 \biggr)
    \overset{(\dagger\dagger)}{\leq} \exp\biggl(-\frac{\alpha^2}{2 \e^{2a^2}} (2^k - 1)^2 \biggr),
  \end{align*}
  where $(\dagger)$ follows from $1-\e^{-x} \leq x$ for $x\ge 0$, and $(\dagger\dagger)$
  uses that $t\le am(x)\le a$.
%   Before we consider the second exponential in the Mehler kernel we note
%   that by Cauchy-Schwarz:
%   \begin{equation}
%     \label{lem:On-diagonal-kernel-estimates-on-Ck-proof-1}
%     |\langle y, \xi \rangle| \leq |\la y - \xi, y \ra| + |\la y, y \ra|
%     \leq |y - \xi||y| + |y|^2.
%   \end{equation}
%   Furthermore we have 
Using the estimate $1+x \ge 2x$ for $0\leq x\leq1$, 
%  \begin{equation*}
%    \frac{\e^{-t}}{1 + \e^{-t}} \leq \frac12, 
%  \end{equation*}
for the second exponential in the Mehler kernel \eqref{eq:Mehler-kernel} we obtain,
by \eqref{eq:C_k-annulus-decomposition-expand}:
  \begin{align*}
    \exp\biggl(2\e^{-t^2} \frac{\la y, \xi \ra}{1 + \e^{- t^{{\red 2}}}}
    \biggr)
    & \leq \exp(|\la y, \xi \ra|)\\
    & \leq \exp(|\langle y, \xi-y\rangle|) \e^{|y|^2}\\
    & \leq \exp\bigl((2^{k + 1} - 1) \alpha t |y| \bigr) \e^{|y|^2}
   \end{align*}
  Combining things, we obtain the estimate in the formulation of the lemma.  
%  can estimate:
%  \begin{equation*}
%    M_{t^2}(y, \xi) \leq \frac{\e^{|y|^2}}{(1 - \e^{-{\red 2}{t^2}})^{\frac{d}2}}
%    \exp\bigl((2^{k + 1} - 1) \alpha t |y| \bigr)
%    \exp\biggl(-\frac{\alpha^2}{2 \e^{2t^2}} (2^k - 1)^2 \biggr). %
%  \end{equation*}
%   Setting $\beta = \frac{\alpha^2}{2 \e^{2T^2}}$ and expanding the last
%   exponential we get:
%   \begin{equation*}
%     M_{t^2}(y, \xi) \leq \frac{\e^{|y|^2}}{(1 - \e^{-{\red 2}{t^2}})^{\frac{d}2}}
%     \exp\bigl((2^{k + 1} - 1) \alpha t |y| \bigr)
%     \exp(- \beta (2^k - 1)^2 ).
%   \end{equation*}
%  Which is as claimed.
  % \eqref{lem:Cone-Gaussians-comparable} gives us
  % by using $|x - y| \leq \alpha t \leq \alpha^2 m(x)$ the following estimate:
  % \begin{equation*}
  %   \e^{|y|^2} \leq \e^{|x|^2}  \e^{\alpha^4} \e^{2\alpha^2}
  % \end{equation*}
  % \begin{align*}
  %   M_{t^2}(y, \xi) &\leq \frac{\e^{-\beta} \e^{|y|^2}}{(1 - \e^{-t^2})^{\frac{d}2}}
  %   \exp\bigl((2^{k + 1} - 1) \alpha (1 + \alpha) \bigr) \e^{\beta
  %     2^{k + 1}} \e^{-\beta 4^k}\\
  %   &\leq \e^{-(\alpha + \beta)} \e^{\alpha^4} \e^{\alpha^2} \frac{ \e^{|x|^2} }{(1 - \e^{-t^2})^{\frac{d}2}}
  %   \exp\bigl(2^{k + 1} \alpha (1 + \alpha) \bigr)  \e^{\beta 2^{k + 1}} \e^{-\beta 4^k}.
  % \end{align*}
  % Which is as claimed.
\end{proof}



\section{The main result}
Our theorem is a small modification of \cite[lemma 1.1]{Pineda2008} with a new proof.
\begin{theorem}\label{thm:Gaussian-maximal-function}
  Let $A, a > 0$. For all $x\in \R^d$ and all $u \in \LHG$ we have
  \begin{equation}
    \label{eq:Maximal-function-cone}
    \sup_{(y, t) \in \Gamma_x^{(A, a)}} |\e^{-t^2 L} u(y)| \lesssim
    \sup_{r > 0} \frac1{\gamma(B_r(x))}\int_{B_r(x)} |u| \, \D\gamma.
  \end{equation}
\end{theorem}
\begin{proof}
  We fix $x\in \R^d$ and $ (y, t) \in \Gamma_x^{(A, a)}$. 
  Set $\alpha = aA$. 
%  By Lemma \ref{lem:m-xy-equivalence} we have 
%  $x \in \Gamma_y^{(\alpha, 1 + \alpha)}$.
%  Set $\Gamma_x^\alpha = \Gamma^{(\alpha, 1 +\alpha)}_x$.
  We will prove \eqref{eq:Maximal-function-cone} by splitting up the
  integration domain into the annuli $C_k$ as defined by \eqref{eq:C_k-annulus-decomposition}:
  \begin{equation}\label{eq:Maximal-function-cone-intermediate-step-1}
    {\red |} \e^{-t^2 L} u(y)| \leq \sum_{k = 0}^\infty I_k(y),
    \:\text{where}\: I_k(y) := \int_{C_k} M_{t^2}(y, \cdot) |u(\cdot)|
    \,\D\gamma.
  \end{equation} 
% in $C_k$
%  in the above and find a
%   suitable upper bound for each integral on the right-hand side which
%   we will denote by $I_k$ for the sake of simplicity.

  %We proceed with applying Lemma
  %\ref{lem:On-diagonal-kernel-estimates-on-Ck}. 
  From $t \leq
  {\red a} m(x)$ we get $t |x| \leq {\red a}$, and by
  Lemma \ref{lem:m-xy-equivalence} we have $t |y| \leq 1 + \alpha$.
  From this and Lemma \ref{lem:On-diagonal-kernel-estimates-on-Ck} we infer, {\red for
  $\xi\in C_k$}, that:
  \begin{equation}
    \label{eq:Mehler-kernel-estimate-one-sided-bound-1}
   M_{t^2}(y, \xi) \leq  \frac{\e^{|y|^2}}{(1 -
      \e^{-{\red 2}{t^2}})^{\frac{d}2}} \exp\bigl((2^{k + 1} - 1) \alpha (1 + \alpha) \bigr)
    \exp\Big(\!-\!\frac{\alpha^2}{2 \e^{2{\red a}^2}} (2^k - 1)^2 \Big).
  \end{equation}
Combining \eqref{eq:Mehler-kernel-estimate-one-sided-bound-1} and Lemma \ref{lem:Cone-Gaussians-comparable}, we obtain
$$  M_{t^2}(y, \xi) \leq  c_k\frac{\e^{|x|^2}}{(1 - \e^{-{\red 2}{t^2}})^{\frac{d}2}}
$$       
where $c_k$ depends only on $A$, $a$, and $t$.
%  where $\beta = $ as before. 
%  \footnote{Hier stond: ``Note that $\beta$
%  is maximal for $\alpha = \frac12$ and after this value, $\beta$ is decreasing.'' Ik snap niet wat je precies bedoelt}
%  Setting $\lambda := \alpha(1 + \alpha)$ we get:
%  \begin{equation}
%    \label{eq:Mehler-kernel-estimate-one-sided-bound-2}
%    M_{t^2}(y, \xi) \lesssim_\alpha  \frac{\e^{|x|^2}}{(1 -
%      \e^{-{t^2}})^{\frac{d}2}} \exp\bigl((2^{k + 1} - 1)\alpha(1 + \alpha) \bigr)
%    \exp(- \beta (2^k - 1)^2 ).
%  \end{equation}
%  \fbox{Implied constant lemma 2}
%  Writing $c_k$ for the constants in the exponentials we get:
%  \begin{equation}
%    \label{eq:Mehler-kernel-estimate-one-sided-bound-2}
%    M_{t^2}(y, \xi) \lesssim_\alpha  \frac{\e^{|x|^2}}{(1 -
%      \e^{-{t^2}})^{\frac{d}2}}
%  \end{equation}
%  Recalling 
%
%By  Lemma \ref{lem:Gaussian-ball-shift-lemma} 
%we get:
%  \begin{equation}
%    \label{eq:Gaussian-ball-Maximal-function-cone-proof-1}
%    \gamma(B_t(x)) \leq V_d d \pi^{-\frac{d}2} t^d \e^{-(t - |x|)^2} .
%  \end{equation}
%  Where we abbreviate $V_d(1)$ with $V_d$.
%  Recall
%and the estimates  
%\begin{equation*}
%    S_d = dV_d, \quad V_d \leq \frac1{\sqrt{\pi}} \biggl(\frac{2\pi e}{d}
%    \biggr)^{\frac{d}{2}}
%  \end{equation*}
%  with $V_d$ the volume of the unit ball in $\R^d$,
%we get: 
%  \begin{equation}
%    \label{eq:Gaussian-ball-Maximal-function-cone-proof-2}
%    \gamma(B_t(x)) {\red \leq \frac12 S_d a^{d} \e^{2a} \e^{-|x|^2}}
%    \leq \frac{d}{{\red 2}\sqrt{\pi}} \biggl(\frac{2{\red \pi} e}{d}
%    \biggr)^{\frac{d}{2}}  {\red a^{d} \e^{2a}}\e^{{\red -|x|^2}} =: C_{d,{\red a}} \e^{{\red -|x|^2}}.
%  \end{equation}
%  This allows us to estimate the remaining unbounded exponential in the
%  Mehler kernel and allow a penalty up to $\e^{-|x|^2}$. 
  %%%%%%% waar wordt dit gebruikt:
%  Furthermore, from $t\leqam(x) \leqa$
%  and the fact that $x\mapsto x/(1-\e^{-x})$ is increasing for $x>0$ we have the following estimate:
% % which will make clear how to handle the time part in the Mehler kernel:
%  \begin{equation*}
%    \frac{t^d}{(1 - \e^{-t^2})^{\frac{d}2}} {\red \ =\ } \biggl(\frac{t^2}{1 -
%      \e^{-t^2}} \biggr)^{\frac{d}2} \leq
%      \biggl(\frac{a^2}{1 -
%      \e^{-a^2}} \biggr)^{\frac{d}2}
%      = \frac{a^d}{(1 -
%      \e^{-a^2})^{\frac{d}2}}.%
%  \end{equation*}
%
% Let % $B' := B(x, 2^{k + 1} \alpha t)$ and
%  $B^\alpha = B(y,
%  \alpha t)$ as before. 
  %In the next step we will bound ....
  %by the maximal function centered at $x$. For this we need to scale
  %up the $C_k$. To this end, note that 
  Also, by \eqref{eq:C_k-annulus-decomposition-expand},
  \begin{equation*}
    |x - \xi| \leq |x - y| + |\xi - y| \leq ({\red A} + 2^{k + 1}\alpha)t .
  \end{equation*}
 {\red It follows that $ C_k$ is contained in $D_k := B_{(A +2^{k + 1}\alpha)t}(x)$.}

 Let us denote the supremum on right-hand side of \eqref{eq:Maximal-function-cone} by $M_\gamma u (x)$.
 Using \eqref{eq:Mehler-kernel-estimate-one-sided-bound-1}, we can bound the integral on the right-hand side of
  \eqref{eq:Maximal-function-cone-intermediate-step-1} by 
  \begin{align*}
    \int_{C_k}  M_{t^2}(y, \cdot) |u(\cdot)| \,\D\gamma & \leq
    {\red c_k} \frac{\e^{|x|^2}}{(1 - \e^{-{\red 2}t^2})^{\frac{d}2}}   \int_{C_k}
    |u| \,\D\gamma\\ 
    &\leq c_k \frac {\e^{|x|^2}} {(1 -
      \e^{-{\red 2}t^2})^{\frac{d}2}} \int_{D_k} |u| \,\D\gamma\\ 
    &\leq c_k  \frac {\e^{|x|^2}}{(1 - \e^{-{\red 2}t^2})^{\frac{d}2}}  \gamma(D_k) M_\gamma u(x)\\
    &\overset{(\dagger)}{\lesssim}_{A,a,d} c_k 
    \frac {{\red t^d}}{(1 - \e^{-{\red 2}t^2})^{\frac{d}2}} {\red   \e^{2((A +2^{k + 1}\alpha)t)|x|}   }M_\gamma u(x)\\
  &\overset{(\dagger\dagger)}{\lesssim}_{A,a,d} c_k 
    {\red  \e^{2^{k + 2}\alpha a}   }M_\gamma u(x),
    %    &\overset{{\red (\dagger\dagger)}}{\leq} c_k C_{d,{\red a}}  2^{k d} \frac{t^d \e^{-t^2}}{(1 -
%      \e^{-{\red 2}t^2})^{\frac{d}2}} \e^{2\alpha} (2\alpha)^d (M_\gamma u)(x) 
  \end{align*}
  where ($\dagger$) uses 
 Lemma \ref{lem:Gaussian-ball-shift-lemma} applied to $D_k$ and 
  ($\dagger\dagger$) uses that $t \leq am(x)$ implies $t|x|\le a$ and $t\le a$, the latter implying
  \begin{equation*}
    \biggl(\frac{t^2}{1 -
      \e^{-t^2}} \biggr)^{\frac{d}2} \leq
      \biggl(\frac{a^2}{1 -
      \e^{-a^2}} \biggr)^{\frac{d}2}
  \end{equation*}
  (note that $x/(1-\e^{-x})$ is increasing).
  
%  {\red so that 
%  $2^{k + 1}  {\max\{A,\alpha\}} t \leq2^{k + 1}  a{\max\{A,\alpha\}}m(x)$
%  and therefore 
%  $$ \gamma(D_k(x)) \lesssim_{a,A,d}
%  \e^{ 2^{k + 2}  a\max\{A,\alpha\}} \e^{-|x|^2}$$
%  by Lemma \ref{lem:Gaussian-ball-shift-lemma}.
%  }

  Inserting the depenceny of $c_k$ upon $k$ as coming from 
  \eqref{eq:Mehler-kernel-estimate-one-sided-bound-1} and using that $t\le a$, 
  we can then bound the maximal function as follows: 
  \begin{align*}
    |\e^{-t^2 L} u(y)| &= \sum_{k = 0}^\infty I_k
    \lesssim_{A,a,d} M_\gamma u(x) \sum_{k = 0}^\infty 2^{kd} \e^{\alpha(1+\alpha) 2^{k + 1}} 
    \e^{-\frac{\alpha^2}{2 \e^{2{\red a}^2}}  4^k}{\red \e^{ 2^{k + 2} \alpha a}}
  \end{align*}
%  Wrapping it up, we have that:
%  \begin{equation*}
%    \e^{-t^2 L} |u(y)| \lesssim \frac1{\gamma(B_r(x))}\int_{B_r(x)} |u| \, \D\gamma.
%  \end{equation*}
%  With implied constant
Evidently the sum on the right-hand side converges.
 \end{proof} 
  \begin{center}

  {\tt \red (Tot hier alles gecontroleerd)}
  
  \end{center}

  
  \begin{equation*}
    \sum_{k = 0}^\infty 2^{kd} \e^{-C 4^k} = \sum_{k = 0}^\infty x^{kd} \e^{-C x^{2k}}
  \end{equation*}
  Noting for $x \geq 1$ that $\exp(-C x^{2k}) \leq \exp(-C k x^2)$,
  thus,
  \begin{equation*}
    \sum_{k = 0}^\infty 2^{kd} \e^{-C 4^k} \leq \sum_{k = 0}^\infty
    x^{kd} (\e^{-C x^2})^k = \sum_{k = 0}^\infty (x^{d} \e^{-C x^2})^k
  \end{equation*}
  Here $x = 2$, so
  \begin{equation*}
    \sum_{k = 0}^\infty 2^{kd} \e^{-C 4^k} \leq \sum_{k = 0}^\infty
    (2^d \e^{-4 C})^k
  \end{equation*}
  If $2^d < \e^{4C}$, that is whenever $d \log 2 < 4C$, we can compute
  using the geometric series that
  \begin{equation*}
    \sum_{k = 0}^\infty 2^{kd} \e^{-C 4^k} \leq \frac1{1 - 2^d
      \e^{-4C}} = \frac{\e^{4C}}{\e^{4C} - 2^d}
  \end{equation*}

\bibliographystyle{plain}
\bibliography{}

\end{document}
