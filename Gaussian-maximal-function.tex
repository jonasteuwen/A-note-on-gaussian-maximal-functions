\RequirePackage[l2tabu, orthodox]{nag}
\documentclass[a4paper,oneside,10pt]{amsproc}

\usepackage{fixltx2e}
\usepackage[all, error]{onlyamsmath}
\usepackage{fixmath} % http://ctan.org/pkg/fixmath
\usepackage{refcheck}
% \usepackage{gitinfo} % Information about the version
\norefnames
% \showrefnames
\usepackage{microtype}
\usepackage{amsmath}
\usepackage{amsthm}
\usepackage[x11names]{xcolor}%
\usepackage{textcomp}
\usepackage[english]{babel}
\usepackage{xfrac} % Nice / fractions
\usepackage[utf8]{inputenc}
\usepackage[T1]{fontenc}
\usepackage[strict=true]{csquotes} % Needs to be loaded *after* inputenc

%% Tekstiin ja taulukoihin liittyvi� lis�paketteja
\usepackage{enumerate} 
\usepackage{booktabs}% Better tables

%% Matematiikkaan liittyvi� lis�paketteja

\usepackage{latexsym}
\usepackage{bbm}
\usepackage{enumitem}
\usepackage[bitstream-charter]{mathdesign}%

\usepackage[bookmarks,colorlinks,breaklinks]{hyperref} % Add hyperref type links in the document, colors
\definecolor{dullmagenta}{rgb}{0.4,0,0.4} % #660066
\definecolor{darkblue}{rgb}{0,0,0.4}%
\hypersetup{linkcolor=red,citecolor=blue,filecolor=dullmagenta,urlcolor=darkblue} % coloured links


\usepackage{etoolbox}
\newcounter{hints}
\renewcommand{\thehints}{\roman{hints}}
\newcommand{\hintedrel}[2][]{%
  \stepcounter{hints}%
  \if\relax\detokenize{#1}\relax\else\csxdef{hint@#1}{\thehints}\fi
  \mathrel{\overset{\textrm{(\thehints)}}{\vphantom{\le}{#2}}}%
}
\newcommand{\restarthintedrel}{\setcounter{hints}{0}}
\newcommand{\hintref}[1]{\csuse{hint@#1}}
\AfterEndEnvironment{align*}{\setcounter{hints}{0}}% Resets numrel at the end of align*
\AfterEndEnvironment{align}{\setcounter{hints}{0}}% Resets numrel at the end of align
\AfterEndEnvironment{equation*}{\setcounter{hints}{0}}% Resets numrel at the end of equation*
\AfterEndEnvironment{equation}{\setcounter{hints}{0}}% Resets numrel at the end of equation



\makeatletter

\@namedef{subjclassname@2010}{%

  \textup{2010} Mathematics Subject Classification}

\makeatother

% \linespread{1.3}


%%%%%%%% Teoreemaymp�rist�t
% HUOM! K�ytet��n amsthm-pakettia, eik� theorem-pakettia, jos
% theorem-pakettia k�ytet��n t�ytyy my�s n�it� muuttaa

% Perusteoreematyyli, lause ja lemma numeroidaan sectionin
% mukaan. Propositio numeroidaan lauseen numeroinnilla

\theoremstyle{plain}
% \newtheorem*{thmA6.6}{[A, Theorem 6.6]}
\newtheorem*{problem}{Problem}


\swapnumbers
\newtheorem{theorem}{Theorem}
\newtheorem{definition}{Definition}
\newtheorem{lemma}{Lemma}
\newtheorem{corollary}{Corollary}
\newtheorem{proposition}{Proposition}
\theoremstyle{remark}
\renewcommand{\qedsymbol}{\ensuremath{\blacksquare}}
\newtheorem*{remark}{Remark}
\newtheorem*{acks}{Acknowledgements}
\newtheorem*{examples}{Examples}


% M��ritelm�tyyli

\theoremstyle{definition} 
% M��ritelm�, jolla on oma juokseva numerointi 
% \newtheorem{definition}[theorem]{Definition}
\newtheorem*{def*}{Definition}

% Huomautustyyli


% Omat k�skyt (esimerkkin� joukkosymbolit)

\newcommand{\E}{\mathbb{E}}
\newcommand{\Sp}{\mathbb{S}}
\newcommand{\prob}{\mathbb{P}}
\newcommand{\D}{\,\textup{d}}
\newcommand{\Dn}{\textup{d}} % One without space.
\newcommand{\Dt}{\,\frac{\textup{d} t}{t}}
\newcommand{\Ds}{\,\frac{\textup{d} s}{s}}
\newcommand{\DyDt}{\frac{\textup{d} y \, \textup{d} t}{t^{n+1}}}
\newcommand{\Dd}{\mathscr{D}}
\newcommand{\Tt}{\mathscr{T}}
\newcommand{\Cc}{\mathscr{C}}
\newcommand{\Bb}{\mathscr{B}}
\newcommand{\Rr}{\mathscr{R}}
\newcommand{\Ff}{\mathscr{F}}
\newcommand{\Ll}{\mathscr{L}}
\newcommand{\Mm}{\mathscr{M}}
\newcommand{\hh}{\mathfrak{h}}
\newcommand{\ttt}{\mathfrak{t}}
\newcommand{\la}{\langle}
\newcommand{\ra}{\rangle}
\newcommand{\Rad}{\textup{Rad}}
\newcommand{\Car}{\textup{Car}}
\newcommand{\BMO}{\textup{BMO}}
\newcommand{\loc}{\textup{loc}}
\newcommand{\LH}{{L^2_\mu}}
\newcommand{\LHG}{{L^2_\gamma}}

%% Bar int
\def\Xint#1{\mathchoice
  {\XXint\displaystyle\textstyle{#1}}%
  {\XXint\textstyle\scriptstyle{#1}}%
  {\XXint\scriptstyle\scriptscriptstyle{#1}}%
  {\XXint\scriptscriptstyle\scriptscriptstyle{#1}}%
  \!\int}
\def\XXint#1#2#3{{\setbox0=\hbox{$#1{#2#3}{\int}$ }
    \vcenter{\hbox{$#2#3$ }}\kern-.535\wd0}}
\def\ddashint{\Xint=}
\def\dashint{\Xint-}

\def\LI{{L^1_\gamma}}


%% Symbols
\renewcommand{\bar}[1]{\overline#1}
\renewcommand{\vec}[1]{\boldsymbol{\mathbf{#1}}}
\renewcommand{\leq}{\leqslant}
\renewcommand{\Im}{\operatorname{Im}}
\renewcommand{\Re}{\operatorname{Re}}
\renewcommand{\leq}{\leqslant}
\renewcommand{\geq}{\geqslant}
\renewcommand{\epsilon}{\varepsilon}
\renewcommand{\emptyset}{\varnothing}
\newcommand{\Fo}{\mathcal{F}}
\newcommand{\R}{\mathbf R}
\newcommand{\C}{\mathbf C}
\newcommand{\N}{\mathbf N}
\newcommand{\T}{\mathbb T}
\newcommand{\Z}{\mathbf Z}
\newcommand{\B}{\mathcal B}
\newcommand{\e}{\mathrm{e}} %Roman e for exponentials

\renewcommand{\leq}{\leqslant} % I like this better, change as you
% please.
\renewcommand{\geq}{\geqslant} % Same.
\DeclareMathOperator{\supp}{supp}
\newcommand{\Dg}{\frac{\textup{d}\gamma (y)}{\gamma (B(y,t))}}
\newcommand{\Dmu}{\frac{\textup{d}\mu (y)}{\mu (B(y,t))}}


\renewcommand{\Re}{\operatorname{Re}}
\renewcommand{\Im}{\operatorname{Im}}
\renewcommand{\bar}{\overline}

\usepackage{tikz}

\def\lemmaautorefname{Lemma}
\def\definitionautorefname{Definition}
\def\theoremautorefname{Theorem}
\def\corollaryautorefname{Corollary}


%% Bibliography
\usepackage[backend=biber,doi=false,url=false,isbn=false]{biblatex}
\bibliography{~/Documents/BibTeX/library.bib}

\newbibmacro{string+doiurlisbn}[1]{%
  \iffieldundef{doi}{%
    \iffieldundef{url}{%
      \iffieldundef{isbn}{%
        \iffieldundef{issn}{%
          #1%
        }{%
          \href{http://books.google.com/books?vid=ISSN\thefield{issn}}{#1}%
        }%
      }{%
        \href{http://books.google.com/books?vid=ISBN\thefield{isbn}}{#1}%
      }%
    }{%
      \href{\thefield{url}}{#1}%
    }%
  }{%
    \href{http://dx.doi.org/\thefield{doi}}{#1}%
  }%
}

\DeclareFieldFormat{title}{\usebibmacro{string+doiurlisbn}{\mkbibemph{#1}}}
\DeclareFieldFormat[article,incollection]{title}%
{\usebibmacro{string+doiurlisbn}{\mkbibquote{#1}}}



\title[Gaussian estimates]{Gaussian estimates}
\author{Mikko Kemppainen}
\address{Department of Mathematics and Statistics, University of Helsinki,
  Gustaf H�llstr�min katu 2b, FI-00014 Helsinki, Finland}
\email{mikko.k.kemppainen@helsinki.fi}

\author{Jonas Teuwen}%
\address{Delft Institute of Applied Mathematics,
  Delft University of Technology, P.O. Box 5031, 2600 GA Delft, The
  Netherlands} \email{j.j.b.teuwen@tudelft.nl}%
\urladdr{http://fa.its.tudelft.nl/~teuwen/}%
\thanks{}%
\date{\today}



\begin{document}

\begin{abstract}
  The required estimates! An attempt! A good attempt! I hope!
\end{abstract}

% \subjclass[2010]{42B25 (Primary); 46E40 (Secondary)}
% \keywords{R-bounds, dyadic cubes}

\maketitle
\section{The Mehler kernel and friends}
\subsection{Notation}
To begin, let us fix some notation. As is common, we use $N$ to
represent a positive integer. Making this more precise, $N$ is taken
out of $\Z_+ = \{1, 2, 3, \dots\}$. In the same way we cast letters
that denote the number of dimensions, e.g.\ $d$ in $\R^d$ as a
positive integer.

The capital letter $T$ we use to denote a ``time'' endpoint, e.g.,
when writing $t$ in $(0, T]$.

\subsection{Setting}
Given the Ornstein-Uhlenbeck operator $L$ defined as:
\begin{equation}
  \label{eq:Ornstein-Uhlenbeck-operator}
  L = -\frac12 \Delta + x \cdot \nabla,
\end{equation}
we have the Mehler kernel (See e.g., \cite{Sjogren1997}) as the
Schwartz kernel associated to the Ornstein-Uhlenbeck semigroup
$\e^{-tL}$. That is, 
\begin{equation}
  \label{eq:Ornstein-Uhlenbeck-semigroup-integral}
  \e^{-tL} u(x) = \int_{\R^d} M_t(x, \cdot) u \, \D\gamma.
\end{equation}

\subsection{Rewriting the Mehler kernel}
There are many ways to write the Mehler kernel, of which the most
common ones computed in \textcite{Sjogren1997}. In many calculations
that will follow it turns out that $t^2$ is comparable to $L$ in ways
that will be made precise when necessary. For this reason we will usually
consider $\e^{-t^2 L}$ instead of $\e^{-t L}$. 

As one tries to exploit the observation that the kernel $M_t$
associated to \eqref{eq:Ornstein-Uhlenbeck-semigroup-integral} should be
symmetric in its arguments one will find a particular useful
representation to be:
\begin{equation}
  \label{eq:Mehler-kernel-1}
  M_{t^2}(y, \xi) = \frac{\exp\biggl(-\dfrac12 \dfrac{|y - \xi|^2}{1
      - \e^{-t^2}}  \biggr)}{(1 - \e^{-t^2})^{\frac{d}2}}
  \frac{\exp\biggl(-\dfrac12 \dfrac{|y + \xi|^2}{1 + \e^{-t^2}}
    \biggr)}{(1 + \e^{-t^2})^{\frac{d}2}} \e^{|y|^2} \e^{|\xi|^2}.
\end{equation}
The second version will be found useful when decomposing the
integration area into annuli.
\begin{equation}
  \label{eq:Mehler-kernel-2}
  M_{t^2}(y, \xi) = \frac{\exp\biggl(-\e^{-t^2} \dfrac{|y - \xi|^2}{1
      - \e^{-2 t^2}}  \biggr)}{(1 - \e^{-t^2})^{\frac{d}2}}
  \frac{\exp\biggl(\e^{-t^2} \dfrac{|y|^2 + |\xi|^2}{1 + \e^{-t^2}}
    \biggr)}{(1 + \e^{-t^2})^{\frac{d}2}}.
\end{equation}

\subsection{Some fine lemmata and definitions}
We recall the lemma from \cite[lemma 2.3]{Maas2011b} which first,
 --although implicitly-- appeared in \cite{Mauceri2007}.
\begin{lemma}\label{lem:m-xy-equivalence-no-cone}
  Let $\alpha$ and $\beta$ be given strictly positive numbers. We then
  have the following for $x$ and $y$ in $\R^d$:
  \begin{enumerate}
  \item If $t \leq \alpha m(x)$ and $|x - y| < \beta t$, then $t
    \leq (1 + \alpha \beta) m(y)$;
  \item Likewise, if $|x - y| < \beta m(x)$, then $m(x) \leq (1 +
    \beta) m(y)$ and $m(y) \leq 2 (1 + \beta) m(x)$. 
  \end{enumerate}
\end{lemma}
We can also write this using the Gaussian cone $\Gamma_x^{(a, A)}$
which is given as:
\begin{equation}
  \label{eq:Gaussian-cone}
  \Gamma_x^{(A, a)} := \Gamma_x^{(A, a)}(\gamma) := \{(y, t) \in
  \R^d_+ : |x - y| < At \:\text{and}\: t \leq a m(x)\}.
\end{equation}
We will also write $\Gamma_x^a$ for $\Gamma_x^{(1, a)}$. Then we can
in from \autoref{lem:m-xy-equivalence-no-cone} that:
\begin{lemma}\label{lem:m-xy-equivalence}
  Let $a > 0$ and $x, y$ in $\R^d$. If $y$ is in $\Gamma_x^{a}$ then
  $m(x) \leq (1 + a) m(y)$ and $m(y) \leq 2(1 + a) m(x)$.
\end{lemma}
We will also use the global/local region dichotomy and define it as
follows:
\begin{definition}
  Given $\tau > 0$, the set $N_\tau$ is given as:
  \begin{equation}
    \label{eq:Definition-local-region}
    N_\tau := \{(x, \xi) \in \R^{2d} : |x - \xi| \leq \tau m(x) \}.
  \end{equation}
  Often it eases the notation to work with the set $N_\tau(B)$ given
  for $B := B_r(\xi)$ as:
  \begin{equation}
    \label{eq:Definition-local-region-ball}
    N_\tau(x) := \{\xi \in \R^d : |x - \xi| \leq \tau m(x) \}.
  \end{equation}
  When we partition the space into $N_\tau$ and its complement, we
  call the part belonging to $N_\tau$ the \emph{local region} and the
  part belonging to $\complement N_\tau$ the \emph{global region}.
\end{definition}
The next lemma will come useful when we want to cancel exponential
growth in one variable with exponential decay in the other as long
both variables are in a Gaussian cone.
\begin{lemma}\label{lem:Cone-Gaussians-comparable}
  Let $(y, t) \in \Gamma_x^{(A, a)}$. Then the Gaussian in $x$
  respectively $y$ are comparable. More precisely this means that,
  \begin{equation*}
    \e^{-|x|^2} \simeq \e^{-|y|^2}.
  \end{equation*}
  Workable constants are given in \label{eq:Cone-Gaussians-comparable-1}
  and \label{eq:Cone-Gaussians-comparable-2}.
\end{lemma}

\begin{proof}
  Let $(y, t) \in \Gamma_x^{(A, a)}$. Unwrapping the definition we
  have
  \begin{equation*}
    |x - y| < A t \:\text{and}\: t \leq a m(x).
  \end{equation*}
  Hence, by the inverse triangle inequality we get,
  \begin{align*}
    |y|^2 &\leq (aA m(x) + |x|)^2\\
    &= (aA m(x))^2 + 2 a A m(x) |x| + |x|^2\\
    &\leq (aA m(x))^2 + 2 a A + |x|^2\\
    &\leq (aA)^2 + 2 a A + |x|^2.
  \end{align*}
  Therefore,
  \begin{equation}
    \label{eq:Cone-Gaussians-comparable-1}
    \e^{-|y|^2} \geq \e^{-(aA)^2} \e^{-2 aA} \e^{-|x|^2}.
  \end{equation}
  By \autoref{lem:m-xy-equivalence-no-cone} we have $t \leq (1 + aA)
  m(y)$
  \begin{align*}
    |x|^2 &\leq ((1 + aA) m(y) + |y|)^2\\
    &= ((1 + aA) m(y))^2 + 2(1 + aA)m(y)|y| + |y|^2\\
    &\leq (1 + aA)^2 + 2(1 + aA) + |y|^2.
  \end{align*}
  Therefore,
  \begin{equation}
    \label{eq:Cone-Gaussians-comparable-2}
    \e^{-|x|^2} \geq \e^{-(1 + aA)^2} \e^{-2(1 + aA)} \e^{-|y|^2}.
  \end{equation}
  Summarizing we thus have that,
  \begin{equation*}
    \e^{-|x|^2} \simeq \e^{-|y|^2},
  \end{equation*}
  as required.
\end{proof}



\begin{lemma}
  \label{def:Global-region-cone-lemma}
  Let $x, y$ and $z$ in $\R^d$. Set
  \begin{equation*}
    \tau = \frac12 (1 + 2aA)(1 + aA),
  \end{equation*}
  Such that if $|y - z| > \tau m(y)$ (i.e., $(y, z) \notin N_\tau$)
  and $(y, t) \in \Gamma_x^{(A, a)}$ then $|x - z| > \frac12 m(x)$
  (i.e., $(x, z) \notin N_{\frac12}$).
\end{lemma}
\begin{proof}
  We assume that $(y, z) \notin N_\tau$ and $(y, t) \in \Gamma_x^{(A,
    a)}$. Written out this gives by \eqref{eq:Definition-local-region}
  the inequality $|y - z| > \tau m(y)$, and by
  \eqref{eq:Gaussian-cone} the inequality $|x - y| < aA m(x)$. Note
  that the latter inequality together with
  \autoref{lem:m-xy-equivalence-no-cone} yields,
  \begin{equation}
    \label{eq:Global-region-cone-lemma-proof-1}
    \frac12 \frac1{1 + aA} m(y) \leq m(x) \leq (1 + aA) m(y).
  \end{equation}
  Combining we get $|x - y| < aA (1 + aA) m(y)$. Now we are in
  position to apply the triangle inequality:
  \begin{equation*}
    |x - z| \geq |y - z| - |x - y| > \tau m(y) - aA (1 + aA) m(y).
  \end{equation*}
  As we require an lower bound in terms of $m(x)$ and not $m(y)$, we
  again apply \eqref{eq:Global-region-cone-lemma-proof-1} to obtain:
  \begin{align*}
    |x - z| \geq |y - z| - |x - y| &> \tau m(y) - aA m(y)\\
    &\geq \tau \frac1{1 + aA} m(x) - aA m(x).
  \end{align*}
  Therefore a $\tau$ as requested should satisfy:
  \begin{equation*}
    \tau \frac1{1 + aA} - aA = \frac12.
  \end{equation*}
  That is,
  \begin{equation*}
    \tau = \frac12 (1 + 2aA)(1 + aA).
  \end{equation*}
  Which is as requested.
\end{proof}
The set $t < m(x)$ is used in the definition of the Gaussian tent
spaces and we will give it the name $D$, that is:
\begin{equation}
  \label{eq:Definition-cut-off-set-D}
  D := \{(x, t) \in \R^d_+ : t < m(x) \}.
\end{equation}

\section{On-diagonal estimates}
\subsection{Kernel estimates}
We begin with a technical lemma which will be useful on several
occasions.
\begin{lemma}\label{lem:Time-part-Mehler-time-transform}
  Let $t$ in $(0, T]$ and $\alpha > 1$. Then,
  \begin{equation}\label{eq:Time-part-Mehler-time-transform}
    \alpha \e^{-T^2} \leq \frac{1 - \e^{-t^2}}{1 -
      \e^{-\frac{t^2}{\alpha}}} \leq \alpha.
  \end{equation}
\end{lemma}
\begin{proof}
  Let $t$ in $(0, T]$ and $\alpha > 1$. Applying the mean value theorem to the function $f(\xi) = \xi^\alpha$ gives, for $0 < \xi < \xi'$:
  \begin{equation*}
    f(\xi) - f(\xi') = \alpha \hat{\xi}^{\alpha - 1} (\xi - \xi') \text{ for some $\hat \xi$ in $[\xi, \xi']$}.
  \end{equation*}
  Picking $\xi = 1$ and $\xi' = \e^{-\frac{t^2}{\alpha}}$ gives
  \begin{equation}
    \label{eq:Time-part-Mehler-time-transform-proof-1}
    \frac{1 - \e^{-t^2}}{1 - \e^{-\frac{t^2}{\alpha}}} = \alpha \hat{\xi}^{\alpha - 1} \:\text{for some}\: \hat{\xi} \:\text{in}\:  [\e^{-\frac{t^2}{\alpha}}, 1].
  \end{equation}
  Using $t \leq T$ we get:
  \begin{equation*}
    \alpha \e^{-T^2} \leq \alpha \exp\Bigl(-t^2 \frac{\alpha -
      1}{\alpha} \Bigr)
    \hintedrel[rel1]{\leq}  \frac{1 - \e^{-t^2}}{1 - \e^{-\frac{t^2}{\alpha}}}
  \end{equation*}

  Where (\hintref{rel1}) uses
  \eqref{eq:Time-part-Mehler-time-transform-proof-1} and the
  monotonicity of $\xi \mapsto \alpha \xi^{\alpha - 1}$.

  Hence, 
  \begin{equation*}
   \alpha \e^{-T^2} \leq \lim_{t \downarrow 0} \frac{1 - \e^{-t^2}}{1
     - \e^{-\frac{t^2}{\alpha}}} \overset{\searrow}{=} \alpha.
  \end{equation*}


 This
  completes the proof.
\end{proof}

The following lemma will be useful when transfering estimates from
$M_{\frac{t^2}{\alpha}}$ to $M_{t^2}$. It follows from the mean value
theorem applied to $\xi \mapsto \xi^\alpha$.
\begin{lemma}\label{lem:Exponential-estimates}
  For $C, T > 0, \alpha > 1, t$ in $(0, T]$ and all $x, y$ in $\R^d$ we have that
  \begin{equation}
    \label{eq:Exponential-estimates-1}
    \exp \biggl (-C \frac{|x - y|^2}{1 - \e^{-\frac{t^2}\alpha}}
    \biggr ) \leq  \exp \biggl (-C \frac{\alpha}{\e^{T^2}} \frac{|x - y|^2}{1 -
      \e^{-t^2}} \biggr ).
  \end{equation}
\end{lemma}
\begin{proof}
  Let $t$ in $(0, T]$ and $\alpha > 1$. By
  \autoref{lem:Time-part-Mehler-time-transform} we get
  \begin{equation*}
    \alpha \e^{-T^2} \leq \alpha \exp\Bigl(-t^2 \frac{\alpha -
      1}{\alpha} \Bigr)
    \hintedrel[rel1]{\leq} \frac{1 - \e^{-t^2}}{1 -
      \e^{-\frac{t^2}{\alpha}}} \leq \lim_{t \downarrow 0} \frac{1 -
      \e^{-t^2}}{1 - \e^{-\frac{t^2}{\alpha}}} = \alpha.
  \end{equation*}
  Noting that $\e^{-t^2} \geq \e^{-T^2}$ we get,
  \begin{align*}
    \exp \biggl (-C \frac{|x - y|^2}{1 - \e^{-t^2}} \biggr )
    &\leq \exp \biggl (-C \frac{|x - y|^2}{1 -
      \e^{-t^2}} \frac{1 -
      \e^{-t^2}}{1 - \e^{-\frac{t^2}\alpha}} \biggr )\\
    &\leq \exp \biggl (-C \frac{\alpha}{\e^{T^2}} \frac{|x - y|^2}{1 -
      \e^{-t^2}} \biggr ).
  \end{align*}
\end{proof}

Later on we will study kernel estimates of kernels related to the
Mehler kernel, but our first lemma is about estimating
$M_{\frac{t}\alpha}$ in terms of $M_t$.
\begin{lemma}\label{lem:Kernel-estimates-1}
 Let $\alpha \geq 2 \e^{T^2}$, $t$ in $(0, T]$ and $x, y$ in $\R^d$.
 Finally let $C > 0$ be independent on $x, y$ and $t$. If $t
 (|x| \vee |y|) \leq C$:
  \begin{equation}
    \label{eq:Kernel-lemma-1-estimate} 
    M_{\frac{t^2}{\alpha}}(x, y) \leq \alpha^{\frac{d}2} \e^{\frac\alpha2 C^2} \exp\biggl (-\frac{\alpha}{2 \e^{T^2}} \frac{|x - y|^2}{1 - \e^{-t^2}} \biggr ) M_{t^2}(x, y).
  \end{equation}
\end{lemma}
\begin{proof}
  First note that \eqref{eq:Time-part-Mehler-time-transform} tells
  that:
  \begin{equation*}
    \alpha \e^{-T^2} \leq \frac{1 - \e^{-t^2}}{1 -
      \e^{-\frac{t^2}{\alpha}}} \leq \alpha.
  \end{equation*}
  Note that since
  \begin{equation*}
    \frac{1}{1 + \e^{-t^2}} - \frac{1}{1 + \e^{-\frac{t^2}{\alpha}}}
    \geq \frac{1}{1 + \e^{-\frac{t^2}{\alpha}}} - \frac{1}{1 +
      \e^{-\frac{t^2}{\alpha}}} = 0.
  \end{equation*}
  So that
  \begin{align*}
    \exp \biggl (-\frac12 \frac{|x + y|^2}{1 + \e^{-\frac{t^2}{\alpha}}}
    \biggr ) &\exp \biggl (\frac12 \frac{|x + y|^2}{1 + \e^{-t^2}}
    \biggr )\\
    &\leq \exp \biggl (\frac12 \biggl[\frac{1}{1 +
      \e^{-t^2}} - \frac{1}{1 + \e^{-\frac{t^2}{\alpha}}}
    \biggr] |x + y|^2 \biggr)\\
    &\leq \exp \biggl (\frac12 \frac1{t^2}\biggl[\frac{1}{1 +
      \e^{-t^2}} - \frac{1}{1 + \e^{-\frac{t^2}{\alpha}}}
    \biggr] t^2 |x + y|^2 \biggr).
  \end{align*}
  Next,
  \begin{align*}
    \lim_{t \to 0} \frac1{t^2}\biggl[\frac{1}{1 +
      \e^{-t^2}} - \frac{1}{1 + \e^{-\frac{t^2}{\alpha}}} \biggr] &=
    \lim_{t \to 0} \frac1{2t} \biggl[\frac{2t \e^{-t^2}}{(1 + \e^{-t^2})^2} -
    \frac1\alpha \frac{2t \e^{-\frac{t^2}{\alpha}}}{(1 +
      \e^{-\frac{t^2}{\alpha}})^2} \biggr]\\
    &= \lim_{t \to 0} \biggl[\frac{\e^{-t^2}}{(1 + \e^{-t^2})^2} -
    \frac1\alpha \frac{\e^{-\frac{t^2}{\alpha}}}{(1 +
      \e^{-\frac{t^2}{\alpha}})^2} \biggr]\\
    &= \frac{1}{4} \biggl(1 - \frac1\alpha \biggr).
  \end{align*}
  So that
  \begin{align*}
    \exp \biggl (-\frac12 \frac{|x + y|^2}{1 + \e^{-\frac{t^2}{\alpha}}}
    \biggr ) &\exp \biggl (\frac12 \frac{|x + y|^2}{1 + \e^{-t^2}}
    \biggr )\\
    &\leq \exp \biggl (\frac18 \biggl(1 - \frac1\alpha \biggr) t^2 |x +
    y|^2 \biggr)\\
    &\leq \exp \biggl (\frac14 t^2 |x|^2 \biggr) \exp \biggl (\frac14
    t^2 |y|^2 \biggr)  \exp \biggl (-\frac{t^2}\alpha \frac18 |x + y|^2 \biggr).
  \end{align*}
  So using \autoref{lem:Exponential-estimates} and equation
  \eqref{eq:Exponential-estimates-1} we get
  \begin{align*}
    \frac{M_{\frac{t^2}{\alpha}}(x, y)}{M_{t^2}(x, y)} &\leq \alpha^{\frac{d}2}  \exp\biggl(\dfrac12 \dfrac{|x - y|^2}{1
      - \e^{-t^2}}  \biggr) \exp\biggl(\dfrac12 \dfrac{|x + y|^2}{1
      + \e^{t^2}} \biggr)\\
    &\quad \times \exp\biggl(-\dfrac12 \dfrac{|x - y|^2}{1
      - \e^{-\frac{t^2}{\alpha}}}  \biggr) \exp\biggl(-\dfrac12
    \dfrac{|x + y|^2}{1 + \e^{-\frac{t^2}{\alpha}}} \biggr)\\
    &\leq \alpha^{\frac{d}2} \exp \biggl (\frac12 \biggl[1 -\frac{\alpha}{2\e^{T^2}} \biggr] \frac{|x - y|^2}{1 - \e^{-t^2}}
    \biggr ) \exp \biggl (-\frac{\alpha}{2\e^{T^2}} \frac{|x - y|^2}{1 - \e^{-t^2}}
    \biggr )\\
    &\quad \times  \exp\biggl(\dfrac12 \dfrac{|x + y|^2}{1 + \e^{t^2}}
    \biggr) \exp\biggl(-\dfrac12 \dfrac{|x + y|^2}{1 + \e^{-\frac{t^2}{\alpha}}} \biggr)\\
  \end{align*}
  Thus that,
  \begin{align*}
    \frac{M_{\frac{t^2}{\alpha}}(x, y)}{M_{t^2}(x, y)} &\leq \alpha^{\frac{d}2}
    \exp \biggl (\frac12 \biggl[1
    -\frac{\alpha}{2\e^{T^2}} \biggr] \frac{|x - y|^2}{1 - \e^{-t^2}}
    \biggr ) \exp \biggl (-\frac{\alpha}{2\e^{T^2}} \frac{|x - y|^2}{1 - \e^{-t^2}}
    \biggr )\\
    &\quad \times \exp \biggl (t^2 \frac{|x|^2 + |y|^2}4 \biggr)
    \exp\biggl(-\frac{t^2}\alpha \frac{|x + y|^2}8 \biggr).
  \end{align*}
  Thus, for $\alpha \geq 2 \e^{T^2}$ we obtain
  \begin{equation*}
    \frac{M_{\frac{t^2}{\alpha}}(x, y)}{M_{t^2}(x, y)} \leq \alpha^{\frac{d}2} \exp\biggl(-\frac{\alpha}{2\e^{T^2}} \frac{|x - y|^2}{1 - \e^{-t^2}} \biggr) \exp \biggl (t^2 \frac{|x|^2 + |y|^2}4 \biggr)
    \exp\biggl(-\frac{t^2}\alpha \frac{|x + y|^2}8 \biggr).
  \end{equation*}
  Using $t (|x| \vee |y|) \leq C$ we get
  \begin{equation*}
    \frac{M_{\frac{t^2}{\alpha}}(x, y)}{M_{t^2}(x, y)} \leq \alpha^{\frac{d}2}
    \e^{\frac{C^2}2} \exp\biggl(-\frac{\alpha}{2\e^{T^2}} \frac{|x - y|^2}{1 - \e^{-t^2}} \biggr)
    \exp\biggl(-\frac{t^2}\alpha \frac{|x + y|^2}8 \biggr).
  \end{equation*}
  Which is as asserted.
\end{proof}

\subsection{On-diagonal kernel estimates on annuli}
As is common in harmonic analysis, we will often wish to decompose
$\R^d$ into sets on which certain phenomena are easier to handle. We
will decompose the space into annuli $C_k$. These are given as:
\begin{equation}
  \label{eq:C_k-annulus-decomposition}
  C_k(B) := C_k =
  \begin{cases}
    2B &\text{if $k = 0$,}\\
    2^{k + 1}B \setminus 2^k B &\text{for $k \geq 1$.}
  \end{cases}
\end{equation}
\begin{lemma}\label{lem:On-diagonal-kernel-estimates-on-Ck}
  Given $a > 0$, let $B = B_{at}(y)$ and $\xi$ in $C_k$. Then we have
  for $k \geq 1$:
  \begin{equation*}
    M_{t^2}(y, \xi) \leq \frac{\e^{|y|^2}}{(1 - \e^{-t^2})^{\frac{d}2}}
    \exp\biggl(-\frac{a^2}{2} 4^{k + 1} \biggr) \exp\bigl(2^{k + 1} a t |y|
    \bigr).
  \end{equation*}
  and for $k = 0$:
  \begin{equation*}
    M_{t^2}(y, \xi) \leq \frac{\e^{|y|^2}}{(1 -
      \e^{-t^2})^{\frac{d}2}} \exp\bigl(2^{k + 1} a t |y| \bigr).
  \end{equation*}
\end{lemma}
\begin{proof}
  Let $B = B_{at}(y)$ and let $C_k$ be as in \eqref{eq:C_k-annulus-decomposition}.
  So, whenever $\xi$ is in $C_k$, we get for $k \geq 1$:
  \begin{equation*}
    2^k a t < |\xi - y| \leq 2^{k + 1} a t.
  \end{equation*}
  For the case $k = 0$ we get $|\xi - y| \leq 2 a t$.
  Recall the Mehler kernel of \eqref{eq:Mehler-kernel-2}:
  \begin{equation*}
    M_{t^2}(y, \xi) = \frac{\exp\biggl(-\e^{-t^2} \dfrac{|y - \xi|^2}{1
        - \e^{-2t^2}}  \biggr)}{(1 - \e^{-t^2})^{\frac{d}2}}
    \frac{\exp\biggl(\e^{-t^2} \dfrac{|y|^2 + |\xi|^2}{1 + \e^{-t^2}}
      \biggr)}{(1 + \e^{-t^2})^{\frac{d}2}}.
  \end{equation*}
  Note that by the triangle inequality there holds that:
  \begin{equation*}
    |\xi|^2 \leq |y - \xi|^2 + |y|^2 + 2 |y||y - \xi|.
  \end{equation*}
  If we consider the first term on the right-hand side and combine
  with the first exponential in the Mehler kernel we see that for $k
  \geq 1$:
  \begin{align*}
    \exp\biggl(-\e^{-t^2} \dfrac{|y - \xi|^2}{1 - \e^{-2t^2}}  \biggr)
    \exp\biggl(-\e^{-t^2} \dfrac{|y - \xi|^2}{1 + \e^{-t^2}}  \biggr)
    &\leq \exp\biggl(-\e^{-2t^2} \dfrac{|y - \xi|^2}{1 - \e^{-2t^2}}
    \biggr)\\
    &\leq \exp\biggl(-\e^{-2t^2} \dfrac{4^{k + 1} a^2 t^2}{1 - \e^{-2t^2}}
    \biggr).
  \end{align*}
  This inequality together with
  \begin{equation*}
    \frac{\e^{-2t^2}}{1 - \e^{-2t^2}} t^2 \geq \frac12,
  \end{equation*}
  yields,
    \begin{equation*}
    \exp\biggl(-\e^{-t^2} \dfrac{|y - \xi|^2}{1 - \e^{-2t^2}}  \biggr)
    \exp\biggl(-\e^{-t^2} \dfrac{|y - \xi|^2}{1 + \e^{-t^2}}  \biggr)
    \leq \exp\biggl(-\dfrac{a^2}{2} 4^{k + 1} \biggr).
  \end{equation*}
  Thus, we can estimate the Mehler kernel $M_{t^2}$ on $C_k$ for $k
  \geq 1$ from above by:
  \begin{equation*}
    M_{t^2}(y, \xi) \leq \frac{\e^{|y|^2}}{(1 - \e^{-t^2})^{\frac{d}2}}
    \exp\biggl(-\frac{a^2}{2} 4^{k + 1} \biggr) \exp\bigl(2^{k + 1} a t |y|
    \bigr).
  \end{equation*}
  We are left with the case $k = 0$, which can be done similarly and
  yields:
  \begin{equation*}
    M_{t^2}(y, \xi) \leq \frac{\e^{|y|^2}}{(1 - \e^{-t^2})^{\frac{d}2}}
    \exp\bigl(2^{k + 1} a t |y| \bigr).
  \end{equation*}
  Done.
\end{proof}


\printbibliography

\end{document}
