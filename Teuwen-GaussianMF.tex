% \RequirePackage[l2tabu, orthodox]{nag}
\documentclass{amsart}

\usepackage{amsmath, amssymb, color, verbatim}
\usepackage[active]{srcltx}
\usepackage{amsthm}

% These ones are needed to properly parse unicode like ö.
\usepackage[utf8]{inputenc}
\usepackage[T1]{fontenc}
% \usepackage[strict=true]{csquotes} % Needs to be loaded *after* inputenc

\usepackage{enumerate} 
% \usepackage{booktabs}% Better tables


% \makeatletter

% \@namedef{subjclassname@2010}{%

% \textup{2010} Mathematics Subject Classification}

% \makeatother

\swapnumbers
\newtheorem{theorem}{Theorem}
\newtheorem{definition}{Definition}
\newtheorem{lemma}{Lemma}
\newtheorem{corollary}{Corollary}
\newtheorem{proposition}{Proposition}
\theoremstyle{remark}
\renewcommand{\qedsymbol}{\ensuremath{\blacksquare}}
\newtheorem*{remark}{Remark}
\newtheorem*{examples}{Examples}


\newcommand{\D}{\,\textup{d}}
\newcommand{\Dn}{\textup{d}} % One without space.
\newcommand{\Dt}{\,\frac{\textup{d} t}{t}}
\newcommand{\Ds}{\,\frac{\textup{d} s}{s}}
\newcommand{\DyDt}{\frac{\textup{d} y \, \textup{d} t}{t^{n+1}}}
\newcommand{\la}{\langle}
\newcommand{\ra}{\rangle}
\newcommand{\LHG}{{L^2(\R^d,\gamma)}}

%% Bar int
% \def\Xint#1{\mathchoice
% {\XXint\displaystyle\textstyle{#1}}%
% {\XXint\textstyle\scriptstyle{#1}}%
% {\XXint\scriptstyle\scriptscriptstyle{#1}}%
% {\XXint\scriptscriptstyle\scriptscriptstyle{#1}}%
% \!\int}
% \def\XXint#1#2#3{{\setbox0=\hbox{$#1{#2#3}{\int}$ }
% \vcenter{\hbox{$#2#3$ }}\kern-.535\wd0}}
% \def\ddashint{\Xint=}
% \def\dashint{\Xint-}

\def\LI{{L^1_\gamma}}


%% Symbols
\renewcommand{\leq}{\leqslant}
\renewcommand{\geq}{\geqslant}
\renewcommand{\epsilon}{\varepsilon}
\newcommand{\Fo}{\mathcal{F}}
\newcommand{\R}{\mathbf R}
\newcommand{\e}{\mathrm{e}} %Roman e for exponentials

\DeclareMathOperator{\supp}{supp}
\newcommand{\Dg}{\frac{\textup{d}\gamma (y)}{\gamma (B(y,t))}}
\newcommand{\Dmu}{\frac{\textup{d}\mu (y)}{\mu (B(y,t))}}


% \usepackage{tikz}

\def\lemmaeqrefname{Lemma}
\def\definitioneqrefname{Definition}
\def\theoremeqrefname{Theorem}
\def\corollaryeqrefname{Corollary}



\begin{document}
\title[Gaussian maximal functions]{A note on the Gaussian maximal
  function}



\author{Jonas Teuwen}%
\address{Delft Institute of Applied Mathematics,
  Delft University of Technology, P.O. Box 5031, 2600 GA Delft, The
  Netherlands}%
\email{j.j.b.teuwen@tudelft.nl}%
\urladdr{http://fa.its.tudelft.nl/~teuwen/}%
\thanks{}%
\date{\today}

\maketitle

\begin{abstract}
  This note presents a proof that the non-tangential maximal function of the
  Ornstein-Uhlenbeck semigroup is bounded almost surely by the Gaussian
  Hardy-Littlewood maximal function.  In particular this entails improvement on
  a result by Pineda and Urbina \cite{Pineda2008} who proved a similar result
  for a `trunctated' version of the non-tangential maximal function. We
  actually obtain boundedness of the maximal function on non-tangential cones
  of arbitrary aperture.
\end{abstract}

% \subjclass[2010]{42B25 (Primary); 46E40 (Secondary)}
% \keywords{R-bounds, dyadic cubes}

\maketitle
\section{Introduction}
Maximal functions are among the most studied objects in harmonic
analysis. 
It is well-known that the classical non-tangential maximal function associated
with the heat semigroup is bounded almost everywhere by the Hardy-Littlewood
maximal function,
\begin{equation}\label{eq:classical}
  \sup_{\substack{(y, t) \in   \R^{d + 1}_+\\ |x - y| < t}} |\e^{-t \Delta} u(y)| \lesssim \sup_{r
    > 0}  \frac1{|B_r(x)|}\int_{B_r(x)} |u| \D\lambda.
\end{equation}
Here the action of \emph{heat semigroup} $\e^{-t \Delta} u = \rho_t \ast u$ is
given by a convolution of $u$ with the \emph{heat kernel}
\begin{equation*}
  \rho_t(\xi) := \frac{\e^{-|\xi|^2/4t}}{(4\pi t)^{\frac{d}2}}, \:\text{with}\:
  t > 0 \:\text{and}\: \xi \in \R^d.
\end{equation*}
In this note we are interested in its Gaussian counterpart. The change from
Lebesgue measure to the \textit{Gaussian measure}
\begin{equation}
  \label{eq:Gaussian-measure}
  \mathrm{d}\gamma(x) := \pi^{-\frac{d}2} \e^{-|x|^2} \D{x}
\end{equation}
introduces quite some intricate technical and conceptual difficulties which are
due to its non-doubling nature. Instead of the Laplacian, will use its Gaussian
analogue, the \emph{Ornstein-Uhlenbeck operator} $L$ which is given by,
\begin{equation}
  \label{eq:Ornstein-Uhlenbeck-operator}
  L := -\frac12 \Delta + \la x, \nabla \ra = \frac12 \nabla_\gamma^* \nabla_\gamma,
\end{equation}
where $\nabla_\gamma$ denotes the realisation of the gradient in $\LHG$.
Our main result, to be proved in Theorem~\ref{thm:Gaussian-maximal-function},
is the following Gaussian analogue of \eqref{eq:classical}:
\begin{equation}
  \label{eq:main}
  \sup_{(y, t) \in \Gamma_x^{(A, a)}} |\e^{-t^2 L} u(y)| \lesssim \sup_{r > 0}
  \frac1{\gamma(B_r(x))}\int_{B_r(x)} |u| \, \D\gamma.
\end{equation}
Here, 
\begin{equation}
  \label{eq:Gaussian-cone}
  \Gamma_x^{(A, a)} := \Gamma_x^{(A, a)}(\gamma) := \{(y, t) \in \R^{d + 1}_+
  \,: \, |x - y| < At \:\text{and}\: t \leq a m(x)\}
\end{equation}
is the \textit{Gaussian cone} with aperture $A$ and cut-off parameter $a$, and
\begin{equation}\label{eq:m-function}
  m(x) := \min\biggl\{1, \frac1{|x|} \biggr\}.
\end{equation}

A slighly weaker version of the inequality \eqref{eq:main} has been proved by 
Pineda and Urbina \cite{Pineda2008} who showed that 
\begin{equation*}
  \sup_{(y, t) \in \widetilde{\Gamma}_x} |\e^{-t^2 L} u(y)|
  \lesssim \sup_{r > 0}  \frac1{\gamma(B_r(x))}\int_{B_r(x)} |u| \D\gamma,
\end{equation*}
where
\begin{equation*}
  \widetilde{\Gamma}_x(x) = \{(y, t) \in \R^d_+ : |x - y| < t \leq
  \widetilde{m}(x)\}
\end{equation*}
is the `reduced' Gaussian cone corresponding to the function
\begin{equation*}
  \widetilde{m}(x) = \min\biggl\{\frac12, \frac1{|x|}\biggr\}.
\end{equation*}
Our proof of \eqref{eq:main} is much shorter and, we believe, more transparent
than the one presented in \cite{Pineda2008}. It has the further advantage of
allowing the extension to cones with arbitrary aperture $A > 0$ and cut-off
parameter $a > 0$ without any additional technicalities. This additional
generality is very important and has already been used by Portal (cf. the claim
made in \cite[discussion preceding Lemma 2.3]{Portal2012}) to prove the
$H^1$-boundedness of the Riesz transform associated with $L$.

\section{The Mehler kernel}
The \textit{Mehler kernel} (see e.g., \cite{Sjogren1997}) is the Schwartz
kernel associated to the Ornstein-Uhlenbeck semigroup $(\e^{-tL})_{t \geq 0}$,
that is,
\begin{equation}
  \label{eq:Ornstein-Uhlenbeck-semigroup-integral}
  \e^{-tL} u(x) = \int_{\R^d} M_t(x, \cdot) u \, \D\gamma.
\end{equation}
There is an abundance of literature on the Mehler kernel and its
properties. We shall only use the fact, proved e.g. in the review paper
\cite{Sjogren1997}, that it is given explicitly by
\begin{equation}
  \label{eq:Mehler-kernel-Sjogren}
  M_t(x,y) = \frac{\exp\biggl(-\dfrac{|\e^{-t} x - y|^2}{1 - \e^{-2t}}
    \biggr)}{(1 - \e^{-2t})^{\frac{d}2}} \e^{|y|^2}.
\end{equation}
Note that $M_t(x,y)$ is symmetric in $x$ and $y$. A formula for
\eqref{eq:Mehler-kernel-Sjogren} honoring this observation is:
\begin{equation}
  \label{eq:Mehler-kernel}
  M_t(x, y) = \frac{\exp\biggl(-\e^{-2t} \dfrac{|x - y|^2}{1
      - \e^{-2 t}}  \biggr)}{(1 - \e^{-t})^{\frac{d}2}}
  \frac{\exp\biggl(2\e^{-t} \dfrac{\la x, y \ra}{1 + \e^{-t}}
    \biggr)}{(1 + \e^{-t})^{\frac{d}2}}.
\end{equation}

\section{Some lemmata}
We use $m$ as defined in \eqref{eq:m-function} in our next lemma,
which is taken from \cite{MaasNeervenPortal2011}.
\begin{lemma}\label{lem:m-xy-equivalence}
  Let $a, A$ be strictly positive real numbers and $t > 0$. We have
  for $x, y \in \R^d$ that:
  \begin{enumerate}
  \item If $|x - y| < A t$ and $t \leq a m(x)$, then $t
    \leq (1 + aA) m(y)$;
  \item If $|x - y| < A m(x)$, then $m(x) \leq (1 +
    A) m(y)$ and $m(y) \leq 2 (1 + A) m(x)$. 
  \end{enumerate}
\end{lemma}

The next lemma will come useful when we want to cancel exponential
growth in one variable with exponential decay in the other as long
both variables are in a Gaussian cone.
\begin{lemma}\label{lem:Cone-Gaussians-comparable}
  Let $\alpha > 0$ and $|x - y| \leq \alpha m(x)$. Then:
  \begin{equation*}
    \e^{-\alpha^2-2\alpha} \e^{|y|^2}
    \leq \e^{|x|^2} \leq
    \e^{\alpha^2(1 + \alpha)^2+2\alpha(1 + \alpha)} \e^{|y|^2} .
  \end{equation*}
\end{lemma}
\begin{proof}
  By the inverse triangle inequality and $m(x)|x| \leq 1$ we get, 
  \begin{equation*}
    |y|^2 \leq (\alpha m(x) + |x|)^2 \leq \alpha^2 + 2 \alpha + |x|^2.
  \end{equation*}
  This gives the first inequality.  For the second we use
  Lemma~\ref{lem:m-xy-equivalence} to infer $m(x) \leq (1 + \alpha)
  m(y)$. Proceeding as before we obtain: 
  \begin{equation*}
    |x|^2 \leq \alpha^2 (1 + \alpha)^2 + 2 \alpha (1 + \alpha) + |y|^2.
  \end{equation*}
  As required.
\end{proof}

\subsection{An estimate on Gaussian balls}
Let $B := B_t(x)$ be the open Euclidean ball with radius $t$ and center $x$
and let $\gamma$ be the Gaussian measure as defined by
\eqref{eq:Gaussian-measure}. We shall denote by $S_d$ the surface area 
of the unit sphere in $\R^d$.

\begin{lemma}\label{lem:Gaussian-ball-shift-lemma}
  For all $x \in \R^d$ and $t > 0$ we have the inequality:
  \begin{equation}\label{eq:Gaussian-ball-shift-lemma}
    \gamma(B_t(x)) \leq  \frac{S_d}{2 \pi^{\frac{d}{2}}} t^d \e^{2t|x|} \e^{-|x|^2}. 
  \end{equation}
\end{lemma}
\begin{proof}
  Remark that, with $B := B_t(x)$,
  \begin{align*}
    \int_B \e^{-|\xi|^2} \D\xi &= \e^{-|x|^2} \int_{B} \e^{-|\xi -
      x|^2} \e^{-2 \la x, \xi - x \ra} \D\xi\\
    &\leq \e^{-|x|^2} \int_{B} \e^{-|\xi - x|^2} \e^{2 |x| |\xi - x|}
    \D\xi\\
    &\leq \e^{-|x|^2} \e^{2 t|x|} \int_{B} \e^{-|\xi - x|^2} \D\xi\\
    &= \pi^{\frac{d}2} \e^{2 t|x|} \e^{-|x|^2} \gamma(B_t(0)).
  \end{align*}
  So, there holds that
  \begin{equation}\label{eq:Gaussian-ball-shift-lemma-proof-1}
    \gamma(B_t(x)) \leq \e^{2 t|x|} \e^{-|x|^2} \gamma(B_t(0)).
  \end{equation}
  We will estimate the Gaussian volume of the ball $B_t(0)$. Using polar coordinates
  we proceed for $d \geq 2$ by: 
  \begin{align*}
    \gamma(B_t(0)) &= \frac1{\pi^{\frac{d}2}} \int_{B_t(0)} \e^{-|\xi|^2} \D\xi\\
    &= \frac{S_d}{\pi^{\frac{d}2}} \int_0^t \e^{-r^2} r^{d - 1} \D{r}\\
    &\leq \frac{S_d}{2 \pi^{\frac{d}2}} t^{d-2} \int_0^t 2r \e^{-r^2} \D{r}\\
    &= \frac{S_d}{2 \pi^{\frac{d}2}} t^{d-2} (1-\e^{-t^2})\\
    &\leq \frac{S_d}{2 \pi^{\frac{d}2}} t^d,
  \end{align*}
  where the last step uses $1 - \e^{-t} \leq t$ for $t \geq 0$. The case
  for $d = 1$ follows by a simplified argument. Upon combining this
  result with \eqref{eq:Gaussian-ball-shift-lemma-proof-1} we obtain
  \eqref{eq:Gaussian-ball-shift-lemma}.
\end{proof}

\subsection{On-diagonal kernel estimates on annuli}
As is common in harmonic analysis, we often wish to decompose
$\R^d$ into sets on which certain phenomena are easier to handle. Here
we will decompose the space into disjoint annuli. 

Throughout this subsection we fix $x \in \R^d$, constants $A, a \geq 1$, a pair
$(y,t) \in \Gamma_x^{(A, a)}$. We use the notation $rB$ to mean the ball obtained
from the ball $B$ by multiplying its radius by $r$.

The annuli $C_k := C_k(B_t(x))$ are given by:
\begin{equation}
  \label{eq:C_k-annulus-decomposition}
  C_k : = 2^{k + 1} B_t(x) \setminus 2^k B_t(x)
  \:\text{with}\: k \geq 0.
\end{equation}
Note that we need to add $B_t$ itself as the first `annulus'. Whenever $\xi$ is in $C_k$, we get for $k
\geq 0$:
\begin{equation}
  \label{eq:C_k-annulus-decomposition-expand}
  2^k t < |y - \xi| \leq 2^{k + 1} t.
\end{equation}
On $C_k$ we have the following bound for $M_{t^2}(y,\cdot)$:
\begin{lemma}\label{lem:On-diagonal-kernel-estimates-on-Ck}
  For all $\xi \in C_k$ we have:
  \begin{equation}
    M_{t^2}(y, \xi) \leq \frac{\e^{|y|^2}}{(1 - \e^{-2t^2})^{\frac{d}2}}
    \exp\bigl(2^{k +  1} t |y| \bigr) \exp\Big(-\frac{4^k}{2 \e^{2 t^2}} \Bigr),
  \end{equation}
\end{lemma}
\begin{proof}
  Considering the first exponential which occurs in the Mehler kernel
  \eqref{eq:Mehler-kernel} together with
  \eqref{eq:C_k-annulus-decomposition-expand} gives for $k \geq 0$:
  \begin{align*}
    \exp\biggl(-\e^{-2t^2} \frac{|y - \xi|^2}{1 - \e^{-2t^2}} \biggr)
    &\overset{\phantom{(\dagger)}}{\leq} \exp\biggl(-\frac{4^k}{\e^{2t^2}}
    \frac{t^2}{1 - \e^{-2t^2}} \biggr)\\
    &\overset{(\dagger)}{\leq} \exp\biggl(-\frac{4^k}{2 \e^{2t^2}} \biggr),
  \end{align*}
  where $(\dagger)$ follows from $1 - \e^{-s} \leq s$ for $s \geq 0$. Using the
  estimate $1 + s \geq 2s$ for $0 \leq s \leq 1$, for the second exponential in
  the Mehler kernel \eqref{eq:Mehler-kernel} we obtain, by
  \eqref{eq:C_k-annulus-decomposition-expand}:
  \begin{align*}
    \exp\biggl(2\e^{-t^2} \frac{\la y, \xi \ra}{1 + \e^{- t^2}} \biggr)
    & \leq \exp(|\la y, \xi \ra|)\\
    & \leq \exp(|\langle y, \xi-y\rangle|) \e^{|y|^2}\\
    & \leq \exp\bigl(2^{k + 1} t |y| \bigr) \e^{|y|^2}.
  \end{align*}
  Combining things, we obtain the estimate in the formulation of the lemma.
\end{proof}

\section{The main result}
In this section we will prove our main theorem for which we have already made
the necessary preparations in the previous sections.
\begin{theorem}\label{thm:Gaussian-maximal-function}
  Let $A, a > 0$. For all $x \in \R^d$ and all $u \in \LHG$ we have
  \begin{equation}
    \label{eq:Maximal-function-cone}
    \sup_{(y, t) \in \Gamma_x^{(A, a)}} |\e^{-t^2 L} u(y)| \lesssim
    \sup_{r > 0} \frac1{\gamma(B_r(x))}\int_{B_r(x)} |u| \, \D\gamma.
  \end{equation}
\end{theorem}
\begin{proof}
  We fix $x \in \R^d$ and $ (y, t) \in \Gamma_x^{(A, a)}$. The proof of
  \eqref{eq:Maximal-function-cone} is based on splitting the
  integration domain into the annuli $C_k$ as defined by
  \eqref{eq:C_k-annulus-decomposition} and estimating on each annulus. More
  explicit,
  \begin{equation}
    \label{eq:Maximal-function-cone-intermediate-step-1}
    |\e^{-t^2 L} u(y)| \leq \sum_{k = 0}^\infty I_k(y),
    \:\text{where}\: I_k(y) := \int_{C_k} M_{t^2}(y, \cdot) |u(\cdot)|
    \,\D\gamma.
  \end{equation} 
  From $t \leq a m(x)$ we get $t |x| \leq a$, and by
  Lemma~\ref{lem:m-xy-equivalence} this implies $t |y| \leq 1 + aA$. Together
  with Lemma~\ref{lem:On-diagonal-kernel-estimates-on-Ck} we infer, for $\xi
  \in C_k$, that:
  \begin{align*}
    \label{eq:Mehler-kernel-estimate-one-sided-bound-1}
    M_{t^2}(y, \xi) &\leq \frac{\e^{|y|^2}}{(1 - \e^{-2t^2})^{\frac{d}2}}
    \exp(2^{k + 1} (1 + aA)) \exp\Big(-\frac{4^k}{2 \e^{2 a^2}} \Bigr)\\
    &=: \frac{\e^{|y|^2}}{(1 - \e^{-2t^2})^{\frac{d}2}} c_k.
  \end{align*}
  Combining this with Lemma~\ref{lem:Cone-Gaussians-comparable}, we obtain
  \begin{equation}
    \label{eq:Mehler-kernel-estimate-one-sided-bound-1}
    M_{t^2}(y, \xi) \lesssim_{A, a} \frac{\e^{|x|^2}}{(1 - \e^{-2t^2})^{\frac{d}2}} c_k.
  \end{equation}       
  Also, by \eqref{eq:C_k-annulus-decomposition-expand},
  \begin{equation*}
    |x - \xi| \leq |x - y| + |\xi - y| \leq (2^{k + 1} + 1) t .
  \end{equation*}
  It follows that $C_k$ is contained in $D_k := B_{(2^{k + 1} + 1) t}(x)$.

  Let us denote the supremum on right-hand side of
  \eqref{eq:Maximal-function-cone} by $M_\gamma u (x)$. Using
  \eqref{eq:Mehler-kernel-estimate-one-sided-bound-1}, we can bound the
  integral on the right-hand side of
  \eqref{eq:Maximal-function-cone-intermediate-step-1} by
  \begin{align*}
    \int_{C_k}  M_{t^2}(y, \cdot) |u(\cdot)| \,\D\gamma & \lesssim_{A, a}
    c_k \frac{\e^{|x|^2}}{(1 - \e^{-2t^2})^{\frac{d}2}}   \int_{C_k}
    |u| \,\D\gamma\\ 
    &\leq c_k \frac {\e^{|x|^2}} {(1 -
      \e^{-2t^2})^{\frac{d}2}} \int_{D_k} |u| \,\D\gamma\\ 
    &\leq c_k  \frac {\e^{|x|^2}}{(1 - \e^{-2t^2})^{\frac{d}2}} \gamma(D_k) M_\gamma u(x)\\
    &\overset{(\dagger)}{\leq} c_k S_d (2^{k + 1} + 1)^d \biggl(\frac{t^2}{1 - \e^{-2t^2}} \biggr)^{\frac{d}2} \e^{(2^{k + 1} + 1) t|x|} M_\gamma u(x)\\
    &\overset{(\ddagger)}{\leq} c_k C_a^d S_d (2^{k + 1} + 1)^d  \e^{(2^{k
        + 1} + 1) a} M_\gamma u(x),
  \end{align*}
  where ($\dagger$) uses Lemma~\ref{lem:Gaussian-ball-shift-lemma} applied to
  $D_k$ and ($\ddagger$) uses that $t \leq a m(x)$ implies that $t |x| \leq a$ and
  $t\leq a$, where the latter implying 
 \begin{equation*}
    \biggl(\frac{t^2}{1 - \e^{-2t^2}} \biggr)^{\frac{d}2} \leq
    \biggl(\frac{a^2}{1 - \e^{-2 a^2}} \biggr)^{\frac{d}2} =: C_a^d.
  \end{equation*}
  (note that $s/(1-\e^{-s})$ is increasing). Similarly, for $\xi \in B_t(x)$ we
 obtain:
   \begin{align*}
    \int_{B_t}  M_{t^2}(y, \cdot) |u(\cdot)| \,\D\gamma & \lesssim_{A, a}
    \frac{\e^{|x|^2}}{(1 - \e^{-2t^2})^{\frac{d}2}} \int_{B_t} |u| \,\D\gamma\\
    &\leq \frac {\e^{|x|^2}}{(1 - \e^{-2t^2})^{\frac{d}2}} \gamma(B_t) M_\gamma
    u(x)\\
    &\lesssim_a S_d a^d M_\gamma u(x).
  \end{align*}
  Inserting the depenceny of $c_k$ upon $k$ as coming from 
  \eqref{eq:Mehler-kernel-estimate-one-sided-bound-1} and using that $t\leq 1$,
  we obtain the bound: 
  \begin{align*}
    |\e^{-t^2 L} u(y)| &= \sum_{k = 0}^\infty I_k\\
    &\lesssim C_a^d S_d  M_\gamma u(x) \sum_{k = 0}^\infty c_k (2^{k + 1} +
    1)^d  \e^{(2^{k + 1} + 1) a}\\
    &=_a C_a^d S_d  M_\gamma u(x) \sum_{k
      = 0}^\infty  c_k (2^{k + 1} + 1)^d  \exp(2^{k + 1} (1 + a(1 + A))) \exp\Big(-\frac{4^k}{2 \e^{2 a^2}} \Bigr)\\
  \end{align*}
  Set $\alpha = \frac1{2\e^2}$
  \begin{align*}
    |\e^{-t^2 L} u(y)| &\lesssim \frac{C^d}{\Gamma\bigl(\frac{d}2 \bigr)} M_\gamma u(x) \sum_{k
      = 0}^\infty 2^{kd} \e^{3 \cdot 2^{k + 1}} \e^{-\alpha (2^k - 1)^2}\\
  \end{align*}


  valid for all $y \in \Gamma_x^{(A, a)}$. As the sum on the right-hand side
  evidently converges, we see that taking the supremum proves
  \eqref{eq:Maximal-function-cone}.
\end{proof}
\bibliographystyle{amsplain}

\bibliography{Teuwen-GaussianMF}
\end{document}
