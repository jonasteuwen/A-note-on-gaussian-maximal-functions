\RequirePackage[l2tabu, orthodox]{nag}
\documentclass[a4paper,oneside,10pt]{amsproc}

\usepackage{fixltx2e}
\usepackage[all, error]{onlyamsmath}
\usepackage{fixmath} % http://ctan.org/pkg/fixmath
\usepackage{refcheck}
\norefnames
% \showrefnames
\usepackage{microtype}
\usepackage{amsmath}
\usepackage{amsthm}
\usepackage[x11names]{xcolor}%
\usepackage{textcomp}
\usepackage[english]{babel}
\usepackage{xfrac} % Nice / fractions
\usepackage[utf8]{inputenc}
\usepackage[T1]{fontenc}
\usepackage[strict=true]{csquotes} % Needs to be loaded *after* inputenc

\usepackage{enumerate} 
\usepackage{booktabs}% Better tables

\usepackage{latexsym}
\usepackage{bbm}
\usepackage{enumitem}
\usepackage[bitstream-charter]{mathdesign}%

\usepackage[bookmarks,colorlinks,breaklinks]{hyperref} % Add hyperref type links in the document, colors
\definecolor{dullmagenta}{rgb}{0.4,0,0.4} % #660066
\definecolor{darkblue}{rgb}{0,0,0.4}%
\hypersetup{linkcolor=red,citecolor=blue,filecolor=dullmagenta,urlcolor=darkblue} % coloured links


\makeatletter

\@namedef{subjclassname@2010}{%

  \textup{2010} Mathematics Subject Classification}

\makeatother

\swapnumbers
\newtheorem{theorem}{Theorem}
\newtheorem{definition}{Definition}
\newtheorem{lemma}{Lemma}
\newtheorem{corollary}{Corollary}
\newtheorem{proposition}{Proposition}
\theoremstyle{remark}
\renewcommand{\qedsymbol}{\ensuremath{\blacksquare}}
\newtheorem*{remark}{Remark}
\newtheorem*{examples}{Examples}


\newcommand{\E}{\mathbb{E}}
\newcommand{\Sp}{\mathbb{S}}
\newcommand{\prob}{\mathbb{P}}
\newcommand{\D}{\,\textup{d}}
\newcommand{\Dn}{\textup{d}} % One without space.
\newcommand{\Dt}{\,\frac{\textup{d} t}{t}}
\newcommand{\Ds}{\,\frac{\textup{d} s}{s}}
\newcommand{\DyDt}{\frac{\textup{d} y \, \textup{d} t}{t^{n+1}}}
\newcommand{\Dd}{\mathscr{D}}
\newcommand{\Tt}{\mathscr{T}}
\newcommand{\Cc}{\mathscr{C}}
\newcommand{\Bb}{\mathscr{B}}
\newcommand{\Rr}{\mathscr{R}}
\newcommand{\Ff}{\mathscr{F}}
\newcommand{\Ll}{\mathscr{L}}
\newcommand{\Mm}{\mathscr{M}}
\newcommand{\hh}{\mathfrak{h}}
\newcommand{\ttt}{\mathfrak{t}}
\newcommand{\la}{\langle}
\newcommand{\ra}{\rangle}
\newcommand{\Rad}{\textup{Rad}}
\newcommand{\Car}{\textup{Car}}
\newcommand{\BMO}{\textup{BMO}}
\newcommand{\loc}{\textup{loc}}
\newcommand{\LH}{{L^2_\mu}}
\newcommand{\LHG}{{L^2_\gamma}}

%% Bar int
\def\Xint#1{\mathchoice
  {\XXint\displaystyle\textstyle{#1}}%
  {\XXint\textstyle\scriptstyle{#1}}%
  {\XXint\scriptstyle\scriptscriptstyle{#1}}%
  {\XXint\scriptscriptstyle\scriptscriptstyle{#1}}%
  \!\int}
\def\XXint#1#2#3{{\setbox0=\hbox{$#1{#2#3}{\int}$ }
    \vcenter{\hbox{$#2#3$ }}\kern-.535\wd0}}
\def\ddashint{\Xint=}
\def\dashint{\Xint-}

\def\LI{{L^1_\gamma}}


%% Symbols
\renewcommand{\bar}[1]{\overline#1}
\renewcommand{\vec}[1]{\boldsymbol{\mathbf{#1}}}
\renewcommand{\leq}{\leqslant}
\renewcommand{\Im}{\operatorname{Im}}
\renewcommand{\Re}{\operatorname{Re}}
\renewcommand{\leq}{\leqslant}
\renewcommand{\geq}{\geqslant}
\renewcommand{\epsilon}{\varepsilon}
\renewcommand{\emptyset}{\varnothing}
\newcommand{\Fo}{\mathcal{F}}
\newcommand{\R}{\mathbf R}
\newcommand{\C}{\mathbf C}
\newcommand{\N}{\mathbf N}
\newcommand{\T}{\mathbb T}
\newcommand{\Z}{\mathbf Z}
\newcommand{\B}{\mathcal B}
\newcommand{\e}{\mathrm{e}} %Roman e for exponentials

\renewcommand{\leq}{\leqslant}%
\renewcommand{\geq}{\geqslant}%
\DeclareMathOperator{\supp}{supp}
\newcommand{\Dg}{\frac{\textup{d}\gamma (y)}{\gamma (B(y,t))}}
\newcommand{\Dmu}{\frac{\textup{d}\mu (y)}{\mu (B(y,t))}}


\renewcommand{\Re}{\operatorname{Re}}
\renewcommand{\Im}{\operatorname{Im}}
\renewcommand{\bar}{\overline}

\usepackage{tikz}

\def\lemmaautorefname{Lemma}
\def\definitionautorefname{Definition}
\def\theoremautorefname{Theorem}
\def\corollaryautorefname{Corollary}


%% Bibliography
\usepackage[backend=biber,doi=false,url=false,isbn=false]{biblatex}
\bibliography{~/Documents/BibTeX/library.bib}

\newbibmacro{string+doiurlisbn}[1]{%
  \iffieldundef{doi}{%
    \iffieldundef{url}{%
      \iffieldundef{isbn}{%
        \iffieldundef{issn}{%
          #1%
        }{%
          \href{http://books.google.com/books?vid=ISSN\thefield{issn}}{#1}%
        }%
      }{%
        \href{http://books.google.com/books?vid=ISBN\thefield{isbn}}{#1}%
      }%
    }{%
      \href{\thefield{url}}{#1}%
    }%
  }{%
    \href{http://dx.doi.org/\thefield{doi}}{#1}%
  }%
}

\DeclareFieldFormat{title}{\usebibmacro{string+doiurlisbn}{\mkbibemph{#1}}}
\DeclareFieldFormat[article,incollection]{title}%
{\usebibmacro{string+doiurlisbn}{\mkbibquote{#1}}}


\title[Gaussian maximal functions]{A note on the Gaussian maximal
  function - Version 9 October 2013 + JvN additions}



\author{Jonas Teuwen}%
\address{Delft Institute of Applied Mathematics,
  Delft University of Technology, P.O. Box 5031, 2600 GA Delft, The
  Netherlands}%
\email{j.j.b.teuwen@tudelft.nl}%
\urladdr{http://fa.its.tudelft.nl/~teuwen/}%
\thanks{}%
\date{\today}

\begin{document}
\begin{abstract}
  This note presents a proof that 
  the non-tangential maximal function of the Ornstein-Uhlenbeck semigroup
  is bounded almost surely by the Gaussian Hardy-Littlewood maximal
  function.  In particular this entails improvement on a result by
  Pineda and Urbina \cite{Pineda2008} who proved a similar result for 
  a `trunctated' version of the non-tangential maximal function. 
  We actually obtain boundedness of the maximal function on non-tangential
  cones of arbitrary aperture.
  % 
  % 
  % This note presents a proof that the Gaussian Hardy-Littlewood maximal
  % function bounds the general non-tangential gaussian heat semigroup -the
  % so-called Ornstein-Uhlenbeck semigroup.
  % 
  % In particular we give an improvement on a result by
  % \textcite{Pineda2008} which gives the boundedness of the Gaussian
  % maximal function associated to the Ornstein-Uhlenbeck operator.
  % 
  % We present a proof which is at least to the author more transparant.
  % Our main finding in this note is that our proof allows to use a
  % larger cone and actually obtain the maximal function boundedness for
  % a whole class of cones $\Gamma^{(A, a)}_x(\gamma)$.
\end{abstract}

% \subjclass[2010]{42B25 (Primary); 46E40 (Secondary)}
% \keywords{R-bounds, dyadic cubes}

\maketitle
\section{Introduction}
Maximal functions are among the most studied objects in harmonic
analysis. 
It is well known that the classical real-valued maximal
function associated with the heat semigroup is bounded almost everywhere
by the Hardy-Littlewood maximal function, 
\begin{equation}\label{eq:classical}
  \sup_{(y, t) \in \Gamma_x} |\e^{-t \Delta} u(y)| \lesssim \sup_{r
    > 0}  \dashint_{B_r(x)} |u| \D\lambda.
\end{equation}
Here the action of \emph{heat semigroup} $\e^{-t^2 \Delta} u = \rho_t \ast u$ is
given by a convolution of $u$ with the \emph{heat kernel}
\begin{equation*}
  \rho_t(s) := \frac{\e^{-|s|^2/4t}}{\pi^{\frac{d}2}}
  \frac1{(4t)^{\frac{d}2}}.
\end{equation*}
% so that,
% \begin{equation*}
%   \e^{-t \Delta} u(x) = (\rho_t \ast u)(x).
% \end{equation*}
In this note we are interested in its gaussian counterpart. 
% Gaussian harmonic analysis seems to be conceptually
% nothing more than harmonic analysis with the gaussian measure, but
% this is far off from reality. 
The change from Lebesgue measure to the gaussian measure
\begin{equation}
  \label{eq:Gaussian-measure}
  \mathrm{d}\gamma(x) := \pi^{-\frac{d}2} \e^{-|x|^2} \D{x}
\end{equation}
introduces quite
some intricate technical and conceptually difficulties which appears
to be due to the fact that the Gaussian measure is non-doubling.

As an analogue to the Laplacian which is symmetric in $L^2$ with
respect to the Lebesgue measure next we introduce the
\emph{Ornstein-Uhlenbeck} operator $L$ which is symmetric with respect
to the Gaussian measure:
\begin{equation}
  \label{eq:Ornstein-Uhlenbeck-operator}
  L := -\frac12 \Delta + \la x, \nabla \ra = \frac12 \nabla_\gamma^* \nabla_\gamma,
\end{equation}
where $\nabla_\gamma$ denotes the realisation of the gradient in $L^2(\R^d,\gamma)$.
Our main result, to be proved in  \eqref{thm:Gaussian-maximal-function}, is the 
following gaussian analogue of \eqref{eq:classical}:
% perhaps we should skip the t^2 too, changed for now.
\begin{equation}\label{eq:main}
  \sup_{(y, t) \in \Gamma_x^{(A, a)}} |\e^{-t L} u(y)| \lesssim
  \sup_{r > 0} \dashint_{B_r(x)} |u| \, \D\gamma.
\end{equation}
Here, $\Gamma_x^{(A, a)}$ is the Gaussian cone defined by
\begin{equation}
  \label{eq:Gaussian-cone}
  \Gamma_x^{(A, a)} := \Gamma_x^{(A, a)}(\gamma) := \{(y, t) \in
  \R^d_+ : |x - y| < At \:\text{and}\: t \leq a m(x)\},
\end{equation}
where
\begin{equation}\label{eq:m-function}
  m(x) := \min\biggl\{1, \frac1{|x|} \biggr\} = 1 \wedge \frac1{|x|}.
\end{equation}

A slighly weaker version of the inequality \eqref{eq:main} has been proved by 
Pineda and Urbina \cite{Pineda2008} who 
shows that 
\begin{equation*}
  \sup_{(y, t) \in \widetilde{\Gamma}_x} |\e^{-t \Delta} u(y)|
  \lesssim \sup_{r > 0}  \dashint_{B_r(x)} |u| \D\gamma,
\end{equation*}
where
\begin{equation*}
  \widetilde{\Gamma}_x(x) = \{(y, t) \in \R^d_+ : |x - y| < t \leq
  \widetilde{m}(x)\}
\end{equation*}
is the `reduced' gaussian cone corresponding to the function
\begin{equation*}
  \widetilde{m}(x) = \min\biggl\{\frac12, \frac1{|x|}\biggr\}.
\end{equation*}
Their proof does not seem to easily generalize the range of $t$ from $\frac12$ up
to $1$. The proof of \eqref{eq:main} is different and, we believe, more transparent 
than the one presented in \cite{Pineda2008}. It has the further virtue of allowing 
the extension to the cones
with arbitrary aperture $A>0$ and cut-off parameter $a>0$. This additional generality
is very important and has already been used by Portal (cf. 
the claim made by \cite[discussion preceding Lemma 2.3]{Portal2012}) to prove the 
$H^1$-boundedness of the Riesz transform associated with $L$.

Before we continue, let us fix some notation. We will use without
further reference notation such as $\Z^d$ while we implicitly imply
that $d$ is a positive integer. To avoid possible confusion, we define
the \emph{positive integers} as the set $\Z_+ = \{1, 2, 3, \dots\}$. 


% How do cone scalings behave wrt several maximal functions?
% Especially for the latter.
% 
% Need to add domain and measure in first section, but that can be in
% the introduction later on.
% 
% 

\subsubsection{$m$inimal function}
We recall the lemma from \cite[lemma 2.3]{Maas2011} which first
--although implicitly-- appeared in \cite{Mauceri2007}. \footnote{Nog materiaal toevoegen?}

\section{The Mehler kernel}
\subsection{Setting}
Recall that we work with the \emph{Ornstein-Uhlenbeck} operator $L$ as
given by \eqref{eq:Ornstein-Uhlenbeck-operator}.

We define the Mehler kernel (see e.g., \textcite{Sjogren1997}) as the
Schwartz kernel associated to the Ornstein-Uhlenbeck semigroup
$(\e^{-tL})_{t \geq 0}$. More precisely, this means:
\begin{equation}
  \label{eq:Ornstein-Uhlenbeck-semigroup-integral}
  \e^{-tL} u(x) = \int_{\R^d} M_t(x, \cdot) u \, \D\gamma.
\end{equation}
It is often more convenient to use $\e^{-t^2 L}$ instead of $\e^{-tL}$
as is done in e.g., \textcite{Portal2012}.

\subsection{The Mehler kernel}
There is an abundance of literature on the Mehler kernel an its
properties available, but for the present purpose
\textcite{Sjogren1997} will suffice. For instance, the Mehler kernel
$M_t$ of \eqref{eq:Ornstein-Uhlenbeck-semigroup-integral} is computed
there. In addition it offers related results with to the Hermite
polynomials.

The kernel $M_t$ is invariant under the permutation $x
\leftrightarrow y$. A formula for $M_t$ which honors this
observation is:
\begin{equation}
  \label{eq:Mehler-kernel}
  M_t(x, y) = \frac{\exp\biggl(-\e^{-2t} \dfrac{|x - y|^2}{1
      - \e^{-2 t}}  \biggr)}{(1 - \e^{-t})^{\frac{d}2}}
  \frac{\exp\biggl(2\e^{-t} \dfrac{\la x, y \ra}{1 + \e^{-t}}
    \biggr)}{(1 + \e^{-t})^{\frac{d}2}}.
\end{equation}

\section{Some lemmata and definitions}
We use $m$ as defined in \eqref{eq:m-function} in our next lemma.
\begin{lemma}\label{lem:m-xy-equivalence}
  Let $a, A$ be strictly positive real numbers and $t > 0$. We have
  for $x, y \in \R^d$ that:
  \begin{enumerate}
  \item If $|x - y| < A t$ and $t \leq a m(x)$, then $t
    \leq (1 + aA) m(y)$;
  \item If $|x - y| < A m(x)$, then $m(x) \leq (1 +
    A) m(y)$ and $m(y) \leq 2 (1 + A) m(x)$. 
  \end{enumerate}
\end{lemma}

The next lemma will come useful when we want to cancel exponential
growth in one variable with exponential decay in the other as long
both variables are in a Gaussian cone.
\begin{lemma}\label{lem:Cone-Gaussians-comparable}
  Let $\alpha > 0$ and $|x - y| \leq \alpha m(x)$. We get the
  equivalence:
  \begin{equation*}
    \e^{-\alpha^2(1 + \alpha)^2} \e^{-2\alpha(1 + \alpha)} \e^{-|y|^2}
    \leq \e^{-|x|^2} \leq \e^{\alpha^2} \e^{2\alpha} \e^{-|y|^2}.
  \end{equation*}
\end{lemma}
\begin{proof}
  By the inverse triangle inequality and $m(x)|x| \leq 1$ we get, 
  \begin{equation}
    |y|^2 \leq (\alpha m(x) + |x|)^2 \leq \alpha^2 + 2 \alpha + |x|^2.
  \end{equation}
  For the reverse direction we use
  \autoref{lem:m-xy-equivalence} to infer $m(x) \leq (1 + \alpha)
  m(y)$. Proceeding as before we obtain: 
  \begin{equation*}
    |x|^2 \leq \alpha^2 (1 + \alpha)^2 + 2 \alpha (1 + \alpha) + |y|^2.
  \end{equation*}
  Combining we get:
  \begin{equation}
    \label{eq:Cone-Gaussians-comparable}
    \e^{-\alpha^2(1 + \alpha)^2} \e^{-2\alpha(1 + \alpha)} \e^{-|y|^2}
    \leq \e^{-|x|^2} \leq \e^{\alpha^2} \e^{2\alpha} \e^{-|y|^2}.
  \end{equation}
  As required.
\end{proof}

\section{On-diagonal estimates}
\subsection{Kernel estimates}
Before we proceed with the technicalities we define $\kappa$ and
$\mu$ as:
\begin{equation*}
  \kappa = 2\Bigl(1 + \frac1\alpha \Bigr)^{-1}, \: \text{and} \: \mu
  = 2\Bigl(1 - \frac1\alpha \Bigr)^{-1}.
\end{equation*}
such that $\kappa$ and $\mu$ are conjugate exponents, which means:
\begin{equation*}
  \frac1{\kappa} + \frac1{\mu} = 1.
\end{equation*}
We proceed with a simple technical lemma which is given here as it
will be used on several occasions.
\begin{lemma}\label{lem:Time-part-Mehler-time-transform}
  Let $t > 0$ and $\alpha \geq 1$. Then,
  \begin{equation}
    \label{eq:Time-part-Mehler-time-transform-1}
    \alpha \e^{-2\frac{t}{\mu}} \leq \frac{1 -
      \e^{-t}}{1 - \e^{-\frac{t}{\alpha}}} \leq \alpha,
  \end{equation}
  \begin{equation}
    \label{eq:Time-part-Mehler-time-transform-2}
    0 \leq \frac1t \biggl[\frac{\e^{-t}}{1 + \e^{-t}} -
    \frac{\e^{-\frac{t}\alpha}}{1 + \e^{-\frac{t}{\alpha}}} \biggr]
    \leq \frac{1}{2 \mu}.
  \end{equation}
\end{lemma}
\begin{proof}
  We start with \eqref{eq:Time-part-Mehler-time-transform-1} and apply
  the mean value theorem to the function $f(\xi) = \xi^\alpha$. For $0
  < \xi < \xi'$ this gives that:
  \begin{equation*}
    f(\xi) - f(\xi') = \alpha \hat{\xi}^{\alpha - 1} (\xi - \xi')
    \text{ for some $\hat \xi$ in $[\xi, \xi']$}.
  \end{equation*}
  Picking $\xi = 1$ and $\xi' = \e^{-\frac{t}{\alpha}}$ yields:
  \begin{equation}
    \label{eq:Time-part-Mehler-time-transform-proof-1}
    \frac{1 - \e^{-t}}{1 - \e^{-\frac{t}{\alpha}}} = \alpha
    \hat{\xi}^{\alpha - 1} \:\text{for some}\: \hat{\xi} \:\text{in}\:
    \Bigl[\e^{-\frac{t^2}{\alpha}}, 1 \Bigr].
  \end{equation}
  Combining this result with the monotonicity of $\xi \mapsto
  \alpha \xi^{\alpha - 1}$ we obtain:
  \begin{equation*}
    \alpha \e^{\frac{t}{\alpha}} \e^{-t} \leq \frac{1 - \e^{-t}}{1 -
      \e^{-\frac{t}{\alpha}}} \leq \alpha,
  \end{equation*}
  where the last bound follows from the monotonicity together with the
  limit as $t \downarrow 0$.
  We proceed with \eqref{eq:Time-part-Mehler-time-transform-2}.
  Recalling that $\alpha \geq 1$ one can directly verify that the
  function
  \begin{equation*}
    \frac1t \biggl[\frac{1}{1 + \e^{-t}} - \frac{1}{1 +
      \e^{-\frac{t}{\alpha}}} \biggr]
  \end{equation*}
  is non-negative and decreasing in $t$. To find
  an upper bound we compute the limit as $t$ goes to $0$. That is:
  \begin{equation*}
    \lim_{t \to 0} \frac1t \biggl[\frac{\e^{-t}}{1 + \e^{-t}} -
    \frac{\e^{-\frac{t}\alpha}}{1 + \e^{-\frac{t}{\alpha}}} \biggr] 
    = \lim_{t \to 0} \biggl[\frac{\e^{-2t}}{(1 + \e^{-t})^2} -
    \frac1\alpha \frac{\e^{-2\frac{t}{\alpha}}}{(1 +
      \e^{-\frac{t}{\alpha}})^2} \biggr] \uparrow \frac{1}{2\mu}.
  \end{equation*}
  Which is as asserted and completes the proof.
\end{proof}

The following lemma will be useful when transfering estimates from
$M_{\frac{t}{\alpha}}$ to $M_t$. It follows from the mean value
theorem applied to $\xi \mapsto \xi^\alpha$.
\begin{lemma}\label{lem:Exponential-estimates}
  For $\alpha \geq 1$ and $0 < t \leq T < \infty$ and all let $x, y \in \R^d$
  we have that:
  \begin{equation}
    \label{eq:Exponential-estimates-1}
    \exp \biggl (-\frac1{\e^{2\frac{t}\alpha}} \frac{|x - y|^2}{1 - \e^{-2\frac{t}\alpha}}
    \biggr ) \leq \exp \biggl(-\frac{\alpha}{2\e^{2\frac{t}\kappa}} \frac{|x -
      y|^2}{1 - \e^{-t}} \biggr).
  \end{equation}
\end{lemma}
\begin{proof}
  It is fruitful to note that
  \begin{equation*}
    1 - \e^{-2t} = (1 - \e^{-t})(1 + \e^{-t})
  \end{equation*}
  holds. The same holds true for the inequality
  \begin{equation*}
    \frac1{2 \e^{2t}} \leq \frac{\e^{-2t}}{1 + \e^{-t}} \leq \frac12.
  \end{equation*}
  Therefore,
  \begin{align*}
    \exp\biggl(-\e^{-2t} \dfrac{|x - y|^2}{1 - \e^{-2 t}} \biggr)
    &= \exp\biggl(-\dfrac{\e^{-2t}}{1 + \e^{-t}} \dfrac{|x -
      y|^2}{1 - \e^{-t}} \biggr)\\
    &\leq \exp\biggl(-\frac1{2\e^{2t}} \dfrac{|x - y|^2}{1 - \e^{-t}} \biggr).
  \end{align*}
  Setting $\beta := 1 + \alpha^{-1}$ and applying \eqref{eq:Time-part-Mehler-time-transform-1} we get:
  \begin{align*}
    \exp \biggl (-\e^{-2\frac{t}\alpha} \frac{|x - y|^2}{1 - \e^{-2\frac{t}\alpha}}
    \biggr ) &= \exp \biggl (-\frac{\e^{-2\frac{t}\alpha}}{1 + \e^{-\frac{t}\alpha}} \frac{1 - \e^{-t}}{1 -
      \e^{-\frac{t}\alpha}}  \frac{|x - y|^2}{1 - \e^{-t}} \biggr )\\
    &\leq \exp \biggl(-\frac{\alpha}{2\e^{2\frac{t}{\alpha}}}
    \frac1{\e^t \e^{-\frac{t}\alpha}} \frac{|x - y|^2}{1 - \e^{-t}}
    \biggr)\\
    &\leq \exp \biggl(-\frac{\alpha}{2\e^{2\frac{t}\kappa}} \frac{|x -
      y|^2}{1 - \e^{-t}} \biggr).
  \end{align*}
  Which is as asserted.
\end{proof}
Our first lemma is about estimating $M_{\frac{t}\alpha}$ in terms of
$M_{t}$.
\subsubsection{Time-scaling of the Mehler kernel}
\begin{lemma}\label{lem:Kernel-estimates-1}
  Let $T > 0$, $\alpha \geq 1$, and $x, y \in \R^d$. Then:
  \begin{equation}
    \label{eq:Kernel-lemma-1-estimate} 
    M_{\frac{t}{\alpha}}(x, y) \leq \alpha^{\frac{d}2}
    \exp\biggl (\frac{t}{2\mu} |\la x, y \ra| \biggr)
   \exp\biggl(-\frac{\alpha}{4\e^{2T}} \frac{|x - y|^2}{1 - \e^{-t}}
   \biggr) M_{t}(x, y).
  \end{equation}
\end{lemma}
\begin{proof}
  To prove the lemma we compute $M_{\frac{t}{\alpha}} M_t^{-1}$.
  First recall that \eqref{eq:Time-part-Mehler-time-transform-1} gives
  \begin{equation*}
    \alpha \e^{-\frac{t}\mu} \leq \frac{1 - \e^{-t}}{1 -
      \e^{-\frac{t}{\alpha}}} \leq \alpha.
  \end{equation*}
  Combining the exponentials also gives,
  \begin{align*}
    \exp \biggl (-2\e^{-\frac{t}\alpha} \frac{\la x, y \ra}{1 + \e^{-\frac{t}{\alpha}}}
    \biggr ) &\exp \biggl (2 \e^{-t} \frac{\la x, y \ra}{1 + \e^{-t}}
    \biggr )\\
    &\stackrel{\phantom{\eqref{eq:Time-part-Mehler-time-transform-2}}}{=}
    \exp\biggl (\frac2{t}\biggl[\frac{\e^{-t}}{1 + \e^{-t}} - \frac{\e^{-\frac{t}\alpha}}{1 +
      \e^{-\frac{t}{\alpha}}} \biggr] t \la x, y \ra \biggr)\\
    &\stackrel{\eqref{eq:Time-part-Mehler-time-transform-2}}{\leq}
    \exp\biggl (\frac{t}{\mu} |\la x, y \ra| \biggr).
  \end{align*}
  Combining \autoref{lem:Exponential-estimates} and equation
  \eqref{eq:Exponential-estimates-1} almost gives the final estimate.
  \begin{align*}
    \frac{M_{\frac{t}{\alpha}}(x, y)}{M_{t}(x, y)} &\leq
    \alpha^{\frac{d}2} \exp\biggl (\frac{t}{\mu} |\la x, y \ra| \biggr)
    \exp\biggl(\e^{-2t} \dfrac{|x - y|^2}{1 - \e^{-2t}} \biggr)
    \exp\biggl(-\e^{-2\frac{t}{\alpha}} \dfrac{|x - y|^2}{1 -
      \e^{-2\frac{t}{\alpha}}}  \biggr)\\ 
    &\leq \alpha^{\frac{d}2} \exp\biggl (\frac{t}{\mu} |\la x, y \ra|
    \biggr) \exp \biggl (\biggl[1 -\frac{\alpha}{4\e^{2
        \frac{t}{\kappa}}} \biggr]  \frac{|x - y|^2}{1 - \e^{-t}}
    \biggr ) \exp \biggl (-\frac{\alpha}{4\e^{2 \frac{t}{\kappa}}} \frac{|x - y|^2}{1
      - \e^{-t}} \biggr). 
  \end{align*}
  Finally, we apply the assumption $\alpha \geq 4 \e^{2
    \frac{t}{\kappa}}$ to obtain: 
  \begin{equation*}
    \frac{M_{\frac{t}{\alpha}}(x, y)}{M_t(x, y)} \leq \alpha^{\frac{d}2}
   \exp\biggl (\frac{t}{\mu} |\la x, y \ra| \biggr)
   \exp\biggl(-\frac{\alpha}{4\e^{2 \frac{t}{\kappa}}} \frac{|x - y|^2}{1 - \e^{-t}}
   \biggr). 
 \end{equation*}
  Which is as asserted. The assumption $\alpha \geq 4 \e^{2
    \frac{t}{\kappa}}$ can be rephrased as the requirement that
  $\Bigl(1 + \frac1\alpha \Bigr)^{-1} \log\Bigl( \frac\alpha4 \Bigr) \geq t$.
\end{proof}
 

\subsection{An estimate on Gaussian balls}
\begin{lemma}\label{lem:Gaussian-ball-shift-lemma}
  Let $B_t(x)$ be the Euclidean ball with radius $t$ and center $x$
  and let $\gamma$ be the Gaussian measure
  \eqref{eq:Gaussian-measure}. We have the inequality:
  \begin{equation}\label{eq:Gaussian-ball-shift-lemma}
    \frac{\gamma(B_t(x))}{V_d(t)} \leq d \pi^{-\frac{d}2} \e^{-(t -
      |x|)^2}.
  \end{equation}
\end{lemma}
\begin{proof}
  Remark that
  \begin{align*}
    \int_B \e^{-|\xi|^2} \D\xi &= \e^{-|x|^2} \int_{B} \e^{-|\xi -
      x|^2} \e^{-2 \la x, \xi - x \ra} \D\xi\\
    &\leq \e^{-|x|^2} \int_{B} \e^{-|\xi - x|^2} \e^{2 |x| |\xi - x|}
    \D\xi\\
    &\leq \e^{-|x|^2} \e^{2 |x| t} \int_{B} \e^{-|\xi - x|^2} \D\xi\\
    &= \pi^{\frac{d}2} \e^{-|x|^2} \e^{2 t |x|} \gamma(B_t(0)).
  \end{align*}
  So, for a ball $B:= B_t(x)$ there holds that
  \begin{equation}\label{eq:Gaussian-ball-shift-lemma-proof-1}
    \gamma(B_t(x)) \leq \e^{-|x|^2} \e^{2 t |x|} \gamma(B_t(0)).
  \end{equation}
  We will estimate the Gaussian volume of the ball $B_t(0)$. To save
  writing, let $S_d$ and $V_d$ be the surface area and volume
  respectively of the $d$-dimensional unit sphere. Using polar coordinates
  we proceed by: 
  \begin{align*}
    \gamma(B_t(0)) &= \pi^{-\frac{d}2} \int_{B_t(0)} \e^{-|\xi|^2} \D\xi\\
    &= S_d \pi^{-\frac{d}2} \int_0^t \e^{-r^2} r^{d - 1} \D{r}\\
    &\leq S_d t^d \pi^{-\frac{d}2}\e^{-t^2}\\
    &= d V_d(t) \pi^{-\frac{d}2}\e^{-t^2}.
  \end{align*}
  Upon combining this result with
  \eqref{eq:Gaussian-ball-shift-lemma-proof-1} we obtain
  \eqref{eq:Gaussian-ball-shift-lemma}, which is as promised.
\end{proof}

\subsection{On-diagonal kernel estimates on annuli}
As is common in harmonic analysis, we often wish to decompose
$\R^d$ into sets on which certain phenomena are easier to handle. Here
we will decompose the space into disjoint annuli $C_k$. For the sake
of simplicity we will write $B := B_t(x)$ and mean that $B$ is the
closed ball with center $x$ and radius $t$. Furthermore, we use
notations such as $2B$ to mean the ball obtained from $B$ by
multiplying its radius by $2$.

The $C_k$ are given by,
\begin{equation}
  \label{eq:C_k-annulus-decomposition}
  C_k(B) := C_k = (2^{k + 1} - 1)B \setminus (2^k - 1)B.
\end{equation}
So, whenever $\xi$ is in $C_k(B_t(x))$, we get for $k \geq 0$:
\begin{equation}
  \label{eq:C_k-annulus-decomposition-expand}
  (2^k - 1) t < |y - \xi| \leq (2^{k + 1} - 1) t.
\end{equation}

\begin{lemma}\label{lem:On-diagonal-kernel-estimates-on-Ck}
  Given $A > 0$, let $B = B_{At}(y)$, $0 < t \leq T < \infty$ and $\xi \in C_k$. Then we have:
  \begin{equation*}
    M_{t^2}(y, \xi) \leq \frac{\e^{-\beta} \e^{|y|^2}}{(1 - \e^{-t^2})^{\frac{d}2}}
    \exp\bigl((2^{k + 1} - 1) A t |y| \bigr) \e^{\beta 2^{k + 1}} \e^{-\beta 4^k},
  \end{equation*}
  where $\beta = \frac{A^2}{2 \e^{2T}}$.
\end{lemma}
\begin{proof}
  Let $B = B_{At}(y)$ and let $C_k$ be as in
  \eqref{eq:C_k-annulus-decomposition}. Considering the first
  exponential which occurs in the Mehler kernel
  \eqref{eq:Mehler-kernel} together with
  \eqref{eq:C_k-annulus-decomposition-expand} gives for $k \geq 0$:
  \begin{align*}
    \exp\biggl(-\e^{-2t^2} \dfrac{|y - \xi|^2}{1 - \e^{-2t^2}} \biggr)
    &\leq \exp\biggl(-\e^{-2t^2} \dfrac{(2^k - 1)^2 A^2 t^2}{1 - \e^{-2t^2}} \biggr)\\
    &\overset{(\dagger)}{\leq} \exp\biggl(-\frac{A^2}{2 \e^{2t^2}} (2^k - 1)^2 \biggr).
  \end{align*}
  Where $(\dagger)$ follows from
  \begin{equation*}
    \frac{t}{1 - \e^{-2t}} \geq \frac12.
  \end{equation*}
  Before we consider the last exponential in the Mehler kernel we note
  that by Cauchy-Schwarz:
  \begin{equation}
    \label{lem:On-diagonal-kernel-estimates-on-Ck-proof-1}
    |\langle y, \xi \rangle| \leq |\la y - \xi, y \ra| + |\la y, y \ra|
    \leq |y - \xi||y| + |y|^2.
  \end{equation}
  Furthermore we have the estimate:
  \begin{equation*}
    \frac{\e^{-t}}{1 + \e^{-t}} \leq \frac12, 
  \end{equation*}
  Using these we get for the last exponential in the Mehler kernel
  \eqref{eq:Mehler-kernel} $M_{t^2}$:
  \begin{align*}
    \exp\biggl(2\e^{-t^2} \dfrac{\la y, \xi \ra}{1 + \e^{-t}}
    \biggr) &\leq \exp(|\la y, \xi \ra|)\\
    &\stackrel{\eqref{lem:On-diagonal-kernel-estimates-on-Ck-proof-1}}{\leq}
    \exp(|y - \xi||y|) \e^{|y|^2}.
  \end{align*}
  Wrapping it up, we can estimate the Mehler kernel
  \eqref{eq:Mehler-kernel} $M_t$ on $C_k$ from above by:
  \begin{equation*}
    M_{t^2}(y, \xi) \leq \frac{\e^{|y|^2}}{(1 - \e^{-{t^2}})^{\frac{d}2}}
    \exp\bigl((2^{k + 1} - 1) A t |y| \bigr) \exp\biggl(-\frac{A^2}{2 \e^{2t^2}} (2^k - 1)^2 \biggr).
  \end{equation*}
  Setting $\beta = \frac{A^2}{2 \e^{2T}}$ and expanding the last
  exponential we get:
  \begin{equation*}
    M_{t^2}(y, \xi) \leq \frac{\e^{-\beta} \e^{|y|^2}}{(1 - \e^{-{t^2}})^{\frac{d}2}}
    \exp\bigl((2^{k + 1} - 1) A t |y| \bigr) \e^{\beta 2^{k + 1}} \e^{-\beta 4^k}.
  \end{equation*}
  Which is as claimed.

  \autoref{lem:Cone-Gaussians-comparable} gives us
  by using $|x - y| \leq \alpha t \leq \alpha^2 m(x)$ the following estimate:
  \begin{equation*}
    \e^{|y|^2} \leq \e^{|x|^2}  \e^{\alpha^4} \e^{2\alpha^2}
  \end{equation*}
  \begin{align*}
    M_{t^2}(y, \xi) &\leq \frac{\e^{-\beta} \e^{|y|^2}}{(1 - \e^{-t^2})^{\frac{d}2}}
    \exp\bigl((2^{k + 1} - 1) \alpha (1 + \alpha) \bigr) \e^{\beta
      2^{k + 1}} \e^{-\beta 4^k}\\
    &\leq \e^{-(\alpha + \beta)} \e^{\alpha^4} \e^{\alpha^2} \frac{ \e^{|x|^2} }{(1 - \e^{-t^2})^{\frac{d}2}}
    \exp\bigl(2^{k + 1} \alpha (1 + \alpha) \bigr)  \e^{\beta 2^{k + 1}} \e^{-\beta 4^k}.
  \end{align*}
  Which is as claimed.
\end{proof}


\section{The boundedness of some non-tangential maximal operators}
Our theorem is a small modification of \cite[lemma 1.1]{Pineda2008} with a new proof.
\begin{theorem}\label{thm:Gaussian-maximal-function}
  Let $A, a > 0$. For all $x$ in $\R^d$ and all $u$ in $\LHG$ we have
  \begin{equation}
    \label{eq:Maximal-function-cone}
    \sup_{(y, t) \in \Gamma_x^{(A, a)}} |\e^{-t^2 L} u(y)| \lesssim
    \sup_{r > 0} \dashint_{B_r(x)} |u| \, \D\gamma.
  \end{equation}
\end{theorem}
\begin{proof}
  First we note that $\Gamma_x^{(A, a)} \subset \Gamma_x^{(1 + aA,
    aA)}$ as $a, A \geq 1$.

  \begin{align*}
    |x - y| \leq At \leq aA t\\
    t \leq a m(x) \leq aA m(x)
  \end{align*}
  So if $y \in \Gamma_x^{(A, a)}$ then $x \in \Gamma_y^{(aA, aA)}$. So
  set $\alpha = aA$ and $\Gamma_x^\alpha = \Gamma^{(\alpha, \alpha)}_x$


  We will prove \eqref{eq:Maximal-function-cone} by splitting up the
  integration domain in annuli.
  \begin{equation*}
    \e^{-t L} |u(y)| \leq \sum_{k = 0}^\infty I_k(y),
    \:\text{where}\: I_k(y) := \int_{C_k(B)} M_t(y, \cdot) |u|
    \,\D\gamma.
  \end{equation*}
  More precisely, we will set $B = B(y, aAt)$ in the above and find a
  suitable upper bound for each integral on the right-hand side which
  we will denote by $I_k$ for the sake of simplicity.


  \begin{equation*}
    M_t(y, \xi) \leq \frac{\e^{-\beta} \e^{|y|^2}}{(1 - \e^{-t})^{\frac{d}2}}
    \exp\bigl((2^{k + 1} - 1) \alpha t |y| \bigr) \e^{\beta 2^{k + 1}} \e^{-\beta 4^k},
  \end{equation*}
  where $\beta = \frac{\alpha^2}{2 \e^{2\alpha^2}}$.

  Since we have $|x - y| <  \alpha t$ and $t \leq a m(x)$ we infer that $t
  |x| \leq \alpha$. By \autoref{lem:m-xy-equivalence} we also have that $t
  |y| \leq 1 + \alpha$. From this and
  \autoref{lem:On-diagonal-kernel-estimates-on-Ck} we infer that:
  \begin{equation}
    \label{eq:Mehler-kernel-estimate-one-sided-bound-1}
    M_t(y, \xi) \leq \e^{-\beta} \e^{-\alpha(1 + \alpha)} \frac{\e^{|y|^2}}{(1 - \e^{-t})^{\frac{d}2}}
    \exp\bigl(2^{k + 1} \alpha (1 + \alpha) \bigr) \e^{\beta 2^{k + 1}} \e^{-\beta 4^k},
  \end{equation}

  Setting $\beta = \frac{\alpha^2}{2 \e^{2\alpha^2}}$. Note that $\beta$
  is maximal for $\alpha = \frac12$ and after this value, $\beta$ is decreasing.
  Setting $\lambda := \alpha(1 + \alpha)$ we get:
  \begin{equation}
    \label{eq:Mehler-kernel-estimate-one-sided-bound-2}
    M_t(y, \xi) \lesssim_\alpha  \e^{-(\alpha + \beta)} \e^{\alpha^4} \e^{\alpha^2} \frac{ \e^{|x|^2} }{(1 - \e^{-t})^{\frac{d}2}} \e^{(\lambda + \beta) 2^{k + 1}} \e^{-\beta 4^k}.
  \end{equation}
  Where the implied constant is given by $\e^{-(\alpha + \beta)} \e^{\alpha^4} \e^{\alpha^2}$

  Or, using $\Lambda = \beta + \lambda$ we get:
  \begin{equation}
    \label{eq:Mehler-kernel-estimate-one-sided-bound-2}
    M_t(y, \xi) \lesssim_\alpha \frac{\e^{|x|^2}}{(1
      - \e^{-t})^{\frac{d}2}} \e^{\Lambda 2^{k + 1}}
    \e^{-\beta 4^k},
  \end{equation}

  Recalling \autoref{lem:Gaussian-ball-shift-lemma} we get:
  \begin{equation}
    \label{eq:Gaussian-ball-Maximal-function-cone-proof-1}
    \gamma(B_t(x)) \leq V_d d \pi^{-\frac{d}2} t^d \e^{-(t - |x|)^2} .
  \end{equation}
  Where we abbreviate $V_d(1)$ with $V_d$.
  Recall
  \begin{equation*}
    V_d \leq \frac1{\sqrt{\pi}} \biggl(\frac{2\pi e}{d}
    \biggr)^{\frac{d}{2}}.
  \end{equation*}
  To get,
  \begin{equation}
    \label{eq:Gaussian-ball-Maximal-function-cone-proof-2}
    \gamma(B_t(x)) \leq \frac{d}{\sqrt{\pi}} \biggl(\frac{2 e}{d}
    \biggr)^{\frac{d}{2}} t^d \e^{-(t - |x|)^2} = C_d t^d \e^{-(t - |x|)^2}.
  \end{equation}
  This allows us to estimate the remaining unbounded exponential in the
  Mehler kernel and allow a penalty up to $\e^{-|x|^2}$. Furthermore,
  we have the following estimate which will make clear how to handle the
  time part in the Mehler kernel:
  \begin{equation*}
    \frac{t^d}{(1 - \e^{-t^2})^{\frac{d}2}} \leq \biggl(\frac{t^2}{1 -
      \e^{-t^2}} \biggr)^{\frac{d}2} \leq \frac{a^d}{(1 -
      \e^{-a^2})^{\frac{d}2}}.
  \end{equation*}
  Let $B' := B(x, 2^{k + 1} \alpha t)$ and $B$ as before the ball $B(y,
  \alpha t)$. In the next step we will bound ....
  by the maximal function centered at $x$. For this we need to scale
  up the $C_k$. So,
  \begin{equation*}
    |x - \xi| \leq |x - y| + |\xi - y| \leq \alpha t + (2^{k + 1} - 1)
    \alpha t = 2^{k + 1} \alpha t.
  \end{equation*}
  And set $D_k = B(2^{k + 1} \alpha t)$.
  So, we can bound the integral on the right-hand side of
  \eqref{eq:Maximal-function-cone-intermediate-step-1} by
  \begin{align*}
    \int_{C_k(B)}  M_{t^2}(y, \cdot) |u| \,\D\gamma & \lesssim_\alpha
    \frac{\e^{\Lambda 2^{k + 1}} \e^{-\beta 4^k}}{(1 -
      \e^{-t^2})^{\frac{d}2}}  \e^{|x|^2} \int_{C_k(B)} |u| \,\D\gamma\\
    &\leq \frac{\e^{\Lambda 2^{k + 1}} \e^{-\beta 4^k}}{(1 -
      \e^{-t^2})^{\frac{d}2}}  \e^{|x|^2} \int_{D_k(B)} |u| \,\D\gamma\\ 
    &\leq (M_\gamma u)(x) \frac{\e^{\Lambda 2^{k + 1}} \e^{-\beta
        4^k}}{(1 -  \e^{-t^2})^{\frac{d}2}}  \e^{|x|^2} \gamma(D_k)\\
    &\overset{(1)}{\leq} (M_\gamma u)(x) C_d \alpha^d 2^{d(k + 1)} t^d
    \frac{\e^{\Lambda 2^{k + 1}} \e^{-\beta 4^k}}{(1 -
      \e^{-t^2})^{\frac{d}2}}  \e^{|x|^2} \e^{-(t - |x|)^2}\\
    &\leq(M_\gamma u)(x)  2^{k d} \e^{\Lambda 2^{k + 1}} \e^{-\beta 4^k}
    \frac{t^d \e^{-t^2}}{(1 - \e^{-t^2})^{\frac{d}2}}  C_d \e^{2\alpha} (2\alpha)^d
  \end{align*}



  Where (1) uses \autoref{lem:Gaussian-ball-shift-lemma}
  and $t |x| \leq a$.



  We can then bound the maximal function:
  \begin{align*}
    \e^{-t^2 L} |u(y)| &= \sum_{k = 0}^\infty I_k\\
    &\leq (M_\gamma u)(x) C_{d, a, A} \sum_{k = 0}^\infty 2^{kd} \e^{\Lambda 2^{k + 1}} \e^{-\beta 4^k}
  \end{align*}
  Wrapping it up, we have that:
  \begin{equation*}
    \e^{-t^2 L} |u(y)| \lesssim \dashint_{B_r(x)} |u| \, \D\gamma.
  \end{equation*}
  With implied constant

  Which is what we wanted to prove.
  \begin{equation*}
    \sum_{k = 0}^\infty 2^{kd} \e^{-C 4^k} = \sum_{k = 0}^\infty x^{kd} \e^{-C x^{2k}}
  \end{equation*}
  Noting for $x \geq 1$ that $\exp(-C x^{2k}) \leq \exp(-C k x^2)$,
  thus,
  \begin{equation*}
    \sum_{k = 0}^\infty 2^{kd} \e^{-C 4^k} \leq \sum_{k = 0}^\infty
    x^{kd} (\e^{-C x^2})^k = \sum_{k = 0}^\infty (x^{d} \e^{-C x^2})^k
  \end{equation*}
  Here $x = 2$, so
  \begin{equation*}
    \sum_{k = 0}^\infty 2^{kd} \e^{-C 4^k} \leq \sum_{k = 0}^\infty
    (2^d \e^{-4 C})^k
  \end{equation*}
  If $2^d < \e^{4C}$, that is whenever $d \log 2 < 4C$, we can compute
  using the geometric series that
  \begin{equation*}
    \sum_{k = 0}^\infty 2^{kd} \e^{-C 4^k} \leq \frac1{1 - 2^d
      \e^{-4C}} = \frac{\e^{4C}}{\e^{4C} - 2^d}
  \end{equation*}


\end{proof}

\printbibliography

\end{document}


\section{The boundedness of some non-tangential maximal operators}
Our theorem is a small modification of \cite[lemma 1.1]{Pineda2008} with a new proof.
\begin{theorem}\label{lem:Maximal-function-cone}
  Let $A, a > 0$. For all $x$ in $\R^d$ and all $u$ in $\LHG$ we have
  \begin{equation}
    \label{eq:Maximal-function-cone}
    \sup_{(y, t) \in \Gamma_x^{(A, a)}} |\e^{-t^2 L} u(y)| \lesssim
    \sup_{r > 0} \dashint_{B_r(x)} |u| \, \D\gamma.
  \end{equation}
\end{theorem}
\begin{proof}
  We will prove \eqref{eq:Maximal-function-cone} by splitting up the
  integration domain in annuli.
  \begin{equation*}
    \e^{-t^2 L} |u(y)| \leq \sum_{k = 0}^\infty I_k(y),
    \:\text{where}\: I_k(y) := \int_{C_k(B)} M_{t^2}(y,
    \cdot) |u| \,\D\gamma.
  \end{equation*}
  More precisely, we will set $B = B(y, At)$ in the above and find a
  suitable upper bound for each integral on the right-hand side which
  we will denote by $I_k$ for the sake of simplicity.

  Since we have $|x - y| < At$ and $t \leq a m(x)$ we infer that $t
  |x| \leq a$. By \autoref{lem:m-xy-equivalence} we also that $t
  |y| \leq 1 + aA$. From this and
  \autoref{lem:On-diagonal-kernel-estimates-on-Ck} we infer that:
  \begin{equation}
    \label{eq:Mehler-kernel-estimate-one-sided-bound-1}
    M_{t^2}(y, \xi) \lesssim_\beta \frac{\e^{|y|^2}}{(1 - \e^{-t^2})^{\frac{d}2}}
    \exp\bigl((2^{k + 1} - 1) A (1 + aA) \bigr) \e^{\beta 2^{k + 1}} \e^{-\beta 4^k}.
  \end{equation}

  \begin{align*}
    M_{t^2}(y, \xi) &\lesssim_\beta C_{a, A} \frac{\e^{|y|^2}}{(1 - \e^{-t^2})^{\frac{d}2}}
    \exp\bigl(2^{k + 1}  A t |y| \bigr) \exp\bigl( -A (1 + aA) \bigr)
    \e^{A^2 2^k} \e^{-\beta 4^k}\\
    &\leq C_{a, A} \frac{\e^{|y|^2}}{(1 - \e^{-t^2})^{\frac{d}2}}
    \exp\bigl(2^{k + 1}  A t |y| \bigr) \e^{\beta 2^{k + 1}} \e^{-\beta 4^k}.
  \end{align*}
  Setting $\beta = \frac{A^2}{2 \e^{2T^2}}$.
  Setting $\alpha := A(1 + aA)$ we get:
  \begin{equation}
    \label{eq:Mehler-kernel-estimate-one-sided-bound}
    M_{t^2}(y, \xi) \leq \e^{-(\alpha + \beta)} \frac{\e^{|y|^2}}{(1 -
      \e^{-t^2})^{\frac{d}2}} \e^{(\alpha + \beta) (2^{k + 1} - 1)} \e^{-\beta 4^k}.
  \end{equation}
  Recalling \autoref{lem:Gaussian-ball-shift-lemma} we get using $t|x|
  \leq a$ that
  \begin{equation}
    \label{eq:Gaussian-ball-Maximal-function-cone-proof-1}
    \gamma(B_t(x)) \leq V_d d \pi^{-\frac{d}2} t^d \e^{-(t - |x|)^2} .
  \end{equation}
  Where we abbreviate $V_d(1)$ with $V_d$.
  Recall
  \begin{equation*}
    V_d \leq \frac1{\sqrt{\pi}} \biggl(\frac{2\pi e}{d}
    \biggr)^{\frac{d}{2}}.
  \end{equation*}
  To get,
  \begin{equation}
    \label{eq:Gaussian-ball-Maximal-function-cone-proof-2}
    \gamma(B_t(x)) \leq \frac{d}{\sqrt{\pi}} \biggl(\frac{2 e}{d}
    \biggr)^{\frac{d}{2}} t^d \e^{-(t - |x|)^2} = C_d t^d \e^{-(t - |x|)^2}.
  \end{equation}
  This allows us to estimate the remaining unbounded exponential in the
  Mehler kernel and allow a penalty up to $\e^{-|x|^2}$. Furthermore,
  we have the following estimate which will make clear how to handle the
  time part in the Mehler kernel:
  \begin{equation*}
    \frac{t^d}{(1 - \e^{-t^2})^{\frac{d}2}} \leq \biggl(\frac{t^2}{1 -
      \e^{-t^2}} \biggr)^{\frac{d}2} \leq \frac{a^d}{(1 -
      \e^{-a^2})^{\frac{d}2}}.
  \end{equation*}
  Let $B' := B(x, 2^{k + 1}At)$ and $B$ as before the ball $B(y,
  At)$. In the next step we will bound ....
  by the maximal function centered at $x$. For this we need to scale
  up the $C_k$. So,
  \begin{equation*}
    |x - \xi| \leq |x - y| + |\xi - y| \leq At + (2^{k + 1} - 1) At =
    2^{k + 1} A t.
  \end{equation*}
  And set $D_k = B(2^{k + 1} A t)$.
  So, we can bound the integral on the right-hand side of
  \eqref{eq:Maximal-function-cone-intermediate-step-1} by
  \begin{align*}
    \int_{C_k(B)}  M_{t^2}(y, \cdot) |u| \,\D\gamma &\leq \int_{C_k(B)}
    M_{t^2}(y, \cdot) |u| \,\D\gamma\\
    &\leq (M_\gamma u)(x) \gamma(B(x, 2^{k + 1} A t))\\
    &\overset{(1)}{\leq} (M_\gamma u)(x) C_d A^d 2^{d(k + 1)} t^d \e^{-(t - |x|)^2}.
  \end{align*}



  \begin{align*}
    \int_{C_k(B)}  |u| \,\D\gamma &\leq \int_{B'}  |u| \,\D\gamma\\
    &\leq (M_\gamma u)(x) \gamma(B(x, 2^{k + 1} A t))\\
    &\overset{(1)}{\leq} (M_\gamma u)(x) C_d A^d 2^{d(k + 1)} t^d \e^{-(t - |x|)^2}.
  \end{align*}
  Where (1) uses \autoref{lem:Gaussian-ball-shift-lemma}
  and $t |x| \leq a$. \autoref{lem:Cone-Gaussians-comparable} gives us
  by using $|x - y| \leq aA m(x)$ the following estimate:
  \begin{equation*}
    \e^{-|x|^2} \leq \e^{(aA)^2} \e^{2aA} \e^{-|y|^2}.
  \end{equation*}
  
  \begin{equation}
    \label{eq:Mehler-kernel-estimate-one-sided-bound}
    M_{t^2}(y, \xi) \leq \e^{-(\alpha + \beta)} \frac{\e^{|y|^2}}{(1 -
      \e^{-t^2})^{\frac{d}2}} \e^{(2\alpha + A^2) 2^k} \e^{-\beta 4^k}
    =: U_k(y)
  \end{equation}

  \begin{align*}
    \int_{C_k(B)} |u| \,\D\gamma &\leq \int_{B'} |u| \,\D\gamma\\
    &\leq \gamma(B(x, 2^{k + 1} A t)) (M_\gamma u)(x)\\
    &\overset{(1)}{\leq} d \e^{2^{k + 1} 2aA} 2^{d(k + 1)} V_d(A t)
    \gamma(x) (M_\gamma u)(x).
  \end{align*}



  \begin{equation*}
    I_k \leq t^d d V_d(A)
    \e^{2^{k + 1} 2aA}  2^{d \cdot 2^{k + 1} a A} \gamma(x) U_k(y) (M_\gamma u)(x).
  \end{equation*}

  \begin{align*}
    U_k(y) t^d \gamma(x) &= \frac{t^d}{(1 - \e^{-t^2})^{\frac{d}2}}
    \e^{-(\alpha + \beta)} \e^{|y|^2} \gamma(x) \e^{(2\alpha + A^2)
      2^k} \e^{-\beta 4^k}\\
    &\leq \frac{a^d}{(1 - \e^{-a^2})^{\frac{d}2}} \e^{-(\alpha +
      \beta)} \e^{|y|^2} \gamma(y) \e^{(aA)^2} \e^{2aA} \e^{(2\alpha + A^2) 2^k} \e^{-\beta 4^k}\\
    &\leq \frac{a^d \pi^{-\frac{d}2}}{(1 - \e^{-a^2})^{\frac{d}2}} 
    \e^{-(\alpha + \beta)} \e^{(aA)^2} \e^{2aA} \e^{(2\alpha + A^2) 2^k} \e^{-\beta 4^k} \\
  \end{align*}


  \begin{align*}
    I_k &\leq t^d d V_d(A) \e^{2^{k + 1} 2aA}  2^{d \cdot 2^{k + 1} a
      A} \gamma(x) U_k(y) (M_\gamma u)(x)\\
    &\leq (M_\gamma u)(x) \pi^{-\frac{d}2} \frac{d V_d(A) a^d}{(1 - \e^{-a^2})^{\frac{d}2}}
    \e^{-(\alpha + \beta)} \e^{(aA)^2} \e^{2aA} \e^{(2\alpha + A^2) 2^k}
    \e^{-\beta 4^k} \e^{2^{k + 1} 2aA}  2^{d \cdot 2^{k + 1} a A}\\
    &\leq (M_\gamma u)(x) C_d
    \e^{-(\alpha + \beta)} \e^{(aA)^2} \e^{2aA} \e^{(2\alpha + A^2) 2^k}
    \e^{-\beta 4^k} \e^{2^{k + 1} 2aA}  2^{d \cdot 2^{k + 1} a A}\\
    &\leq (M_\gamma u)(x) C_{d, a, A} \e^{(2\alpha + A^2) 2^k}
    \e^{-\beta 4^k} \e^{2^{k + 1} 2aA}  2^{d \cdot 2^{k + 1} a A}\\
  \end{align*}
  We can then bound the maximal function:
  \begin{align*}
    \e^{-t^2 L} |u(y)| &= \sum_{k = 0}^\infty I_k\\
    &\leq (M_\gamma u)(x) C_{d, a, A} \sum_{k = 0}^\infty  \e^{(2\alpha + A^2) 2^k}
    \e^{-\beta 4^k} \e^{2^{k + 1} 2aA}  2^{d \cdot 2^{k + 1} a A}
  \end{align*}
  Wrapping it up, we have that:
  \begin{equation*}
    \e^{-t^2 L} |u(y)| \lesssim \dashint_{B_r(x)} |u| \, \D\gamma.
  \end{equation*}
  With implied constant

  Which is what we wanted to prove.
\end{proof}



\RequirePackage[l2tabu, orthodox]{nag}
\documentclass[a4paper,oneside,10pt]{amsproc}

\usepackage{fixltx2e}
\usepackage[all, error]{onlyamsmath}
\usepackage{fixmath} % http://ctan.org/pkg/fixmath
\usepackage{refcheck}
% \usepackage{gitinfo} % Information about the version
\norefnames
% \showrefnames
\usepackage{microtype}
\usepackage{amsmath}
\usepackage{amsthm}
\usepackage[x11names]{xcolor}%
\usepackage{textcomp}
\usepackage[english]{babel}
\usepackage{xfrac} % Nice / fractions
\usepackage[utf8]{inputenc}
\usepackage[T1]{fontenc}
\usepackage[strict=true]{csquotes} % Needs to be loaded *after* inputenc

%% Tekstiin ja taulukoihin liittyvi� lis�paketteja
\usepackage{enumerate} 
\usepackage{booktabs}% Better tables

%% Matematiikkaan liittyvi� lis�paketteja

\usepackage{latexsym}
\usepackage{bbm}
\usepackage{enumitem}
\usepackage[bitstream-charter]{mathdesign}%

\usepackage[bookmarks,colorlinks,breaklinks]{hyperref} % Add hyperref type links in the document, colors
\definecolor{dullmagenta}{rgb}{0.4,0,0.4} % #660066
\definecolor{darkblue}{rgb}{0,0,0.4}%
\hypersetup{linkcolor=red,citecolor=blue,filecolor=dullmagenta,urlcolor=darkblue} % coloured links



\makeatletter

\@namedef{subjclassname@2010}{%

  \textup{2010} Mathematics Subject Classification}

\makeatother

% \linespread{1.3}


%%%%%%%% Teoreemaymp�rist�t
% HUOM! K�ytet��n amsthm-pakettia, eik� theorem-pakettia, jos
% theorem-pakettia k�ytet��n t�ytyy my�s n�it� muuttaa

% Perusteoreematyyli, lause ja lemma numeroidaan sectionin
% mukaan. Propositio numeroidaan lauseen numeroinnilla

\theoremstyle{plain}
% \newtheorem*{thmA6.6}{[A, Theorem 6.6]}
\newtheorem*{problem}{Problem}


\swapnumbers
\newtheorem{theorem}{Theorem}
\newtheorem{definition}{Definition}
\newtheorem{lemma}{Lemma}
\newtheorem{corollary}{Corollary}
\newtheorem{proposition}{Proposition}
\theoremstyle{remark}
\renewcommand{\qedsymbol}{\ensuremath{\blacksquare}}
\newtheorem*{remark}{Remark}
\newtheorem*{examples}{Examples}


% M��ritelm�tyyli

\theoremstyle{definition} 
% M��ritelm�, jolla on oma juokseva numerointi 
% \newtheorem{definition}[theorem]{Definition}
\newtheorem*{def*}{Definition}

% Huomautustyyli


% Omat k�skyt (esimerkkin� joukkosymbolit)

\newcommand{\E}{\mathbb{E}}
\newcommand{\Sp}{\mathbb{S}}
\newcommand{\prob}{\mathbb{P}}
\newcommand{\D}{\,\textup{d}}
\newcommand{\Dn}{\textup{d}} % One without space.
\newcommand{\Dt}{\,\frac{\textup{d} t}{t}}
\newcommand{\Ds}{\,\frac{\textup{d} s}{s}}
\newcommand{\DyDt}{\frac{\textup{d} y \, \textup{d} t}{t^{n+1}}}
\newcommand{\Dd}{\mathscr{D}}
\newcommand{\Tt}{\mathscr{T}}
\newcommand{\Cc}{\mathscr{C}}
\newcommand{\Bb}{\mathscr{B}}
\newcommand{\Rr}{\mathscr{R}}
\newcommand{\Ff}{\mathscr{F}}
\newcommand{\Ll}{\mathscr{L}}
\newcommand{\Mm}{\mathscr{M}}
\newcommand{\hh}{\mathfrak{h}}
\newcommand{\ttt}{\mathfrak{t}}
\newcommand{\la}{\langle}
\newcommand{\ra}{\rangle}
\newcommand{\Rad}{\textup{Rad}}
\newcommand{\Car}{\textup{Car}}
\newcommand{\BMO}{\textup{BMO}}
\newcommand{\loc}{\textup{loc}}
\newcommand{\LH}{{L^2_\mu}}
\newcommand{\LHG}{{L^2_\gamma}}

%% Bar int
\def\Xint#1{\mathchoice
  {\XXint\displaystyle\textstyle{#1}}%
  {\XXint\textstyle\scriptstyle{#1}}%
  {\XXint\scriptstyle\scriptscriptstyle{#1}}%
  {\XXint\scriptscriptstyle\scriptscriptstyle{#1}}%
  \!\int}
\def\XXint#1#2#3{{\setbox0=\hbox{$#1{#2#3}{\int}$ }
    \vcenter{\hbox{$#2#3$ }}\kern-.535\wd0}}
\def\ddashint{\Xint=}
\def\dashint{\Xint-}

\def\LI{{L^1_\gamma}}


%% Symbols
\renewcommand{\bar}[1]{\overline#1}
\renewcommand{\vec}[1]{\boldsymbol{\mathbf{#1}}}
\renewcommand{\leq}{\leqslant}
\renewcommand{\Im}{\operatorname{Im}}
\renewcommand{\Re}{\operatorname{Re}}
\renewcommand{\leq}{\leqslant}
\renewcommand{\geq}{\geqslant}
\renewcommand{\epsilon}{\varepsilon}
\renewcommand{\emptyset}{\varnothing}
\newcommand{\Fo}{\mathcal{F}}
\newcommand{\R}{\mathbf R}
\newcommand{\C}{\mathbf C}
\newcommand{\N}{\mathbf N}
\newcommand{\T}{\mathbb T}
\newcommand{\Z}{\mathbf Z}
\newcommand{\B}{\mathcal B}
\newcommand{\e}{\mathrm{e}} %Roman e for exponentials

\renewcommand{\leq}{\leqslant}%
\renewcommand{\geq}{\geqslant}%
\DeclareMathOperator{\supp}{supp}
\newcommand{\Dg}{\frac{\textup{d}\gamma (y)}{\gamma (B(y,t))}}
\newcommand{\Dmu}{\frac{\textup{d}\mu (y)}{\mu (B(y,t))}}


\renewcommand{\Re}{\operatorname{Re}}
\renewcommand{\Im}{\operatorname{Im}}
\renewcommand{\bar}{\overline}

\usepackage{tikz}

\def\lemmaautorefname{Lemma}
\def\definitionautorefname{Definition}
\def\theoremautorefname{Theorem}
\def\corollaryautorefname{Corollary}


%% Bibliography
\usepackage[backend=biber,doi=false,url=false,isbn=false]{biblatex}
\bibliography{~/Documents/BibTeX/library.bib}

\newbibmacro{string+doiurlisbn}[1]{%
  \iffieldundef{doi}{%
    \iffieldundef{url}{%
      \iffieldundef{isbn}{%
        \iffieldundef{issn}{%
          #1%
        }{%
          \href{http://books.google.com/books?vid=ISSN\thefield{issn}}{#1}%
        }%
      }{%
        \href{http://books.google.com/books?vid=ISBN\thefield{isbn}}{#1}%
      }%
    }{%
      \href{\thefield{url}}{#1}%
    }%
  }{%
    \href{http://dx.doi.org/\thefield{doi}}{#1}%
  }%
}

\DeclareFieldFormat{title}{\usebibmacro{string+doiurlisbn}{\mkbibemph{#1}}}
\DeclareFieldFormat[article,incollection]{title}%
{\usebibmacro{string+doiurlisbn}{\mkbibquote{#1}}}


\title[Gaussian estimates]{Gaussian estimates}
% \author{Mikko Kemppainen}
% \address{Department of Mathematics and Statistics, University of Helsinki,
% Gustaf H�llstr�min katu 2b, FI-00014 Helsinki, Finland}
% \email{mikko.k.kemppainen@helsinki.fi}

\author{Jonas Teuwen}%
\address{Delft Institute of Applied Mathematics,
  Delft University of Technology, P.O. Box 5031, 2600 GA Delft, The
  Netherlands} \email{j.j.b.teuwen@tudelft.nl}%
\urladdr{http://fa.its.tudelft.nl/~teuwen/}%
\thanks{}%
\date{\today}



\begin{document}

\begin{abstract}
  Maximal function! An attempt! A good attempt! I hope!
\end{abstract}

% \subjclass[2010]{42B25 (Primary); 46E40 (Secondary)}
% \keywords{R-bounds, dyadic cubes}

\maketitle
\section{The Mehler kernel and friends}
\subsection{Notation}
To begin, let us fix some notation. As is common, we use $N$ to
represent a positive integer. That is, $N \in \Z_+ = \{1, 2, 3,
\dots\}$. In the same way we cast letters that denote the number of
dimensions, e.g.\ $d$ in $\R^d$ as positive integers.

We use the capital letter $T$ to denote a ``time'' endpoint, for
instance, when writing $t$ in $(0, T]$.

\subsection{Setting}
Given the Ornstein-Uhlenbeck operator $L$ defined as:
\begin{equation}
  \label{eq:Ornstein-Uhlenbeck-operator}
  L = -\frac12 \Delta + x \cdot \nabla,
\end{equation}
We define the Mehler kernel (see e.g., \textcite{Sjogren1997}) as the
Schwartz kernel associated to the Ornstein-Uhlenbeck semigroup
$(\e^{-tL})_t$. More precisely, this means:
\begin{equation}
  \label{eq:Ornstein-Uhlenbeck-semigroup-integral}
  \e^{-tL} u(x) = \int_{\R^d} M_t(x, \cdot) u \, \D\gamma.
\end{equation}
It is often more convenient to use $\e^{-t^2 L}$ instead of $\e^{-tL}$
as is done in e.g., \textcite{Portal2012} and we will also do so.

\subsection{The Mehler kernel}
For the computation of the Mehler kernel in
\eqref{eq:Ornstein-Uhlenbeck-semigroup-integral} we refer to e.g.,
\textcite{Sjogren1997} which additionally offers related results such
as those concerning Hermite polynomials.

If one observes that the kernel $M_{t^2}$ is symmetric in its
arguments, a useful expression is:
\begin{equation}
  \label{eq:Mehler-kernel}
  M_{t^2}(x, y) = \frac{\exp\biggl(-\e^{-t^2} \dfrac{|x - y|^2}{1
      - \e^{-2 t^2}}  \biggr)}{(1 - \e^{-t^2})^{\frac{d}2}}
  \frac{\exp\biggl(\e^{-t^2} \dfrac{|x|^2 + |y|^2}{1 + \e^{-t^2}}
    \biggr)}{(1 + \e^{-t^2})^{\frac{d}2}}.
\end{equation}


\section{Some fine lemmata and definitions}
\subsection{$m$inimal function}
We recall the lemma from \cite[lemma 2.3]{Maas2011} which first,
--although implicitly-- appeared in \cite{Mauceri2007}. It will be
convenient to define a function $m$ as:
\begin{equation*}
  m(x) := \min\biggl\{1, \frac1{|x|} \biggr\} = 1 \vee \frac1{|x|}.
\end{equation*}
\begin{lemma}\label{lem:m-xy-equivalence-no-cone}
  Let $a, A$ be strictly positive numbers. We have for $x, y$
  in $\R^d$ that:
  \begin{enumerate}
  \item If $|x - y| < A t$ and $t \leq a m(x)$, then $t
    \leq (1 + aA) m(y)$;
  \item Likewise, if $|x - y| < A m(x)$, then $m(x) \leq (1 +
    A) m(y)$ and $m(y) \leq 2 (1 + A) m(x)$. 
  \end{enumerate}
\end{lemma}
We rewrite this lemma using the Gaussian cone $\Gamma_x^{(A, a)}$.
Recall that:
\begin{equation}
  \label{eq:Gaussian-cone}
  \Gamma_x^{(A, a)} := \Gamma_x^{(A, a)}(\gamma) := \{(y, t) \in
  \R^d_+ : |x - y| < At \:\text{and}\: t \leq a m(x)\}.
\end{equation}
We will also write $\Gamma_x^a$ to mean $\Gamma_x^{(1, a)}$. So we can
infer from \autoref{lem:m-xy-equivalence-no-cone} that:
\begin{lemma}\label{lem:m-xy-equivalence}
  Let $a, A$ be strictly positive numbers. Then:
  \begin{enumerate}
  \item If $(y, t) \in \Gamma_x^{(A, a)}$ then $t \leq (1 + aA) m(y)$;
  \item If $(y, t) \in \Gamma_x^{(A, a)}$ then  $(x, t) \in
    \Gamma_y^{(1 + aA, a)}$.
  \end{enumerate}
\end{lemma}
We will use a global/local region dichotomy which we define as
follows.
\begin{definition}
  Given $\tau > 0$, the set $N_\tau$ is given as:
  \begin{equation}
    \label{eq:Definition-local-region}
    N_\tau(x) := N_\tau := \{(x, y) \in \R^{2d} : |x - y| \leq \tau
    m(x) \}.
  \end{equation}
  Sometimes it is easier to work with the set $N_\tau(B)$, which is
  given for $B := B_r(y)$ as:
  \begin{equation}
    \label{eq:Definition-local-region-ball}
    N_\tau(B) := \{y \in \R^d : |x - y| \leq \tau m(x) \}.
  \end{equation}
  When we partition the space into $N_\tau$ and its complement, we
  call the part belonging to $N_\tau$ the \emph{local region} and the
  part belonging to $\complement N_\tau$ the \emph{global region}.
\end{definition}
The set $t \leq a m(x)$ is used in the definition of the cones
$\Gamma_x^{(A, a)}$ and we will name it $D^a$, that is:
\begin{equation}
  \label{eq:Definition-cut-off-set-D}
  D^a := \{(x, t) \in \R^d_+ : t \leq a m(x)\}.
\end{equation}
We will write $D := D^1$ for simplicity.

The next lemma will come useful when we want to cancel exponential
growth in one variable with exponential decay in the other as long
both variables are in a Gaussian cone.
\begin{lemma}\label{lem:Cone-Gaussians-comparable}
  Let $(y, t) \in \Gamma_x^{(A, a)}$. Then the Gaussians in $x$ and $y$
  respectively are comparable. In particular this means that,
  \begin{equation*}
    \e^{-|x|^2} \simeq \e^{-|y|^2}.
  \end{equation*}
\end{lemma}
\begin{remark}
  More precisely, from the proof we get by the  in estimates
  \eqref{eq:Cone-Gaussians-comparable}. That is:
  \begin{equation*}
    \e^{-|x|^2} \leq \e^{(1 + aA)^2 - 1} \e^{-|y|^2},
  \end{equation*}
  and,
  \begin{equation*}
    \e^{-|y|^2} \leq \e^{(1 + aA)^2} \e^{2(1 + aA)} \e^{-|x|^2}.
  \end{equation*}
\end{remark}

\begin{proof}
  Let $(y, t) \in \Gamma_x^{(A, a)}$. Unwrapping the definition we
  have
  \begin{equation*}
    |x - y| < A t \:\text{and}\: t \leq a m(x).
  \end{equation*}
  Hence, by the inverse triangle inequality we get,
  \begin{align*}
    |y|^2 &\leq (aA m(x) + |x|)^2\\
    &= (aA)^2 + 2 a A m(x) |x| + |x|^2\\
    &\leq (aA)^2 + 2 a A + |x|^2.
  \end{align*}
  Therefore,
  \begin{equation}
    \label{eq:Cone-Gaussians-comparable-1}
    \e^{-|y|^2} \geq \e^{-(aA)^2} \e^{-2 aA} \e^{-|x|^2}.
  \end{equation}
  By \autoref{lem:m-xy-equivalence-no-cone} we have $t \leq (1 + aA)
  m(y)$
  \begin{align*}
    |x|^2 &\leq ((1 + aA) m(y) + |y|)^2\\
    &= ((1 + aA) m(y))^2 + 2(1 + aA)m(y)|y| + |y|^2\\
    &\leq (1 + aA)^2 + 2(1 + aA) + |y|^2.
  \end{align*}
  Therefore,
  \begin{equation}
    \label{eq:Cone-Gaussians-comparable-2}
    \e^{-|x|^2} \geq \e^{-(1 + aA)^2} \e^{-2(1 + aA)} \e^{-|y|^2}.
  \end{equation}
  Summarizing we thus have that,
  \begin{equation*}
    \e^{-|x|^2} \simeq \e^{-|y|^2},
  \end{equation*}
  as required.
\end{proof}
\begin{lemma}
  \label{def:Global-region-cone-lemma}
  Let $x, y$ and $z$ in $\R^d$. Set
  \begin{equation*}
    \tau = \frac12 (1 + 2aA)(1 + aA).
  \end{equation*}
  If $|y - z| > \tau m(y)$ (i.e., $(y, z) \notin N_\tau$) and $(y, t)
  \in \Gamma_x^{(A, a)}$ then $|x - z| > \frac12 m(x)$ (i.e., $(x, z)
  \notin N_{\frac12}$).
\end{lemma}
\begin{proof}
  We assume that $(y, z) \notin N_\tau$ and $(y, t) \in \Gamma_x^{(A,
    a)}$. Written out this gives by \eqref{eq:Definition-local-region}
  the inequality $|y - z| > \tau m(y)$, and by
  \eqref{eq:Gaussian-cone} the inequality $|x - y| < aA m(x)$. Note
  that the latter inequality together with
  \autoref{lem:m-xy-equivalence-no-cone} yields,
  \begin{equation}
    \label{eq:Global-region-cone-lemma-proof-1}
    \frac12 \frac1{1 + aA} m(y) \leq m(x) \leq (1 + aA) m(y).
  \end{equation}
  Combining we get $|x - y| < aA (1 + aA) m(y)$. Now we are in
  position to apply the triangle inequality:
  \begin{equation*}
    |x - z| \geq |y - z| - |x - y| > \tau m(y) - aA (1 + aA) m(y).
  \end{equation*}
  As we require an lower bound in terms of $m(x)$ and not $m(y)$, we
  again apply \eqref{eq:Global-region-cone-lemma-proof-1} to obtain:
  \begin{align*}
    |x - z| \geq |y - z| - |x - y| &> \tau m(y) - aA m(y)\\
    &\geq \tau \frac1{1 + aA} m(x) - aA m(x)\\
    &\geq \frac12 m(x).
  \end{align*}
  So we are done.
\end{proof}


\section{On-diagonal estimates}
\subsection{Kernel estimates}
We begin with a technical lemma which will be useful on several
occasions.
\begin{lemma}\label{lem:Time-part-Mehler-time-transform}
  Let $t$ in $(0, T]$ and $\alpha > 1$. Then,
  \begin{equation}\label{eq:Time-part-Mehler-time-transform}
    \alpha \e^{-T^2} \leq \frac{1 - \e^{-t^2}}{1 -
      \e^{-\frac{t^2}{\alpha}}} \leq \alpha.
  \end{equation}
\end{lemma}
\begin{proof}
  Let $t$ in $(0, T]$ and $\alpha > 1$. Applying the mean value theorem to the function $f(\xi) = \xi^\alpha$ gives, for $0 < \xi < \xi'$:
  \begin{equation*}
    f(\xi) - f(\xi') = \alpha \hat{\xi}^{\alpha - 1} (\xi - \xi') \text{ for some $\hat \xi$ in $[\xi, \xi']$}.
  \end{equation*}
  Picking $\xi = 1$ and $\xi' = \e^{-\frac{t^2}{\alpha}}$ gives:
  \begin{equation}
    \label{eq:Time-part-Mehler-time-transform-proof-1}
    \frac{1 - \e^{-t^2}}{1 - \e^{-\frac{t^2}{\alpha}}} = \alpha
    \hat{\xi}^{\alpha - 1} \:\text{for some}\: \hat{\xi} \:\text{in}\:
    [\e^{-\frac{t^2}{\alpha}}, 1].
  \end{equation}
  Applying this result together with the monotonicity of $\xi \mapsto
  \alpha \xi^{\alpha - 1}$ we get:
  \begin{equation*}
    \alpha \e^{-T^2} \leq \alpha \e^{-t^2} \leq \alpha \exp\biggl(-t^2 \frac{\alpha -
      1}{\alpha} \biggr) \leq \frac{1 - \e^{-t^2}}{1 -
      \e^{-\frac{t^2}{\alpha}}}.
  \end{equation*}
  Hence,
  \begin{equation*}
    \alpha \e^{-T^2} \leq \frac{1 - \e^{-t^2}}{1 - \e^{-\frac{t^2}{\alpha}}} \downarrow \alpha.
  \end{equation*}
  Which completes the proof.
\end{proof}

The following lemma will be useful when transfering estimates from
$M_{\frac{t^2}{\alpha}}$ to $M_{t^2}$. It follows from the mean value
theorem applied to $\xi \mapsto \xi^\alpha$.
\begin{lemma}\label{lem:Exponential-estimates}
  For $C, T > 0, \alpha > 1, t$ in $(0, T]$ and all $x, y$ in $\R^d$ we have that
  \begin{equation}
    \label{eq:Exponential-estimates-1}
    \exp \biggl (-C \frac{|x - y|^2}{1 - \e^{-\frac{t^2}\alpha}}
    \biggr ) \leq  \exp \biggl (-C \frac{\alpha}{\e^{T^2}} \frac{|x - y|^2}{1 -
      \e^{-t^2}} \biggr ).
  \end{equation}
\end{lemma}
\begin{proof}
  Let $t$ in $(0, T]$. Applying
  \autoref{lem:Time-part-Mehler-time-transform} we get:
  \begin{align*}
    \exp \biggl (-C \frac{|x - y|^2}{1 - \e^{-\frac{t^2}\alpha}} \biggr )
    \leq \exp \biggl (-C \frac{|x - y|^2}{1 - \e^{-t^2}} \frac{1 -
      \e^{-t^2}}{1 - \e^{-\frac{t^2}\alpha}} \biggr ) \leq \exp \biggl
    (-C \frac{\alpha}{\e^{T^2}} \frac{|x - y|^2}{1 - \e^{-t^2}} \biggr ).
  \end{align*}
  Which is as asserted.
\end{proof}
Later on we will study kernel estimates of kernels related to the
Mehler kernel, but our first lemma is about estimating
$M_{\frac{t}\alpha}$ in terms of $M_t$.
\begin{lemma}\label{lem:Kernel-estimates-1}
  Let $\alpha \geq 2 \e^{T^2}$, $t$ in $(0, T]$ and $x, y$ in $\R^d$.
  If $t |x| \lesssim 1$ and $t |y| \lesssim 1$ then:
  \begin{equation}
    \label{eq:Kernel-lemma-1-estimate} 
    M_{\frac{t^2}{\alpha}}(x, y) \lesssim \exp\biggl (-\frac{\alpha}{2
      \e^{T^2}} \frac{|x - y|^2}{1 - \e^{-t^2}} \biggr ) M_{t^2}(x,
    y),
  \end{equation}
  where the implied constant does not depend on $x, y$ and $t$.
\end{lemma}
\begin{remark}
  If $C$ is the positive constant such that $t |x| \leq C$ and $t |y|
  \leq C$ then the proof gives that the implied constant is bounded
  from above by $\alpha^{\frac{d}2} \e^{\frac{\alpha}2 C^2}$.
\end{remark}
\begin{proof}
  To prove the lemma we compute $M_{\frac{t^2}{\alpha}} M_{t^2}^{-1}$.
  First note that dividing the time-parts by
  \eqref{eq:Time-part-Mehler-time-transform} gives the upper-bound
  $\alpha^{\frac{d}2}$. Furthermore, we can verify that  
  \begin{equation*}
    \frac{1}{1 + \e^{-t^2}} - \frac{1}{1 + \e^{-\frac{t^2}{\alpha}}}
    \geq 0.
  \end{equation*}
  Also,
  \begin{align*}
    \exp \biggl (-\frac12 &\frac{|x + y|^2}{1 + \e^{-\frac{t^2}{\alpha}}}
    \biggr ) \exp \biggl (\frac12 \frac{|x + y|^2}{1 + \e^{-t^2}}
    \biggr )\\
    &\leq \exp \biggl (\frac12 \biggl[\frac{1}{1 +
      \e^{-t^2}} - \frac{1}{1 + \e^{-\frac{t^2}{\alpha}}}
    \biggr] |x + y|^2 \biggr)\\
    &= \exp \biggl (\frac12 \frac1{t^2}\biggl[\frac{1}{1 +
      \e^{-t^2}} - \frac{1}{1 + \e^{-\frac{t^2}{\alpha}}}
    \biggr] t^2 |x + y|^2 \biggr).
  \end{align*}
  Next, as the inner most function is decreasing,
  \begin{align*}
    \lim_{t \to 0} \frac1{t^2}\biggl[\frac{1}{1 +
      \e^{-t^2}} - \frac{1}{1 + \e^{-\frac{t^2}{\alpha}}} \biggr] 
    &= \lim_{t \to 0} \frac1{2t} \biggl[\frac{2t \e^{-t^2}}{(1 +
      \e^{-t^2})^2} - \frac1\alpha \frac{2t
      \e^{-\frac{t^2}{\alpha}}}{(1 + \e^{-\frac{t^2}{\alpha}})^2}
    \biggr]\\ 
    &= \lim_{t \to 0} \biggl[\frac{\e^{-t^2}}{(1 + \e^{-t^2})^2} -
    \frac1\alpha \frac{\e^{-\frac{t^2}{\alpha}}}{(1 +
      \e^{-\frac{t^2}{\alpha}})^2} \biggr]\\
    &\uparrow \frac{1}{4} \biggl(1 - \frac1\alpha \biggr).
  \end{align*}
  So that
  \begin{align*}
    \exp \biggl (-\frac12 &\frac{|x + y|^2}{1 + \e^{-\frac{t^2}{\alpha}}}
    \biggr ) \exp \biggl (\frac12 \frac{|x + y|^2}{1 + \e^{-t^2}}
    \biggr )\\
    &\leq \exp \biggl (\frac18 \biggl(1 - \frac1\alpha \biggr) t^2 |x +
    y|^2 \biggr)\\
    &\leq \exp \biggl (\frac14 t^2 |x|^2 \biggr) \exp \biggl (\frac14
    t^2 |y|^2 \biggr).
  \end{align*}
  So using \autoref{lem:Exponential-estimates} and equation
  \eqref{eq:Exponential-estimates-1} we get
  \begin{align*}
    \frac{M_{\frac{t^2}{\alpha}}(x, y)}{M_{t^2}(x, y)} &\leq
    \alpha^{\frac{d}2}  \exp\biggl(\dfrac12 \dfrac{|x - y|^2}{1 -
      \e^{-t^2}}  \biggr) \exp\biggl(\dfrac12 \dfrac{|x + y|^2}{1 +
      \e^{t^2}} \biggr)\\ 
    &\quad \times \exp\biggl(-\dfrac12 \dfrac{|x - y|^2}{1
      - \e^{-\frac{t^2}{\alpha}}}  \biggr) \exp\biggl(-\dfrac12
    \dfrac{|x + y|^2}{1 + \e^{-\frac{t^2}{\alpha}}} \biggr)\\
    &\leq \alpha^{\frac{d}2} \exp \biggl (\frac12 \biggl[1
    -\frac{\alpha}{2\e^{T^2}} \biggr] \frac{|x - y|^2}{1 - \e^{-t^2}}
    \biggr ) \exp \biggl (-\frac{\alpha}{2\e^{T^2}} \frac{|x - y|^2}{1
      - \e^{-t^2}} \biggr )\\
    &\quad \times  \exp\biggl(\dfrac12 \dfrac{|x + y|^2}{1 + \e^{t^2}}
    \biggr) \exp\biggl(-\dfrac12 \dfrac{|x + y|^2}{1 +
      \e^{-\frac{t^2}{\alpha}}} \biggr).
  \end{align*}
  Thus that,
  \begin{align*}
    \frac{M_{\frac{t^2}{\alpha}}(x, y)}{M_{t^2}(x, y)} &\leq
    \alpha^{\frac{d}2} \exp \biggl (\frac12 \biggl[1 -
    \frac{\alpha}{2\e^{T^2}} \biggr] \frac{|x - y|^2}{1 - \e^{-t^2}}
    \biggr ) \exp \biggl (-\frac{\alpha}{2\e^{T^2}} \frac{|x - y|^2}{1
      - \e^{-t^2}}  \biggr )\\ 
    &\quad \times \exp \biggl (t^2 \frac{|x|^2 + |y|^2}4 \biggr).
  \end{align*}
  For $\alpha \geq 2 \e^{T^2}$ we then obtain:
  \begin{equation*}
    \frac{M_{\frac{t^2}{\alpha}}(x, y)}{M_{t^2}(x, y)} \leq
    \alpha^{\frac{d}2} \exp\biggl(-\frac{\alpha}{2\e^{T^2}} \frac{|x -
      y|^2}{1 - \e^{-t^2}} \biggr) \exp \biggl (\frac14 t^2 |x|^2 \biggr)
    \exp \biggl (\frac14 t^2 |y|^2 \biggr).
  \end{equation*}
  From $t |x| \lesssim 1$ and $t |y| \lesssim 1$ we infer that there
  exists a positive constant $C$ such that $t |x| \leq C$ and $t |y|
  \leq C$.
  \begin{equation*}
    \frac{M_{\frac{t^2}{\alpha}}(x, y)}{M_{t^2}(x, y)} \leq \alpha^{\frac{d}2}
    \e^{\frac{C^2}2} \exp\biggl(-\frac{\alpha}{2\e^{T^2}} \frac{|x -
      y|^2}{1 - \e^{-t^2}} \biggr).
  \end{equation*}
  Which is as asserted.
\end{proof}
\begin{remark}
  More precisely we have the estimate:
  \begin{align*}
    \exp \biggl (-\frac12 &\frac{|x + y|^2}{1 + \e^{-\frac{t^2}{\alpha}}}
    \biggr ) \exp \biggl (\frac12 \frac{|x + y|^2}{1 + \e^{-t^2}}
    \biggr )\\
    &\leq \exp \biggl (\frac14 t^2 |x|^2 \biggr) \exp \biggl (\frac14
    t^2 |y|^2 \biggr) \exp \biggl (-\frac{t^2}\alpha \frac18 |x + y|^2
    \biggr).
  \end{align*}
  Which then produces:
  \begin{equation*}
    \frac{M_{\frac{t^2}{\alpha}}(x, y)}{M_{t^2}(x, y)} \leq \alpha^{\frac{d}2}
    \e^{\frac{C^2}2} \exp\biggl(-\frac{\alpha}{2\e^{T^2}} \frac{|x - y|^2}{1 - \e^{-t^2}} \biggr)
    \exp\biggl(-\frac{t^2}\alpha \frac{|x + y|^2}8 \biggr).
  \end{equation*}
\end{remark}

\subsection{On-diagonal kernel estimates on annuli}
As is common in harmonic analysis, we often wish to decompose
$\R^d$ into sets on which certain phenomena are easier to handle. Thus
we will decompose the space into annuli $C_k$. We will write $B :=
B_t(x)$ and assume that $B$ is the closed ball with center $x$ and
radius $t$. Recall that $2B$ is the ball obtain from $B$ by
multiplying its radius by $2$.

The $C_k$ are given by,
\begin{equation}
  \label{eq:C_k-annulus-decomposition}
  C_k(B) := C_k = (2^{k + 1} - 1)B \setminus (2^k - 1)B.
  \begin{cases}
    2B &\text{if $k = 0$,}\\
    2^{k + 1}B \setminus 2^k B &\text{for $k \geq 1$.}
  \end{cases}
\end{equation}
So, whenever $\xi$ is in $C_k$, we get for $k \geq 1$:
\begin{equation}
  \label{eq:C_k-annulus-decomposition-expand-nonzero}
  2^k a t < |y - \xi| \leq 2^{k + 1} a t.
\end{equation}
While we get for $k = 0$:
\begin{equation}
  \label{eq:C_k-annulus-decomposition-expand-zero}
  |y - \xi| \leq 2 a t.
\end{equation}

\begin{lemma}\label{lem:On-diagonal-kernel-estimates-on-Ck}
  Given $a > 0$, let $B = B_{at}(y)$ and $\xi$ in $C_k$. Furthermore,
  assume that $t \leq aA m(y)$ for some $A > 0$. Then we have
  for $k \geq 1$:
  \begin{equation*}
    M_{t^2}(y, \xi) \leq \frac{\e^{|y|^2}}{(1 - \e^{-t^2})^{\frac{d}2}}
    \exp\biggl(-\frac{a^2}{2} 4^{k + 1} \biggr) \exp\bigl(2^{k + 1} a t |y|
    \bigr).
  \end{equation*}
  and for $k = 0$:
  \begin{equation*}
    M_{t^2}(y, \xi) \leq \frac{\e^{|y|^2}}{(1 -
      \e^{-t^2})^{\frac{d}2}} \exp\bigl(2^{k + 1} a t |y| \bigr).
  \end{equation*}
\end{lemma}
\begin{proof}
  Let $B = B_{at}(y)$ and let $C_k$ be as in
  \eqref{eq:C_k-annulus-decomposition}. We consider the two
  exponentials in the Mehler kernel \eqref{eq:Mehler-kernel}
  separately. First we consider  
  \begin{equation*}
    \exp\biggl(\e^{-t^2} \dfrac{|y|^2 + |\xi|^2}{1 + \e^{-t^2}}
    \biggr).
  \end{equation*}
  Using the triangle inequality we note that:
  \begin{equation}
    \label{lem:On-diagonal-kernel-estimates-on-Ck-proof-1}
    |\xi|^2 \leq |y - \xi|^2 + |y|^2 + 2 |y - \xi||y|.
  \end{equation}
  Next, note that
  \begin{equation*}
    \frac{\e^{-t^2}}{1 + \e^{-t^2}} \leq \frac12.
  \end{equation*}
  Together with \eqref{lem:On-diagonal-kernel-estimates-on-Ck-proof-1}
  this gives for $k \geq 1$:
  \begin{align*}
    \exp\biggl(\e^{-t^2} \dfrac{|y|^2 + |\xi|^2}{1 + \e^{-t^2}}
    \biggr) &\leq \exp\biggl(\e^{-t^2} \dfrac{|y - \xi|^2}{1 +
      \e^{-t^2}} \biggr) \exp(|y - \xi||y|) \exp(|y|^2)\\
    &\overset{(1)}{\leq} \exp\biggl(\e^{-t^2} \dfrac{|y - \xi|^2}{1 +
      \e^{-t^2}} \biggr) \exp( 2^{k + 1} a t |y|) \exp(|y|^2)
  \end{align*}
  Where (1) uses \eqref{eq:C_k-annulus-decomposition-expand-nonzero}
  or \eqref{eq:C_k-annulus-decomposition-expand-zero}. Next we
  consider the exponential:
  \begin{equation*}
    \exp\biggl(\e^{-t^2} \dfrac{|y - \xi|^2}{1 + \e^{-t^2}} \biggr).
  \end{equation*}
  Combining this with the first exponential in the Mehler kernel
  \eqref{eq:Mehler-kernel} we get:
  \begin{align*}
    \exp\biggl(-\e^{-t^2} \dfrac{|y - \xi|^2}{1 - \e^{-2t^2}} & \biggr)
    \exp\biggl(\e^{-t^2} \dfrac{|y - \xi|^2}{1 + \e^{-t^2}}  \biggr)\\
    &\leq \exp\biggl(-\e^{-t^2}\frac{|y - \xi|^2}{1 + \e^{-t^2}}
    \biggl[\dfrac{1}{1 - \e^{-t^2}} - 1 \biggr] \biggr)\\
    &\leq \exp\biggl(-\e^{-t^2}\frac{|y - \xi|^2}{1 + \e^{-t^2}}
    \biggl[\frac{1}{1 - \e^{-t^2}} - \dfrac{1 - \e^{-t^2}}{1 -
      \e^{-t^2}} \biggr] \biggr)\\
    &\leq \exp\biggl(-\e^{-2t^2}\frac{|y - \xi|^2}{1 - \e^{-2t^2}}
    \biggr).
  \end{align*}
  Using \eqref{eq:C_k-annulus-decomposition-expand-nonzero}
  or \eqref{eq:C_k-annulus-decomposition-expand-zero} we get:
  \begin{equation*}
    \exp\biggl(-\e^{-2t^2}\frac{|y - \xi|^2}{1 - \e^{-2t^2}} \biggr) \leq
    \begin{cases}
      1 &\text{if $k = 0$,}\\
      \exp\biggl(-\dfrac{a^2}{2 \e^{2t^2}} 4^{k + 1} \biggr) &\text{if $k
        \geq 1$.}
    \end{cases}
  \end{equation*}
  Using the assumption that $t \leq aA m(y)$ gives that
  \begin{equation*}
    \exp\biggl(-\dfrac{a^2}{2 \e^{2t^2}} 4^{k + 1} \biggr) \leq
    \exp\biggl(-\dfrac{a^2}{2 \e^{2a^2 A^2}} 4^{k + 1} \biggr).
  \end{equation*}
  Combining we get for the Mehler kernel \eqref{eq:Mehler-kernel}:
  \begin{align*}
    M_{t^2}(y, \xi) &\leq \frac{\e^{|y|^2}}{(1 -
      \e^{-t^2})^{\frac{d}2}} \exp(2^{k + 1} a t |y|)\\
    &\leq \frac{\e^{|y|^2}}{(1 -
      \e^{-t^2})^{\frac{d}2}} \exp(2^{k + 1} a^2 A).
  \end{align*}

  This inequality together with
  \begin{equation*}
    \frac{t^2}{1 - \e^{-2t^2}} \geq \frac12,
  \end{equation*}
  yields,
  \begin{equation*}
    \exp\biggl(-\e^{-t^2} \dfrac{|y - \xi|^2}{1 - \e^{-2t^2}}  \biggr)
    \exp\biggl(-\e^{-t^2} \dfrac{|y - \xi|^2}{1 + \e^{-t^2}}  \biggr)
    \leq \exp\biggl(-\dfrac{a^2}{2 \e^{2 t^2}} 4^{k + 1} \biggr).
  \end{equation*}
  Thus, we can estimate the Mehler kernel $M_{t^2}$ on $C_k$ for $k
  \geq 1$ from above by:
  \begin{equation*}
    M_{t^2}(y, \xi) \leq \frac{\e^{|y|^2}}{(1 - \e^{-t^2})^{\frac{d}2}}
    \exp\biggl(-\frac{a^2}{2} 4^{k + 1} \biggr) \exp\bigl(2^{k + 1} a t |y|
    \bigr).
  \end{equation*}
  We are left with the case $k = 0$, which can be done similarly and
  yields:
  \begin{equation*}
    M_{t^2}(y, \xi) \leq \frac{\e^{|y|^2}}{(1 - \e^{-t^2})^{\frac{d}2}}
    \exp\bigl(2^{k + 1} a t |y| \bigr).
  \end{equation*}
  Done.
\end{proof}

\subsection{The Ornstein-Uhlenbeck maximal function}


\printbibliography

\end{document}

